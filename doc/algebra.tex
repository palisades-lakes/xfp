%-----------------------------------------------------------------
\setcounter{currentlevel}{\value{baseSectionLevel}}
\levelstay{Algebraic structures}
\label{sec:Algebraic-structures}

\textbf{TODO:} redo as Universal Algebras?
Implies more operations, typically binary, unary, and nullary
(nullary function equivalent to element of set).

An algebraic 
structure~\cite{
wiki:Algebraic-structure,
wiki:Mathematical-structure,
wiki:Outline-of-algebraic-structures,
BWilliams:Algebraic-structure-and-protocols,
BWilliams:Algebra-of-predicates-and-sorting-functions,
BWilliams:Semirings-and-predicates}
consists of:
\begin{itemize}
  \item A primary set --- the \textit{elements} of the structure.
  \item zero or a few auxilliary sets.
  \item functions (called \textit{operations})
that take a small number of arguments from one or more of the sets
and return elements of the sets.
\end{itemize}

The type of an algebraic structure corresponds to identities
that the operations satisfy.
(\textbf{TODO:} Examples of operation identities?)

Unfortunately, the names for algebraic structures 
are, as a rule, not very informative.

%-----------------------------------------------------------------
\setcounter{currentlevel}{\value{baseSectionLevel}-1}
\levelstay{One set, one operation}
%-----------------------------------------------------------------
\setcounter{currentlevel}{\value{baseSectionLevel}-2}
\levelstay{Monoid}
\label{sec:Monoid}

Set $\Set{S}$ and operation $\diamond$ such that,
if $a, b, c \in \Set{S}$, then
\begin{description}
\item[Closed] $a \diamond b = \diamond \left( a, b \right) 
= \left( \diamond \; a \; b \right) \in \Set{S}$
\item[Associative] $a \diamond b \diamond c =
 \left( a \diamond b \right) \diamond c =  
 a \diamond \left( b \diamond c \right) $
 \item[Identity] There is an $i \in \Set{S}$ such that 
 $i \diamond a = a \; \forall a \in \Set{S}$.
 Exercise: show that $i$ is unique.
 \end{description}

Example: 

$\Set{S}$ the functions from some domain $\Set{X}$ to itself. 
Operation $\diamond$ is function composition:
$\left( f \diamond g \right) (x) = 
f \left( g \left( x \right) \right)$.


%-----------------------------------------------------------------
\setcounter{currentlevel}{\value{baseSectionLevel}-2}
\levelstay{Group}

Monoid $\left[ \Set{S}, \diamond \right]$ such that
\begin{description}
 \item[Inverse] For every $a \in \Set{S}$ there exists an
 $a^{-1}$ such that $a^{-1} \diamond a = i$.
 Exercise: show that this implies that $a \diamond a^{-1} = i$ .
\end{description}

%-----------------------------------------------------------------
\setcounter{currentlevel}{\value{baseSectionLevel}-2}
\levelstay{Commutative group}

(aka abelian group.)

A group $\left[ \Set{S}, \diamond \right]$ where
\begin{description}
 \item[Commutative] $a \diamond b = b \diamond a \; \forall a,b \in \Set{S}$.
\end{description}

Example: 
$\left[ \glssymbol{Integers}, *_{\glssymbol{Integers}} \right]$

%-----------------------------------------------------------------
\setcounter{currentlevel}{\value{baseSectionLevel}-1}
\levelstay{One set, two operations}
%-----------------------------------------------------------------
\setcounter{currentlevel}{\value{baseSectionLevel}-2}
\levelstay{Semiring}
\label{sec:Semiring}

A set and $2$ operations: $\left[ \Set{S}, +, * \right]$
where
\begin{description}
  \item[Addition] $+$ is commutative, associative, and has an 
  identity element ($0$):
  $a + b = b + a$; 
  $a + \left( b + c \right) =\left( a + b  \right) + c $;
   and $ 0 + a = a$, for all $a,b,c \in \Set{S}$.
  \item[Multiplication] $*$ is associative, and has an 
  identity element ($1$):
 $a * \left( b * c \right) =\left( a * b \right) * c $
  and $ 1 * a = a$, for all $a,b,c \in \Set{S}$.]
  \item[Distributive] $a * \left( b + c \right) 
  = \left( a * b \right) + \left( a * c \right)$
\end{description}

Example: 

The \textit{\gls{NaturalNumbers}}: 
$\glssymbol{NaturalNumbers} = \glsdesc{NaturalNumbers}$.
(sec~\ref{sec:Natural-numbers}).

%-----------------------------------------------------------------
\setcounter{currentlevel}{\value{baseSectionLevel}-2}
\levelstay{Ring}
\label{sec:Ring}
\cite{wiki:Ring-mathematics}

A set and $2$ operations: $\left[ \Set{S}, +, * \right]$
where
\begin{description}
  \item[Addition group] $\left[ \Set{S}, + \right]$ 
  is a commutative group (with the identity written $0$).
  \item[Multiplication monoid] $\left[ \Set{S}, * \right]$ 
  is a monoid (with the identity written $1$).
  Note this means $*$ may not commute. If it does, 
  then this is a \textit{commutative ring}.
  \item[Distributive] $a * \left( b + c \right) 
  = \left( a * b \right) + \left( a * c \right)$
\end{description}

Example:

The \textit{\gls{Integers}}: 
$\glssymbol{Integers} = \{\cdots, -2, -1, 0, 1, 2, \cdots\}$.

%-----------------------------------------------------------------
\setcounter{currentlevel}{\value{baseSectionLevel}-2}
\levelstay{Field}
\label{sec:Field}
\cite{wiki:Field-mathematics}

A ring $\left[ \Set{S}, +, * \right]$ where
$*$ is commutative and the nonzero elements of $\Set{S}$
have multiplicative inverses.

Without refering to other structures:
A \textit{field} is set and 2 operations: 
$\left[ \Set{S}, +, * \right]$
where
\begin{description}
  \item[Additive closure] $a + b \in \Set{S} \; 
  \forall \, a,b \in \Set{S}$ 
  \item[Additive associativity] 
  $\left( a + b \right) + c = a + \left( b + c \right)$ 
  \item[Additive commutativity] $a + b = b + a$ 
  \item[Additive identity] $\exists \, 0 \in \Set{S} \text{ s.t. } 
  a + 0 = 0 + a = a \; \forall \, a \in \Set{S}$ 
  \item[Additive inverse] $\forall a \in \Set{S} \; 
  \exists \, {-a} \in \Set{S} 
  \text{ s.t. }  a + \left( -a \right) = \left( -a \right) + a = 0$ 
  \item[Multiplicative closure] $a * b \in \Set{S} \; \forall \,a,b \in \Set{S}$ 
  \item[Multiplicative associativity] 
  $\left( a * b \right) * c = a * \left( b * c \right)$ 
  \item[Multiplicative commutativity] $a * b = b * a$ 
  \item[Multiplicative identity] $\exists 1 \in \Set{S}
   \text{ s.t. } 
  a * 1 = 1 * a = a \; \forall \, a \in \Set{S}$ 
  \item[Multiplicative inverse] $\forall a \ne 0 \in \Set{S} 
  \exists a^{-1} \in \Set{S}
  \text{ s.t. }  a * a^{-1} = a^{-1} * a = 1$ 
  (Note the restriction to elements other than the additive identity.) 
  \item[Distributive] $a * \left( b + c \right) 
  = \left( a * b \right) + \left( a * c \right)$
\end{description}


Examples: 
\glssymbol{RationalNumbers}, \glssymbol{RealNumbers}.

%-----------------------------------------------------------------
\setcounter{currentlevel}{\value{baseSectionLevel}-1}
\levelstay{Two sets, one operation}
%-----------------------------------------------------------------
\setcounter{currentlevel}{\value{baseSectionLevel}-2}
\levelstay{Category}

A base set, $\Set{O}$, 
whose elements are ususally refered to as \textit{objects}.
A second set, $\Set{M}$, of \textit{morphisms}, 
each of which depends on an ordered pair 
$\left( \text{source} , \text{target} \right)$ of elements of 
$\Set{O}$. 
A \textit{composition} operation, $\circ$, on 
the subset of the pairs of morphisms $f , g \in \Set{M}$
where $\text{source}(f) = \text{target}(g)$, satisfying:
\begin{description}
\item[Associativity:] $f \circ g \circ g \defeq 
\left( f \circ g \right) \circ h 
= f \circ \left( g \circ h \right)$
\item[Identity:] For each $x \in \Set{O}$, there exists an element
$1_x \in \Set{M}$ such that 
$x = \text{source}(1_x) = \text{target}(1_x)$, and 
$1_x \circ f = f$ (when $x = \text{target(f)}$) and
$f \circ 1_x = f$ (when $x = \text{source(f)}$).
\end{description}

 \begin{example}[The Set category]
 $\Set{O} = $ some collection of sets;
 $\Set{M} = $ the functions between those sets;
 $\circ = $ function composition.  
 \end{example}
%-----------------------------------------------------------------
 