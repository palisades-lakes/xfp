%-----------------------------------------------------------------
\setcounter{currentlevel}{\value{baseSectionLevel}}
\levelstay{Cartesian products}
\label{sec:Cartesian-products}

Combining and factoring sets and functions.

Factor set into product of other sets.

Factor function into 'coordinate' functions:
$f : \Set{X} \rightarrow \Set{Y}_0 \times \Set{Y}_1$
via
$f \left( x \right) = 
\left[ f_0\left( x \right) , f_1\left( x \right) \right]$

Also 'reduce':
$f : \Set{X}_0 \times \Set{X}_1 \rightarrow \Set{Y}$
via 
 $f\left( \left[ x_0, x_1 \right] \right)
= r \left(
\left[ f_0 \left( x_0 \right) , f_1 \left( x_1 \right) \right] 
\right)$,
eg, for linear functions, 
 $f\left( \left[ x_0, x_1 \right] \right)
=  f_0 \left( x_0 \right) + f_1 \left( x_1 \right)$,

%-----------------------------------------------------------------
\setcounter{currentlevel}{\value{baseSectionLevel}-1}
\levelstay{Cartesian product sets}
\label{sec:Cartesian-product-sets}

% \gls{tuple}
A tuple is a fixed length list, which I will write, for example,
$[a \, b \, c]$, in a Clojure-like syntax,
or sometimes as $a \times b \times c$.

% \gls{cartesian-product}
The cartesian product set 
$\Set{A} \times \Set{B} =
\SetSpec{[a \, b] }{ a \in \Set{A} \text{ and } b \in \Set{B} }$.

Extending this notation beyond two terms,
$\Set{A} \times \Set{B} \times \Set{C}$,
introduces ambiguity typically ignored in mathematics literature,
using the implied natural identity mentioned above.
Strictly speaking,
\begin{itemize}
  \item $\Set{A} \times \Set{B} \times \Set{C} = 
  \SetSpec{[a \, b \, c] }{ \ldots }$
  \item $\Set{A} \times (\Set{B} \times \Set{C}) = 
  \SetSpec{[a \, [b \,c]] }{ \ldots }$
  \item $(\Set{A} \times \Set{B}) \times \Set{C} = 
  \SetSpec{[[a \, b] \, c]] }{ \ldots }$
\end{itemize}
are 3 distinct sets.

%-----------------------------------------------------------------
\setcounter{currentlevel}{\value{baseSectionLevel}-1}
\levelstay{Generalized tuples}
\label{sec:Generalized-tuples}

Tuples as functions from index set to set elements.

'Subspace' corresponds to subset of index set.

Canonical projections and embeddings.

Generalized coordinates.

%-----------------------------------------------------------------
\setcounter{currentlevel}{\value{baseSectionLevel}-1}
\levelstay{Cartesion product functions}
\label{sec:Cartesion-product-functions}

Aka \textit{coordinate functions}.

Suppose $f : \Set{X} \rightarrow \Set{Y}_0 \times \Set{Y}_1$,
and $f \left( x \right) = \left[ y_0 ,y_1 \right]$
Then we can write $f = f_0 \times f_1$ where
$f_0 \left( x \right) = y_0 $
and
$f_1 \left( x \right) = y_1 $.

\textbf{TODO:} $f$ corresponds to a functional relation on 
$\Set{X} \times \left( \Set{Y}_0 \times \Set{Y}-1 \right)$).
Does this imply functional relations $f_i$ on
$\Set{X} \times \Set{Y}_i$?
How about $f_i = e_i \circ f$,
where $e_i : \times_j \Set{X}_j  \rightarrow \Set{X}_i$ by
$e_i \left( [x_0, x_1, \ldots , x_{n-1} \right) = x_i$ 

% \end{figure}
