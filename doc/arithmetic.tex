%-----------------------------------------------------------------
% \setcounter{currentlevel}{\value{baseSectionLevel}}
% \levelstay{Peano arithmetic}
% 
% \begin{description}
% \item[\textsf{PA}] Peano arithmetic
% \end{description}


%-----------------------------------------------------------------
%-----------------------------------------------------------------
%-----------------------------------------------------------------
\setcounter{currentlevel}{\value{baseSectionLevel}}
\levelstay{Paradoxes}
\label{sec:Paradoxes}

Russell paradox~\cite{iep:Russell_paradox}

Russell-Myhill paradox(\cite{iep:Russell_Myhill_paradox})

%-----------------------------------------------------------------
%-----------------------------------------------------------------
%-----------------------------------------------------------------
%-----------------------------------------------------------------
\setcounter{currentlevel}{\value{baseSectionLevel}}
\levelstay{Set theories}


Axiomatic theories of various kinds.
Notes almost entirely from 
Wikipedia\cite{wiki:Set_theory,iep:Set_theory,eom:Set_theory,sep:Set_theory}.

\setcounter{currentlevel}{\value{currentlevel}-1}
%-----------------------------------------------------------------
%-----------------------------------------------------------------
%-----------------------------------------------------------------
\levelstay{Cantor set theory}
\label{sec:Cantor_set_theory}

``In his development of set theory, 
Cantor identified a single fundamental principle, 
called the Comprehension Principle, 
under which one can form a set.
Cantor’s principle states that, 
given any specific property $\varphi(x)$ 
concerning a variable $x$, 
the collection $\{x:\varphi(x)\}$ is a set, 
where $\{x:\varphi(x)\}$ is 
the set of all objects x that satisfy the property φ(x).
For example, let $\varphi(x)$ be the property that 
'$x$ is an odd natural number.' 
The Comprehension Principle implies that

$\Set{S}=\{x:\varphi(x)\}=\{1,3,5,7,\dots\}$ ``\cite{iep:Set_theory}

Problems: \hfill\break
where do we get the $x$ to plug into $\varphi(x)$?\hfill\break
if $x\in\mathbb{N}$, I think I know how to tell if it's odd;
how do I verify $x\in\mathbb{N}$,?\hfill\break
what does ``\dots'' mean?\hfill\break
what is $\varphi$ exactly?\hfill\break
if $\varphi$ is ``computable'' in some sense,
then it might return true, or return false, or not return
(infinite loop or recursion). what happens then?

%-----------------------------------------------------------------
%-----------------------------------------------------------------
%-----------------------------------------------------------------
\levelstay{von Neumann universe}
\label{sec:von_Neumann_universe}

Attributed to von Neumann, but may really be due to
Zermelo\cite{Zermelo:1930:ZERBGU}.

von Neumann universe: \textsf{V}\cite{wiki:Von_Neumann_universe},
the class of hereditary well-founded sets.

Hereditary set: elements are all hereditary sets, 
starting from empty set\cite{wiki:Hereditary-set}.

All elements of sets are sets themselves.

No \textsl{ur-elements}\cite{wiki:Urelement} 
(aka ``atoms'', ``individuals'') ---
non-set things that may belong to a set.

(Note: \textsl{proper class} is dual to ur-element in the sense that
ur-elements do not have elements and proper classes cannot be 
elements.)

(Perhaps elegant---sets all the way down---, but unsatisfactory. 
In application, sets are useful for organizing the things
of real interest---the ur-elements.
Constructing an isomorphism between hereditary sets
and numbers, and isomorhisms between structures built
from numbers and other axiomatic structures seems
backwards to me.
I think the abstract axiomatic structure comes first, 
and isomorphic representations (or models) in terms of 
(equivalence classes over) more concrete/primitive
entities a secondary technique useful in proofs and calculation.)

Well-founded set: if set membership relation
is well-founded on the transistive closure of the set.

Well-founded relation\cite{wiki:Well-founded-relation}: 
 
Relation $R$ on class $\Set{X}$ s.t. 
$(\forall \Set{S} \subseteq \Set{X})
\; \left(\Set{S} \neq \emptyset \implies 
(\exists m \in \Set{S})(\forall s\in \Set{S})\lnot (sRm)\right).$
Well-founded relations enable a version of transfinite induction.

Class\cite{wiki:Class_set_theory}:
a collection of sets that is unambiguously defined
by property that all elements share. 
(Constrast with ``proper class'' 
in NBG set theory\cite{wiki:NBG-set-theory},
ie, entities that are not elements of another entity.)

Transitive closure: the elements, the elements of the elements,
etc.

Transfinite recursive definition:

$V_{0} \,\doteq\, \varnothing$.

$V_{\beta +1} \,\doteq\, \wp(V_{\beta })$.
$\beta$ any ordinal\cite{wiki:Ordinal_number}.
EG: $V_1 = \{ V_0 \} = \{ \varnothing \}$,
 $V_2 = \{ \varnothing,  \{ \varnothing \} \}$, \ldots .

$V_{\lambda }\,\doteq\,\bigcup _{\beta <\lambda }V_{\beta }$.
$\lambda$ any limit ordinal\cite{wiki:Limit_ordinal}.

$V_{\alpha}$ are the stages or ranks.

$V\,\doteq\,\bigcup _{\alpha }V_{\alpha }$.

Alternative definition of ranks:

$V_{\alpha }
\,\doteq\,
\bigcup _{\beta <\alpha }\wp(V_{\beta })$.

$V$ is not the set of all sets, 
because (a) it is a proper class, not a set,
(b) it only contains well-founded sets,
and (c) it may be that not all sets are pure,
ie, there may be ur-elements (non-set things) in sets.

Existence of $V$: Wikipedia article pivots to consistency instead.
Discussion seems to leave the question open (???!!!).
 

Models for set theories:

$V_{\omega}$ is the set of hereditary finite sets,
a model for set theory without axiom of infinity.

$V_{\omega+1}$ corresonds to $\mathbb{N}, \mathbb{Z}$;
$V_{\omega+2}$ corresonds to $\mathbb{R}$
 
$V_{\omega+\omega}$ model for \nameref{sec:Zermelo_set_theory},
``universe of ordinary mathemetics''.

$V_{\kappa}$ is a model for \textsf{ZFC} 
if $\kappa$ is an inaccessible cardinal\cite{wiki:Inaccessible_cardinal}.

$\kappa$ is an strongly inaccessible cardinal 
if it is uncountable, 
not a sum of fewer than $\kappa$ cardinals each $<\,\kappa$,
and
$(\alpha \,<\, \kappa) \;\Rightarrow\; (2^{\alpha}\,<\,\kappa)$.
(Not the most intuitive\ldots .)

$\kappa$ is an weakly inaccessible cardinal 
if it is uncountable,
and
it is a regular weak limit 
cardinal\cite{wiki:Regular_cardinal,wiki:Limit_cardinal}.

%-----------------------------------------------------------------
%-----------------------------------------------------------------
%-----------------------------------------------------------------
\levelstay{Constructible universe}
\label{sec:Constructible_universe}

aka G\"{o}del's Constructible 
universe\cite{wiki:Constructible_universe}.

Roughly a restriction of the \nameref{sec:von_Neumann_universe}. 

%-----------------------------------------------------------------
%-----------------------------------------------------------------
%-----------------------------------------------------------------
\levelstay{Zermelo set theory}
\label{sec:Zermelo_set_theory}

\textsf{Z}\cite{wiki:Zermelo_set_theory}

Predecessor of \nameref{sec:Zermelo-Fraenkel-set-theory}.

Insufficient for theory of infinite 
ordinals\cite{wiki:Ordinal_number}
and cardinals.

Strongler, original ``2nd order'' version 
vs  later 1st order variants.

See also MacLane set theory\cite{maclane:mff:1986}.
%-----------------------------------------------------------------
%-----------------------------------------------------------------
%-----------------------------------------------------------------
\levelstay{Zermelo Fraenkel set theory}
\label{sec:Zermelo-Fraenkel-set-theory}

Purpose: avoid Russell's paradox\cite{wiki:Russell-paradox}.

``Over time, it became clear that, to resolve the 
paradoxes in Cantor’s set theory, the Comprehension Principle 
needed to be modified. Thus, the following question needed to 
be addressed:

How can one correctly construct a set? 
Ernst Zermelo (1871–1953) observed that t
o eliminate the paradoxes, 
the Comprehension Principle could be restricted as follows: 
Given any set A and any property ψ(x), 
one can form the set {x∈A:ψ(x)}, that is, 
the collection of all elements x∈A that satisfy ψ(x), is a set.
Zermelo’s approach differs from Cantor’s method of forming a set. 
Cantor declared that for every property one can form 
a a set of all the objects that satisfy the property.
Zermelo adopted a different approach: 
To form a set, one must use a property together with a set.

Zermelo also realized that in order to more fully develop 
Cantor’s set theory, 
one would need additional methods for forming sets. 
Moreover, these additional methods would need 
to avoid the paradoxes. 
In 1908, 
Zermelo published an axiomatic system for set theory that, 
to the best of our knowledge, 
avoids the difficulties faced by 
Cantor’s development of set theory. 
In 1930, 
after receiving some proposed revisions from Abraham Fraenkel,
 Zermelo presented his final axiomatization of set theory, 
 now known as the Zermelo–Fraenkel axioms and denoted by ZF. 
These axioms have become the accepted formulation 
of Cantor’s ideas about the nature of sets.''\cite{iep:Set_theory} 

``As noted by Zermelo, to avoid paradoxes, 
the Comprehension Principle can be replaced with the principle: 
Given a set A and a property φ(x) with a variable x, the 
collection {x∈A:φ(x)} is a set. 
However, this raises a new question: 
What is a property? 
The most favored way to address this question is to express 
the axioms of set theory
in the formal language of first-order logic, 
and then declare that its formulas designate properties. 
This language involves variables and the logical connectives
 ∧ (and), ∨ (or), ¬ (not), → (if … then …), 
 and ↔ (if and only if), 
 together with the quantifier symbols ∀ (for all) 
 and ∃ (there exists). 
 In addition, this language uses the relation symbols
  = and ∈ (as well as ≠ and ∉). 
  In this language, the variables and quantifiers range over sets 
  and only sets. 
  A formula constructed in this formal language is referred to as
   a formula in the language of set theory. Such formulas are used 
   to give meaning to the notion of 
   'property.'''\cite{iep:Set_theory}

Question: 
contrast 1st order logic above 
with intuitionist logic?\cite{wiki:Intuitionistic_logic}

\begin{description}
\item[\textsf{ZF}] Zermelo-Fraenkel set theory\cite{wiki:Zermelo–Fraenkel-set-theory}
\item[\textsf{ZFC}] \textsf{ZF} with axiom of choice\cite{wiki:Axiom_of_choice}.
\item[\textsf{ZFA}] \textsf{ZF} with ur-elements\cite{wiki:Urelement}
\item[\textsf{ZFAC}] \textsf{ZFA} with axiom of choice\cite{wiki:Axiom_of_choice}.
\end{description}

Universe of discourse: 
Hereditary well-founded sets 
(from the \nameref{sec:von_Neumann_universe}).

Existence of empty set is either an 
axiom\cite{wiki:Axiom_of_empty_set}
or a theorem dereived from \nameref{sec:Axiom-of-infinity}
or derived from \nameref{sec:Axiom-schema-of-specification}
and ``a set-existence axiom''.

(Question: is the empty set a special case of an ur-element?
See also: Quine atoms\cite{wiki:Urelement}.)

Consistency of \textsf{ZFC} cannot be proved, 
by G\"{o}del's 2nd incompleteness theorem.

Multiple equivalent sets of axioms.
Each axiom should be true if interpreted as a statement about the
collection of all sets in the Von Neumann 
universe\cite{wiki:Von_Neumann_universe},
more or less the closure under powerset and union starting
from the empty set.

G\"{o}del's 2nd incompleteness theorem implies \textsf{ZFC}
cannot be proved consistent (within \textsf{ZFC}) 
unless it is inconsistent 
(an inconsistent set of axioms can prove anything).

%-----------------------------------------------------------------
\setcounter{currentlevel}{\value{currentlevel}-1}
%-----------------------------------------------------------------
\levelstay{Axiom of extensionality}

Sets are equal if they have the same 
elements.\cite{wiki:Axiom_of_extensionality,wiki:Extensionality}

$\forall \Set{A}\,\forall \Set{B}\,
(\forall \Set{X}\,(\Set{X}\in \Set{A}\iff \Set{X}\in \Set{B})
\implies \Set{A}=\Set{B})$

What about identity? 

What's the difference between two things being equal vs
being the \textit{same} thing? 
\textsf{= a b)} vs \textsf{(identical? a b)}?

Is it really some kind of ``set specification'' that's equal?

Version without equality:

$\forall x
\forall y
[\forall z(z\in x\Leftrightarrow z\in y)
\Rightarrow 
\forall w(x\in w\Leftrightarrow y\in w)]$

In other words: if $x$ and $y$ have the same elements, 
they are in the same sets.

This is even more unsatisfying. 
Raises the same unanswered issues about set identity.
Worse, it relies on some property of all sets in an undefined 
universe of possible sets.

Version with ur-elements:

$\forall \Set{A}\,
\forall \Set{B}\,
(\exists \Set{X}\,(\Set{X}\in \Set{A})
\implies
 [\forall Y\,
 (Y\in \Set{A}
 \iff 
 Y\in \Set{B}) \implies \Set{A}=\Set{B}]\,)$
 
In words: 
if $\Set{\Set{A}}$ is non-empty, $\Set{B}$ any set, 
and $\Set{\Set{A}}$ and $\Set{B}$ have the same
elements, they are equal.

(Question: ur-element version without equality?)

%-----------------------------------------------------------------
\levelstay{Axiom of regularity}

(aka axiom of foundation)~\cite{wiki:Axiom_of_regularity}

Every non-empty set $\Set{A}$ contains a set which is disjoint from $\Set{A}$.

$\forall x\,(x \neq \varnothing
\rightarrow 
\exists y\in x\,(y\cap x=\varnothing ))$.
 
or
 
$\forall x\,(x\neq \varnothing \Rightarrow 
 \exists y\in x\,(y\cap x=\varnothing ))$
 
(No infinite loops? Guarantee halting? 
For $\Set{A} = \{\Set{A}\}$,
\textsf{(hasElement A x)} 
returns \textsf{true} if \textsf{(== A x)}
and \textsf{false} otherwise. )

Consequences:
\begin{itemize}
\item 
Equivalent to axiom of induction\cite{wiki:Epsilon-induction} 
in \textsf{ZF}:
$\forall x
[\forall y\,(y\in x\rightarrow P(y))\rightarrow P(x)]
\rightarrow \forall z\,P(z)$.
(Assuming induction preferred in constructionist theories?)

\item With the \nameref{sec:Axiom-of-pairing}, 
implies no set contains itself.

\item No infinite descending sequences of sets.
``Descending'' in the sense that each set is an element of the
preceeding set in the sequence.

\item Simplifies definition of ordered pair.

\item Enables defining ordinal rank for every set.

\item Given 2 sets, at most one can be an element of the other.
\end{itemize}

Implied by axiom of dependent choice and 
no descending infinite sequence.

Independent of other \textsf{ZF} axioms, like axiom of choice.
Added to \textsf{ZF} to 
``exclude models with some undesirable properties''.

%-----------------------------------------------------------------
\levelstay{Axiom schema of specification}
\label{sec:Axiom-schema-of-specification}

aka axiom schema of separation, subset axiom scheme,
axiom schema of restricted comprehension \ldots .

Restricted comprehension avoids Russell paradox,
so claimed to be ``most important'' axiom.

``Given any set $\Set{A}$, there is a set $\Set{B}$
 (a subset of $\Set{A}$) 
such that, given any set $x$, 
$x$ is a member of $\Set{B}$ if and only if $x$ 
is a member of $\Set{A}$ 
and $\varphi$ holds for $x$.''
\cite{wiki:Axiom_schema_of_specification}

${\displaystyle 
\forall w_{1},\ldots ,w_{n}\,
\forall \Set{A}\,\exists \Set{B}\,
\forall x\,
(x\in \Set{B}
\Leftrightarrow
[x\in \Set{A}\land \varphi (x,w_{1},\ldots ,w_{n},\Set{A})])}$

Essence:
Every \textsl{subclass}
of a set defined by a predicate is a set.

(What is a predicate?
A predicate is a true/false valued function.
What is a function? 
Does platonic relation-based definition require sets,
thus circular?
If functions are computable, does that make the
russell paradox just an infinite loop,
and not a contradiction?
IS a known infinite loop different from 
unknown predicate value, or knowable but unproven?
)

\ldots``a subclass is a class contained in some other class in 
the same way that a subset is a set contained in some other set.''
\cite{wiki:Subclass_set_theory}
(???!!!)

``\dots a \textsl{class} is a collection of sets 
(or \ldots other \ldots objects)
the can be unambiguously defined by a property 
that all its members share.``\cite{wiki:Class_set_theory}
(???!!!)

A \textsl{proper} class is not a set. (???!!)

2nd order axiom schema because it is quantified over predicates
 $\varphi$.

Implied by \nameref{sec:Axiom-schema-of-replacement}
and \cite{wiki:Axiom_of_empty_set}.

Distinction from Axiom schema of (unrestricted) comprehension:

$\forall w_1,\ldots,w_n \, \exists \Set{B} \, 
\forall x \, ( x \in \Set{B} 
\Leftrightarrow \varphi(x, w_1, \ldots, w_n) )$
There exists a (unique) set $\Set{B}$ 
whose members are precisely those objects 
that satisfy the predicate $\varphi$.
This give the Russell paradox if 
$\varphi(x) \doteq \neg(x \in x)$

%-----------------------------------------------------------------
\levelstay{Axiom of pairing}
\label{sec:Axiom-of-pairing}

Axiom of pairing unnecessary.
A consequence of \nameref{sec:Axiom-schema-of-replacement}
applied to any set with $2$ or more elements.
Existence of a set with $\geq 2$ elements
(eq $\{ \{\}, \{ \{\} \} \}$)
deduced from \nameref{sec:Axiom-of-infinity}
(or axiom of empty set\cite{wiki:Axiom_of_empty_set}
and \nameref{sec:Axiom-of-power-set}).

For any $2$ sets $\Set{A}$ and $\Set{B}$,
there exists a set containing exactly $\Set{A}$ and 
$\Set{B}$~\cite{wiki:Axiom_of_pairing}.

$\forall \Set{A}\,\forall \Set{B}\,\exists \Set{C}\,\forall \Set{D}\,
[\Set{D}\in \Set{C}\iff (\Set{D}=\Set{A}\lor \Set{D}=\Set{B})]$;

Given any set $\Set{A}$ and any set $\Set{B}$, 
there is a set $\Set{C}$ such that, 
given any set $\Set{D}$, 
$\Set{D}$ is a member of $\Set{C}$ 
if and only if 
$\Set{D}$ is equal to $\Set{A}$ 
or 
$\Set{D}$ is equal to $\Set{B}$.

Special case of axiom of elementary 
sets\cite{wiki:Zermelo_set_theory}.

Singleton:
(Another identity/equality issue)
$\Set{S}=\{\Set{A},\Set{A}\}$, abbreviated $\{\Set{A}\}$,
defines a singleton as a repeated pair---which doesn't make sense,
sonce there's no provision for having the same element
in a set ``more than once'', whatever that might mean.
Axiom as stated doesn't specify $\Set{A} \neq \Set{B}$

Ordered pair:
$(a,b)=\{\{a\},\{a,b\}\}$.
See also Halmos\cite{Halmos1960Naive}.
Is this really necessary? 

Weaker versions:
 
${\displaystyle 
\forall \Set{A} \forall \Set{B} 
\exists \Set{C}
\forall \Set{D}((\Set{D}=\Set{A}\lor \Set{D}=\Set{B}) \Rightarrow \Set{D}\in \Set{C})}$
plus 
\nameref{sec:Axiom-schema-of-specification}
implies usual axiom of pairing.

${\displaystyle 
\forall \Set{A}\,\forall \Set{B}\,
\exists \Set{C}\,
\forall \Set{D}\,[\Set{D}\in \Set{C}
\iff (\Set{D}\in \Set{A}\lor \Set{D}=\Set{B})]}$.
plus 
axiom of empty set
implies usual axiom of pairing.

Stronger versions:

With axiom of empty set\cite{wiki:Axiom_of_empty_set} 
and \nameref{sec:Axiom-of-union},
implies existence of a (unique) set containing exactly
any finite number of given sets.

%-----------------------------------------------------------------
\levelstay{Axiom of union}
\label{sec:Axiom-of-union}

The union over the elements of a set 
exists\cite{wiki:Axiom_of_union}.

Note that this is using the incestuous nature of 
\textsf{ZF} without ur-elements---the elements of sets
are all sets themselves. 
 
$\forall {\Set{F}}\,
\exists \Set{A}\,
\forall \Set{Y}\,\forall x
[(x\in \Set{Y} \land \Set{Y}\in \Set{F})
\Rightarrow x\in \Set{A}]$

Using \nameref{sec:Axiom-schema-of-replacement}:

Union defined as an operation on a single set:\hfill\break
${\displaystyle 
\cup {\Set{F}} \; \doteq \;
\{x\in \Set{A}:\exists \Set{Y} 
(x \in \Set{Y} \land \Set{Y} \in \Set{F})\}.}$

Define usual set union via axiom of pairing:
$\Set{A} \cup \Set{B} \; \doteq \; \cup \{ \Set{A}, \Set{B}\}$

Define union of indexed family (?) of sets
using \nameref{sec:Axiom-schema-of-replacement}.

With \nameref{sec:Axiom-schema-of-specification},
can define a weaker form:
$\forall {\Set{F}}\,\exists \Set{A}\,\forall \Set{Y}\,
\forall x
[(x\in \Set{Y} 
\land 
\Set{Y}\in {\Set{F}})
\Rightarrow 
x\in \Set{A}]$.

Note that intersection can be defined from 
\nameref{sec:Axiom-schema-of-specification}:\hfill\linebreak
${\displaystyle 
\bigcap \Set{A}=\{c\in \Set{E}:
\forall \Set{D} (\Set{D} \in \Set{A} \Rightarrow c\in \Set{D})\}}$.
Note also that applying this to the empty set
is not permitted by ''the axioms'';
otherwise we would get a universe set 
(skeptical about this.
couldn't we require the elements of the intersection
to be elements of some element of $\Set{A}$?
or evaluate the predicate over $\bigcup \Set{A}$)

%-----------------------------------------------------------------
\levelstay{Axiom schema of replacement}
\label{sec:Axiom-schema-of-replacement}

Version 1: 
The image of a definable (?) function applied to a set
is contained in some set.\cite{wiki:Axiom_schema_of_replacement}
(``image of set under function'' is usually defined as a set.)
aka axiom schema of collection.

Version 2: 

The image of any set under a definable mapping is a set.

Suppose $P$ is a definable (?) relation such that for every
set $X$ there is a unique set $Y$ such that $P(X,Y)$.
$P$ may be a proper class\cite{wiki:Class_set_theory}, 
ie, not a \textit{set}, of ordered pairs. 
(Leads to some hackiness about relations that are not sets,
for unconvincing reasons\cite{wiki:Binary_relation}.)
Definable (class) function: $F_P(X)\,=\,Y \iff P(X,Y)$.
Collection $\Set{B}$ (may be a proper class)
defined so that for all $y\in\Set{B}$ there exist $x \in \Set{A}$
such that $y=F_P(x)$.
(Something backwards about this.)

$\Set{B}\,=\,F_p(\Set{A}) \,=\, \{F_p(x) : x \in \Set{A}\}$ 
is the \textsl{image} of $\Set{A}$ under $F_p$.

Principle of smallness: if $\Set{A}$ is 
``small enough'' to be a set,
then so is $F(\Set{A})$.

\nameref{sec:Axiom-schema-of-replacement} implied by stronger
axiom of limitation of 
size\cite{wiki:Axiom_of_limitation_of_size}:
a class that is a member of a class is a set (???!!!).

nameref{sec:Axiom-schema-of-replacement} not needed for most math;
not present in \textsf{Z};
``drastically increases the strength of \textsf{ZF}''.

Used in proving theorems about ordinals.

Axiom schema of collection: some superclass of 
the image of relation is a set.
Stronger than replacement
in some axiom systems, weaker in others.
See also axiom shcema of boundedness.

With axiom of empty set, replacement implies specification.,
using law of excluded middle.

Replacement is main thing distinguishing \textsf{Z}
from \textsf{ZF}.

Skoelem's 1st order version vs Fraenkel's 2nd order?
%-----------------------------------------------------------------
\levelstay{Axiom of infinity}
\label{sec:Axiom-of-infinity}

Roughly, there exists a set with infinitely many 
elements\cite{wiki:Axiom_of_infinity}.
(Some nonsense about 2 elements being the same.)

More formally, asserts existance of an infinite set 
constructed via von Neumann:
$\emptyset, \{\emptyset\}, \{ \emptyset, \{\emptyset\} \} \cdots$

Even more formal, take the minimal set satisfying:
${\displaystyle 
\exists \mathbf {I}
 \,(\emptyset \in \mathbf {I}
 \,\land \,
 \forall x\in \mathbf {I} \,
 (\,(x\cup \{x\})\in \mathbf {I} )).}$
 
 Proved independent of other \textsf{ZFC} axioms.
 
 
%-----------------------------------------------------------------
\levelstay{Axiom of power set}
\label{sec:Axiom-of-power-set}

\textsl{Subset}:
$(z\subseteq x)
 \Leftrightarrow 
 (\forall q(q\in z\Rightarrow q\in x)).$
 
The power set is the set of all subsets of a given set.
The \nameref{sec:Axiom-of-power-set} says the the power set
always exists\cite{wiki:Axiom_of_power_set}:

${\displaystyle \wp(x) \,=\, \{z\in y:z\subseteq x\}}$

%-----------------------------------------------------------------
\levelstay{Well-ordering theorem}
\label{sec:Well_ordering_theorem}

Previous 8 axioms define \textsf{ZF}.

The Well-ordering ``theorem'' axiom turns \textsf{ZF}
into \textsf{ZFC}\cite{wiki:Well_ordering_theorem}.

For any set $\Set{X}$ there is a binary relation $R$ which
 well-orders $\Set{X}$.
 $R$ is a linear order and every non-empty subset of $\Set{X}$
 has an element which is minimal under $R$.

Independent of previous 8 axioms.

Equivalent to axiom of choice\cite{wiki:Axiom_of_choice}, 
assuming previous 8 axioms, in 1st order logic;
stronger in 2nd order logic.

Stronger than axiom of choice without 8 axioms.

Consequence of Zorn's lemma\cite{wiki:Zorns_lemma}.

%-----------------------------------------------------------------
%-----------------------------------------------------------------
\setcounter{currentlevel}{\value{baseSectionLevel}-1}
\levelstay{vonNeumann-Bernays-G\"{o}del set theory}
\label{vonNeumann-Bernays-Godel_set_theory}
\textsf{NBG}\cite{wiki:NBG-set-theory}

Axiomatic definition of ``class'', ``proper class''.

%-----------------------------------------------------------------
%-----------------------------------------------------------------
\setcounter{currentlevel}{\value{baseSectionLevel}-1}
\levelstay{Tarski–Grothendieck set theory}
\textsf{TG}

%-----------------------------------------------------------------
%-----------------------------------------------------------------
\setcounter{currentlevel}{\value{baseSectionLevel}-1}
\levelstay{Constructive set theory}

%-----------------------------------------------------------------
%-----------------------------------------------------------------
\setcounter{currentlevel}{\value{baseSectionLevel}}
\levelstay{Church, G\"{o}del, Turing}


%-----------------------------------------------------------------
%-----------------------------------------------------------------
\setcounter{currentlevel}{\value{baseSectionLevel}}
\levelstay{Libraries for exact or multi-precision arithmetic}

\setcounter{currentlevel}{\value{baseSectionLevel}-1}
\levelstay{GMP}

\setcounter{currentlevel}{\value{baseSectionLevel}-1}
\levelstay{MPFR}

\setcounter{currentlevel}{\value{baseSectionLevel}-1}
\levelstay{ICReals}

\cite{Briggs:2006,Briggs:XRC:2013}

\setcounter{currentlevel}{\value{baseSectionLevel}-1}
\levelstay{XRC}

\cite{Briggs:2006,Briggs:XRC:2013}

\setcounter{currentlevel}{\value{baseSectionLevel}-1}
\levelstay{LEDA}

\cite{Burnikel:1996,Burnikel:1999,Mehlhorn:1995,LEDA:2009}

\setcounter{currentlevel}{\value{baseSectionLevel}-1}
\levelstay{CGAL}

\setcounter{currentlevel}{\value{baseSectionLevel}-1}
\levelstay{Core2}


\cite{Karamcheti:1999}

\setcounter{currentlevel}{\value{baseSectionLevel}-1}
\levelstay{iRRAM}

\setcounter{currentlevel}{\value{baseSectionLevel}-1}
\levelstay{CRcalc}

\setcounter{currentlevel}{\value{baseSectionLevel}-1}
\levelstay{Spire}

\cite{Spire:2019}

\setcounter{currentlevel}{\value{baseSectionLevel}-2}
\levelstay{Rational}

\setcounter{currentlevel}{\value{baseSectionLevel}-2}
\levelstay{Algebraic}

Based on
\cite{yap:guaranteed:2004,li-pion-yap:progress:2004,pion-yap:kary:2003,Pion:2006,Li:2001,Burnikel:2001}

\setcounter{currentlevel}{\value{baseSectionLevel}-2}
\levelstay{Real}

Based on \cite{Lester:2012}.

 