%-----------------------------------------------------------------
\setcounter{currentlevel}{\value{baseSectionLevel}}
\levelstay{Peano arithmetic}

\begin{description}
\item[\textbf{PA}] Peano arithmetic
\end{description}


%-----------------------------------------------------------------
%-----------------------------------------------------------------
%-----------------------------------------------------------------
%-----------------------------------------------------------------
\setcounter{currentlevel}{\value{baseSectionLevel}}
\levelstay{Set theories}

\setcounter{currentlevel}{\value{currentlevel}-1}
%-----------------------------------------------------------------
%-----------------------------------------------------------------
%-----------------------------------------------------------------
\levelstay{Zermelo set theory}

\textbf{Z}\cite{wiki:Zermelo_set_theory}

Predecessor of \nameref{sec:Zermelo-Fraenkel-set-theory}.

Insufficient for theory of infinite 
ordinals\cite{wiki:Ordinal_number}
and cardinals.

Strongler, original ``2nd order'' version 
vs  later 1st order variants.

See also MacLane set theory\cite{maclane:mff:1986}.
%-----------------------------------------------------------------
%-----------------------------------------------------------------
%-----------------------------------------------------------------
\levelstay{Zermelo Fraenkel set theory}
\label{sec:Zermelo-Fraenkel-set-theory}

Purpose: avoid Russell's paradox\cite{wiki:Russell-paradox}.

\begin{description}
\item[\textbf{ZF}] Zermelo-Fraenkel set theory\cite{wiki:Zermelo–Fraenkel-set-theory}
\item[\textbf{ZFC}] \textbf{ZF} with axiom of choice\cite{wiki:AxiomOfChoice}.
\item[\textbf{ZFA}] \textbf{ZF} with ur-elements
\item[\textbf{ZFAC}] \textbf{ZFA} with axiom of choice.
\end{description}

Universe of discourse: 
Hereditary well-founded sets.
All elements of sets are sets themselves.

No \textsl{ur-elements}\cite{wiki:Urelement} 
(aka ``atoms'', ``individuals'') ---
non-set things that may belong to a set.

(Note: \textsl{proper class} is dual to ur-element in the sense that
ur-elements do not have elements and proper classes cannot be 
elements.)

(Perhaps elegant---sets all the way down---, but unsatisfactory. 
In application, sets are useful for organizing the things
of real interest---the ur-elements.
Constructing an isomorphism between hereditary sets
and numbers, and isomorhisms between structures built
from numbers and other axiomatic structures seems
backwards to me.
I think the abstract axiomatic structure comes first, 
and isomorphic representations (or models) in terms of 
(equivalence classes over) more concrete/primitive
entities a secondary technique useful in proofs and calculation.)

Hereditary set: elements are all hereditary sets, 
starting from empty set\cite{wiki:Hereditary-set}.

(Question: is the empty set a special case of an ur-element?
See also: Quine atoms\cite{wiki:Urelement}.)

Well-founded set: if set membership relation
is well-founded on the transistive closure of the set.

Well-founded relation\cite{wiki:Well-founded-relation}: 
 
Relation $R$ on class $\Set{X}$ s.t. 
$(\forall S\subseteq X)
\; \left(S\neq \emptyset \implies 
(\exists m\in S)(\forall s\in S)\lnot (sRm)\right).$
Well-founded relations enable a version of transfinite induction.

Class\cite{wiki:Class-set-theory}:
a collection of sets that is unambiguously defined
by property that all elements share. 
(Constrast with ``proper class'' 
in NBG set theory\cite{wiki:NBG-set-theory},
ie, entities that are not elements of another entity.)

Transitive closure: the elements, the elements of the elements,
etc.

Consistency of \textbf{ZFC} cannot be proved, 
by G\"{o}del's 2nd incompleteness theorem.

Multiple equivalent sets of axioms.
Each axiom should be true if interpreted as a statement about the
collection of all sets in the Von Neumann 
universe\cite{wiki:Von_Neumann_universe},
more or less the closure under powerset and union starting
from the empty set.

%-----------------------------------------------------------------
\leveldown{Axiom of extensionality}

Sets are equal if they have the same 
elements.\cite{wiki:Axiom_of_extensionality,wiki:Extensionality}

$\forall A\,\forall B\,
(\forall X\,(X\in A\iff X\in B)
\implies A=B)$

What about identity? 

What's the difference between two things being equal vs
being the \textit{same} thing? 
\textsf{= a b)} vs \textsf{(identical? a b)}?

Is it really some kind of ``set specification'' that's equal?

Version without equality:

$\forall x
\forall y
[\forall z(z\in x\Leftrightarrow z\in y)
\Rightarrow 
\forall w(x\in w\Leftrightarrow y\in w)]$

In other words: if $x$ and $y$ have the same elements, 
they are in the same sets.

This is even more unsatisfying. 
Raises the same unanswered issues about set identity.
Worse, it relies on some property of all sets in an undefined 
universe of possible sets.

Version with ur-elements:

$\forall A\,\forall B\,(\exists X\,(X\in A)\implies
 [\forall Y\,(Y\in A\iff Y\in B)\implies A=B]\,)$
 
In words: 
if $A$ is non-empty, $B$ any set, and $A$ and $B$ have the same
elements, they are equal.

(Question: ur-element version without equality?)

%-----------------------------------------------------------------
\levelstay{Axiom of regularity}

(aka axiom of foundation)~\cite{wiki:Axiom_of_regularity}

Every non-empty set $A$ contains a set which is disjoint from $A$.

$\forall x\,(x\neq \varnothing
\rightarrow 
\exists y\in x\,(y\cap x=\varnothing ))$.
 
or
 
$\forall x\,(x\neq \varnothing \Rightarrow 
 \exists y\in x\,(y\cap x=\varnothing ))$
 
(No infinite loops? Guarantee halting? 
For $\Set{A} = \{\Set{A}\}$,
\textsf{(hasElement A x)} 
returns \textsf{true} if \textsf{(== A x)}
and \textsf{false} otherwise. )

Consequences:
\begin{itemize}
\item 
Equivalent to axiom of induction\cite{wiki:Epsilon-induction} 
in ZF:
$\forall x
{\Big (}
\forall y\,(y\in x\rightarrow P(y))\rightarrow P(x)
{\Big )}
\rightarrow \forall z\,P(z)$.
(Assuming induction preferred in constructionist theories?)

\item With the \nameref{sec:Axiom-of-pairing}, 
implies no set contains itself.

s\item No infinite descending sequences of sets.
``Descending'' in the sense that each set is an element of the
preceeding set in the sequence.

\item Simplifies definition of ordered pair.

\item Enables defining ordinal rank for every set.

\item Given 2 sets, at most one can be an element of the other.
\end{itemize}

Implied by axiom of dependent choice and 
no descending infinite sequence.

Independent of other \textbf{ZF} axioms, like axiom of choice.
Added to \textbf{ZF} to 
``exclude models with some undesirable properties''.

%-----------------------------------------------------------------
\levelstay{Axiom schema of specification}
\label{sec:Axiom-schema-of-specification}

``Given any set $A$, there is a set $B$ (a subset of $A$) 
such that, given any set $x$, 
$x$ is a member of $B$ if and only if $x$ is a member of $A$ 
and $\varphi$ holds for $x$.''
\cite{wiki:Axiom_schema_of_specification}

${\displaystyle 
\forall w_{1},\ldots ,w_{n}\,
\forall A\,\exists B\,
\forall x\,
(x\in B
\Leftrightarrow
[x\in A\land \varphi (x,w_{1},\ldots ,w_{n},A)])}$
 
2nd order axiom schema because it is quantified over predicates
 $\varphi$.
 
%-----------------------------------------------------------------
\levelstay{Axiom of pairing}
\label{sec:Axiom-of-pairing}

Axiom of pairing unnecessary.
A consequence of \nameref{sec:Axiom-schema-of-replacement}
applied to any set with $2$ or more elements.
Existence of a set with $\geq 2$ elements
(eq $\{ \{\}, \{ \{\} \} \}$)
deduced from \nameref{sec:Axiom-of-infinity}
(or 
axiom of empty set\cite{wiki:Axiom_of_empty_set}
and \nameref{sec:Axiom-of-power-set}).

For any $2$ sets $\Set{A}$ and $\Set{B}$,
there exists a set containing exactly $\Set{A}$ and 
$\Set{B}$~\cite{wiki:Axiom_of_pairing}.

$\forall A\,\forall B\,\exists C\,\forall D\,
[D\in C\iff (D=A\lor D=B)]$;

Given any set $\Set{A}$ and any set $\Set{B}$, 
there is a set $\Set{C}$ such that, 
given any set $\Set{D}$, 
$\Set{D}$ is a member of $\Set{C}$ 
if and only if 
$\Set{D}$ is equal to $\Set{A}$ 
or 
$\Set{D}$ is equal to $\Set{B}$.

Special case of axiom of elementary 
sets\cite{wiki:Zermelo_set_theory}.

Singleton:
(Another identity/equality issue)
$\Set{S}=\{\Set{A},\Set{A}\}$, abbreviated $\{\Set{A}\}$,
defines a singleton as a repeated pair---which doesn't make sense,
sonce there's no provision for having the same element
in a set ``more than once'', whatever that might mean.
Axiom as stated doesn't specify $\Set{A} \neq \Set{B}$

Ordered pair:
$(a,b)=\{\{a\},\{a,b\}\}$.
See also Halmos\cite{Halmos1960Naive}.
Is this really necessary? 

Weaker versions:
 
${\displaystyle 
\forall A \forall B 
\exists C
\forall D((D=A\lor D=B) \Rightarrow D\in C)}$
plus 
\nameref{sec:Axiom-schema-of-specification}
implies usual axiom of pairing.

${\displaystyle 
\forall A\,\forall B\,
\exists C\,
\forall D\,[D\in C\iff (D\in A\lor D=B)]}$.
plus 
axiom of empty set
implies usual axiom of pairing.

Stronger versions:

With axiom of empty set and axiom of union,
implies existence of a (unique) set containing exactly
any finite number of given sets.


%-----------------------------------------------------------------
\levelstay{Axiom of union}
\label{sec:Axiom-of-union}

The union over the elements of a set 
exists\cite{wiki:Axiom_of_union}.

$\forall {\Set{F}}\,
\exists A\,\
forall Y\,\forall x
[(x\in Y\land Y\in {\Set{F}})\Rightarrow x\in A]$

Using \nameref{sec:Axiom-schema-of-replacement}:

${\displaystyle 
\cup {\Set{F}}:=
\{x\in A:\exists Y(x\in Y\land Y\in {\Set{F}})\}.}$

%-----------------------------------------------------------------
\levelstay{Axiom schema of replacement}
\label{sec:Axiom-schema-of-replacement}

%-----------------------------------------------------------------
\levelstay{Axiom of infinity}
\label{sec:Axiom-of-infinity}

%-----------------------------------------------------------------
\levelstay{Axiom of power set}
\label{sec:Axiom-of-power-set}

%-----------------------------------------------------------------
\levelstay{Well-ordering theorem}

%-----------------------------------------------------------------
%-----------------------------------------------------------------
\setcounter{currentlevel}{\value{baseSectionLevel}}
\levelstay{Church, G\"{o}del, Turing}


%-----------------------------------------------------------------
%-----------------------------------------------------------------
\setcounter{currentlevel}{\value{baseSectionLevel}}
\levelstay{Libraries for exact or multi-precision arithmetic}

\setcounter{currentlevel}{\value{baseSectionLevel}-1}
\levelstay{GMP}

\setcounter{currentlevel}{\value{baseSectionLevel}-1}
\levelstay{MPFR}

\setcounter{currentlevel}{\value{baseSectionLevel}-1}
\levelstay{ICReals}

\cite{Briggs:2006,Briggs:XRC:2013}

\setcounter{currentlevel}{\value{baseSectionLevel}-1}
\levelstay{XRC}

\cite{Briggs:2006,Briggs:XRC:2013}

\setcounter{currentlevel}{\value{baseSectionLevel}-1}
\levelstay{LEDA}

\cite{Burnikel:1996,Burnikel:1999,Mehlhorn:1995,LEDA:2009}

\setcounter{currentlevel}{\value{baseSectionLevel}-1}
\levelstay{CGAL}

\setcounter{currentlevel}{\value{baseSectionLevel}-1}
\levelstay{Core2}


\cite{Karamcheti:1999}

\setcounter{currentlevel}{\value{baseSectionLevel}-1}
\levelstay{iRRAM}

\setcounter{currentlevel}{\value{baseSectionLevel}-1}
\levelstay{CRcalc}

\setcounter{currentlevel}{\value{baseSectionLevel}-1}
\levelstay{Spire}

\cite{Spire:2019}

\setcounter{currentlevel}{\value{baseSectionLevel}-2}
\levelstay{Rational}

\setcounter{currentlevel}{\value{baseSectionLevel}-2}
\levelstay{Algebraic}

Based on
\cite{yap:guaranteed:2004,li-pion-yap:progress:2004,pion-yap:kary:2003,Pion:2006,Li:2001,Burnikel:2001}

\setcounter{currentlevel}{\value{baseSectionLevel}-2}
\levelstay{Real}

Based on \cite{Lester:2012}.

 