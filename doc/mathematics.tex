%-----------------------------------------------------------------
\setcounter{currentlevel}{\value{baseSectionLevel}}
\levelstay{Mathematical Structures}

A mathematical structure~\cite{wiki:Mathematical-structure} 
consists of a few sets, and functions between those sets.
Types of structure:
\begin{description}
\item[Algebraic] 
Characterized by operations (functions of $2$ arguments, 
sometimes more) that have given properties, like commutativity:
$\left(+ \, a \, b \right) = \left(+ \, b \, a \right)$.

\item[Order and equivalence] 
A set plus a one or more relations. 
See~\ref(sec:Relations).

\item[Topology] A primary set plus a family of 'open' subsets of the
primary set. The subsets have to obey certain properties which 
enable to be used to define what is 'connected' to what in the 
primary set.

\item[Metric] A set plus a \textit{distance} function,
mapping each pairs of elements to non-negative real number,
having certain properties.
Metric/distance induces a topology:
$\Set{B}(x_0,r) = \SetSpec{ x \in \Set{X} }{ 
\text{distance} \left(x_0 , x \right) < r }$ 
See~\ref{sec:Metric-spaces}.

\item[Measure] A set plus a function that assigns a non-negative
real value to some subsets of the primary set. 
Examples are length, area, volume.

\item[Geometry] \textbf{TODO:}
\end{description}

For \glssymbol{RealNumbers}:
Order and Metric induce Topology.
Order and Algebraic structure lead to ordered field.
Algebraic structure and topology make Lie group.

There is usually a primary set, whose elements are the 'elements'
of the structure:

\begin{example}[Linear Space]
\bigskip
A \textit{linear space} $\Space{V}$ is:
\begin{itemize}
  \item a set of vectors $\Set{V}$,
  \item a field of scalars $\Space{F}$,
  \item a linear combination operation/function: 
\begin{equation}
\left( \text{linear-combination} 
\, a_0 \, \Vector{v}_0 \, a_1 \, \Vector{v}_1 \right) \; 
= \; a_0*\Vector{v}_0 + a_1*\Vector{v}_1
\; \rightarrow \; \Vector{v}_2  \in \Set{V}
\end{equation}
for $\Vector{v}_0, \Vector{v}_1 \in \Set{V} $
and $a_0, a_1 \in \Space{F}$.
Linear combination is often defined in terms of
$2$ other operations:
scalar multiplication $a * \Vector{v} \in \Set{V}$,
and vector addition $\Vector{v}_0 + \Vector{v}_1 \in \Set{V}$
\end{itemize}
Usually the distinction between $\Set{V}$ and 
$\Space{V} = \left[ \Set{V}, \Space{F}, \text{linear-combination} \right]$
is ignored.
\end{example}

It's tempting to identify a mathematical structure with a class,
but that won't work, because you will need to support multiple
representations (dense and sparse vectors, \texttt{double}
and \texttt{BigFraction}) and the functions
operate on multiple sets (vectors and scalars)
each with multiple representations.
Interfaces also don't work, because 
operations are typically functions of more than one argument,
with multiple representations for each.

Implementation is easier with \textit{generic functions}
(aka ``multimethods'').
Or context object determining what $\left(+ \, a \, b \right)$
means for any acceptable implementation of $a$ and $b$.

%-----------------------------------------------------------------
\setcounter{currentlevel}{\value{baseSectionLevel}}
%-----------------------------------------------------------------
\setcounter{currentlevel}{\value{baseSectionLevel}}
\levelstay{Sets}
\index{Set}
\epigraph{A pack of wolves, a bunch of grapes, or a flock of
pigeons are all examples of sets of things.
The mathematical concept of a set can be used as the foundation
for all known mathematics.
The purpose of this little book is to develop the basic properties
of sets.\\
\ldots \\
One thing that this development will not include is a definition
of sets.
The situation is analogous to the familiar axiomatic approach to
elementary geometry.
That approach does not offer a definition of points and lines;
instead it describes what it is that one can do with those
objects.
The semi-axiomatic point of view adopted here assumes that the
reader has the ordinary, human, intuitive (and frequently
erroneous) understanding of what sets are; the purpose of the
exposition is to delineate some of the things one can correctly do
with them.}%
{Halmos, \textit{Naive Set Theory}\cite{Halmos1960Naive}}

\vfill

\emph{Sets} are a foundation for everything we are going to do,
but they turn out to be surprisingly hard to define precisely,
from first principles, without introducing paradoxes.

\vfill

\label{sec:math-sets}
\lstset{language=Clojure}

%-----------------------------------------------------------------
\setcounter{currentlevel}{\value{baseSectionLevel}-1}
\levelstay{Defining sets}
\epigraph{All the basic principles of set theory, except only the
axiom of extension, are designed to make new sets out of old ones.
The first and most important of these basic principles of set
manufacture says, roughly speaking, that anything intelligent one
can assert about the elements of a set specifies a subset, namely,
the subset of those elements about which the assertion is true.}
{Halmos, \textit{Naive Set Theory}~\cite[section~2]{Halmos1960Naive}}

\begin{description}

\item[Specification]

A \gls{Set}, at its most fundamental, is a defined by a rule,
or method,
for determining if any given 'thing' is an element: 
$s \glssymbol{elementOf} \Set{S}$.
\footnote{$(s \notin \Set{S}) = \text{not}(s \in \Set{S})$.}
($s \not\in \Set{S}$)
%, or in pseudocode:
%\lstinline|(element-of $\Set{S}$ $s$)|.
An example, in standard set notation:
the even numbers 
$= \SetSpec{ i }{ \, i/2 \, \text{is an integer} }$.

\item[Enumeration]

However, it's difficult to do much interesting with a set
unless we have some way of finding, or generating elements.
So another common way to specify a set is by enumeration:
For example, the even numbers $= \{ \ldots, -4, -2, 0, 2 ,4,
\ldots \}$.

The enumeration notation above is something of a cheat ---
although it's straightforward enough for finite sets, it relies on
the reader correctly recognizing the implied pattern for countable
sets.
And it fails for uncountable sets like the real numbers.

\end{description}

What's unstated, but required, in both these approaches, is the
existence of some prior 'universe' set, $\Set{U}$, and a function
(see~\autoref{sec:Functions}) we can apply to the elements of
$\Set{U}$ to determine the set of interest:

\begin{description}
\item[Specification] 
$\Set{S} =
\SetSpec{ u \glssymbol{elementOf} \Set{U} }{ p(u) = \text{true} }$ 

For example: the even numbers
 $= \SetSpec{ i \glssymbol{elementOf} 
\glssymbol{Integers} }{
 i/2 \glssymbol{elementOf} \glssymbol{Integers} }$.

\item[Enumeration]
$\Set{S} =
\SetSpec{ f(u) }{ u \glssymbol{elementOf} \Set{U} }$ 

For example: the even numbers 
$= \SetSpec{ 2*i }{ i \glssymbol{elementOf} \glssymbol{Integers} }$.

\end{description}
A variation on 
``turtles all the way down''~\cite{xkcd:Turtles, wiki:Turtles},
this approach may be a bit unsatisfying.
It does, however, have the advantage of preventing self-reference
paradoxes (for example, 
the set of all sets that do not contain themselves); 
see Halmos, 
\textit{Naive Set Theory}~\cite[section 2]{Halmos1960Naive} 
for details.

% \gls{empty set}
One set that doesn't require any turtles is 
the empty set, $\varnothing$.


Both approaches raise the fascinating issue of 
computability/decidability~\cite{church1936unsolvable,
turing1936computability,
turing1938computable-correction,
turing1937computability-lambda},
which I am going to pass over here, except to note the 
'halting problem'~\cite{wiki:Halting-problem}:

If we call $p(u)$ to determine if $u \in \Set{S}$, 
it might return $\text{true}$, might return $\text{false}$,
or it might not return at all.
In the third case, we might define $u \in \Set{S}$ as only those
$u$ where $p(u)$ halts and returns $\text{true}$, but it turns 
out that given $p$ and $u$, deciding whether $p(u)$ will ever
finish is itself undecidable (more turtles \ldots).

The practical version of this is that, even in cases where we
might be able to determine that $p(u)$ that will eventually halt
and return something, we might not be able to afford to wait that
long.
Pragmatically, we will have to restrict ourselves to sets where we
can guarantee an answer fast enough for the context in which the
sets are being used.

%-----------------------------------------------------------------
\setcounter{currentlevel}{\value{baseSectionLevel}-1}
\levelstay{Cardinality}

% \gls{cardinality}  \gls{countable infinity} \gls{uncountable
% infinity} \gls{cardinal numbers}
The cardinality of a set in the number of elements.
For our purposes, the possibilities that matter are \emph{finite,}
\emph{countably infinite}, the cardinality of the
\gls{NaturalNumbers}, $\aleph_{0}$, or \emph{uncountably
infinite}, which means the set has more elements than the
\gls{NaturalNumbers}.
(I'm ignoring the distinction between different uncountable
infinities.
See~\cite{wiki:cardinal-number} for an introduction to the other
possibilities.)

Notation for set cardinality varies widely:
$\#\Set{S}$,  $|\Set{S}|$,
$\bar{\bar{\Set{S}}}$, $n(\Set{S})$,
$\text{card}(\Set{S})$, \ldots, nearly all of which conflict
with something (eg $\#\{ 0 \, 1 \, 2 \} = 3$ vs 
\lstinline|#{1 2 3}| Clojure set syntax).
I will use $\text{cardinality}(\Set{S})$.

%-----------------------------------------------------------------
\setcounter{currentlevel}{\value{baseSectionLevel}-1}
\levelstay{Identity}

The question of whether two sets are the same set is a recurring
locus of ambiguity in mathematics.

One issue is the difference between the 'set' and the 
'set definition', which is rarely made clear.

Halmos~\cite{Halmos1960Naive} starts with the \emph{Axiom of
extension:} two sets are the \emph{same thing} if they have the
same elements (regardless of how they are defined).

Example: 
$\SetSpec{ 2*i }{ \, i \in \glssymbol{Integers} }$
and
$\SetSpec{ i \in \glssymbol{Integers} }
{ i/2 \in \glssymbol{Integers} }$
are two distinct definitions of the same set. 
 
Pragmatically, however, it's going to be much easier to determine
if two set definitions are the same, than determining if two
different definitions generate the same elements, which would,
after all, be undecidable in general.

Another source of ambiguity is a common tendency to blur the 
distinction between sets whose elements are related by some natural
identity. 

Example: we can define the rational numbers as (equivalence
classes of) pairs of integers:
$\glssymbol{RationalNumbers} = \glsdesc{RationalNumbers}$.
Strictly speaking, the rational number $i/1$ is a different thing
from the integer $i$, but it's nearly universal to identify the
subset of the rationals with denominator $1$ with the integers and
take $\glssymbol{Integers} \subset \glssymbol{RationalNumbers}$.

%-----------------------------------------------------------------
\setcounter{currentlevel}{\value{baseSectionLevel}-1}
\levelstay{Subsets, partitions, and quotients}

Families: 
$\SetSpec{ \Set{S}_{i} }{ i \in \, \text{index set} \, \Set{I} }$.


%\gls{subset} \gls{superset}

$\Set{A} \subseteq \Set{B}$ ($\Set{B} \supseteq \Set{A}$)
means
every element of $\Set{A}$ is an element of $\Set{B}$, 
that is, $a \in \Set{A}$ 
implies $a \in \Set{B}$.
I reserve $\Set{A} \subset \Set{B}$ (($\Set{B} \supset
\Set{A}$)) for strict subsets;
it means $\Set{A} \subseteq \Set{B}$
and there is some $\b\in\Set{B}$ which
is not in $\Set{A}$.

%\gls{power set}

The \emph{power set}, $\PowerSet{\Set{S}}$,
 is the set 
of all subsets of
$\Set{S}$.

%\gls{intersection} \gls{union} \gls{set-difference}
As usual, $\Set{A} \cap \Set{B}$ is the set of elements in both 
$\Set{A}$ and $\Set{B}$; $\Set{A} \cup \Set{B}$ are those things
that are in either $\Set{A}$ or $\Set{B}$.

$\Set{A} \cap \Set{B} \cap \Set{C} 
= (\Set{A} \cap \Set{B}) \cap \Set{C} 
= \Set{A} \cap (\Set{B} \cap \Set{C}) $

$\bigcap_{i \in \Set{I}} \Set{S}_{i} = 
\SetSpec{ s }{ s \in \Set{S}_{i} \forall \Set{S}_{i} }$

$\Set{A} \setminus \Set{B}$ are the elements of $\Set{A}$ 
not in $\Set{B}$.

%\gls{partition}
A partition of $\Set{S}$ is a set of disjoint nonempty subsets of
$\Set{S}$ whose union is $\Set{S}$.

%-----------------------------------------------------------------
\setcounter{currentlevel}{\value{baseSectionLevel}-1}
\levelstay{Implementation}

%-----------------------------------------------------------------
\setcounter{currentlevel}{\value{baseSectionLevel}-2}
\levelstay{Java}
\lstset{language=Java}

Java provides an interface \lstinline|java.util.Set| intended for
possibly mutable, finite sets (\autoref{java.util.Set:general},
 \autoref{java.util.Set:countable}, 
 \autoref{java.util.Set:finite}, 
 and
\autoref{java.util.Set:optional}).

%-----------------------------------------------------------------
\begin{lstlisting}[
 caption={[\texttt{java.util.Set} general]\texttt{java.util.Set} 
 operations applicable to any set.}, 
 label=java.util.Set:general]
boolean contains (Object o) //$ \; o \in \Set{S}$
boolean containsAll (Collection c) //$\;\Set{C}\subseteq\Set{S}$ 
boolean isEmpty () //$ \; \Set{S} = \varnothing $
boolean equals (Object o) //$\; \Set{S} = \Set{O} $
\end{lstlisting}
%-----------------------------------------------------------------
\begin{lstlisting}[
 caption={[\texttt{java.util.Set} countable]\texttt{java.util.Set} 
 operations requiring countable sets. Note, however, that 
 iterators that never end will cause havoc with almost all Java
 code}, 
 label=java.util.Set:countable]
Iterator iterator ()
Spliterator spliterator ()
\end{lstlisting}
%-----------------------------------------------------------------
\begin{lstlisting}[
 caption={[\texttt{java.util.Set} finite]\texttt{java.util.Set} 
 operations requiring finite sets. Note that changing to
 \texttt{size()} to be Object valued would enable representing sets
 of arbitrary cardinality.}, 
 label=java.util.Set:finite] 
int size () 
Object[] toArray ()
Object[] toArray (Object[] a)
\end{lstlisting}
%-----------------------------------------------------------------
\begin{lstlisting}[
 caption={[\texttt{java.util.Set}]\texttt{java.util.Set} optional
 operations, requiring mutable sets.}, 
 label=java.util.Set:optional,]
boolean add(E e) //$\; \Set{S} \leftarrow \Set{S} \cup \{e\} $
boolean addAll(Collection c) //$\;\Set{S}\leftarrow\Set{S}\cup\Set{C}$ 
void  clear() //$\; \Set{S} \leftarrow \varnothing $ 
boolean remove (Object o) //$\;\Set{S}\leftarrow\Set{S}\setminus\{e\}$
boolean removeAll (Collection c) //$\;\Set{S}\leftarrow\Set{S}\setminus\Set{C}$ 
boolean retainAll (Collection c) //$\;\Set{S}\leftarrow\Set{S}\cap\Set{C}$
\end{lstlisting}
%-----------------------------------------------------------------

More Java set classes? Guava or Apache Commons?
Set operations, union, cartesian products, \ldots.

%-----------------------------------------------------------------
\setcounter{currentlevel}{\value{baseSectionLevel}-2}
\levelstay{Clojure}
\lstset{language=Clojure}

Idiomatic Clojure sets are immutable, although it provides easy
access to any mutable or immutable Java implementation of
\lstinline|java.util.Set|.

Clojure provides an (unfortunate) literal syntax for finite
enumerated sets: 
\lstinline|#{0 1 2}|, and 4 ways to create sets:
\begin{description}
\item[\texttt{hash-set}] A function equivalent to the
literal syntax \lstinline|#{}|
\item[\texttt{sorted-set}/\texttt{sorted-set-by}] Returns
sets that iterate over their elements in their natural order (in the
order of the supplied comparator). Like Java sorted collections,
can't handle partial orderings.
\item[\texttt{set}] Coerce any collection into a set.
\item[\texttt{into}] A general way to coerce any collection
into another type.
\end{description}

Idiomatic Clojure provides an informally specified functional
'API' for finite sets. Most of these functions do will something
for all Clojure collections, not always what you would expect:
\begin{lstlisting}[
 caption={Clojure set 'API'}, 
 label=Clojure:set-API,]
(contains? s x) ;;  $x \in \Set{S}$.
(empty? s) ;; $\Set{S} = \varnothing$
(count s) ;; $\text{cardinality}(\Set{S})$
(conj s x) ;; $\Set{S} \cup \{x\}$
(disj s x) ;; $\Set{S} \setminus \{x\}$
(clojure.set/intersection s0 s1 s2 $\ldots$) ;; $\Set{S}_0 \cap \Set{S}_1 \cap \Set{S}_2 \cap \ldots$ 
(clojure.set/union s0 s1 s2 $\ldots$) ;; $\Set{S}_0 \cup \Set{S}_1 \cup \Set{S}_2 \cup \ldots$ 
(clojure.set/difference s0 s1 s2 $\ldots$) ;; $\Set{S}_0 \setminus (\Set{S}_1 \cup \Set{S}_2 \cup \ldots)$
\end{lstlisting}

Confusion with dictionary 'API':
 \lstinline|(s x)| and \lstinline|(get s x)| are similar to
 \lstinline|(contains? s x)| except \lstinline|(contains? s x)| returns
 \lstinline|true| or \lstinline|false|, while the other two return
 \lstinline|x| if \lstinline|x| is in \lstinline|s| \lstinline|nil| otherwise.
The behavior of \lstinline|(s x)| and \lstinline|(get s x)| reflect an
incomplete/inconsistent API treating Clojure collections as
dictionaries (key-value pairs), modeling sets as dictionaries
where the key and value are constrained to be the same.
However, other functions designed for dictionaries (eg
\lstinline|keys|) don't work for sets.
 
%-----------------------------------------------------------------
\setcounter{currentlevel}{\value{baseSectionLevel}-2}
\levelstay{Les Elemens}

Set interface.

Implementation almost always requires multiple objects
representing a single set element

\begin{equation}
a \parallel b
\end{equation}
vs
\begin{equation}
a = b
\end{equation}

vs
\begin{equation}
a \equiv b
\end{equation}

% \begin{lstlisting}[
% caption={[checksum namespace]Checksum and related functions via
% Java calls}, label=checksum-namespace,]
% (set! *warn-on-reflection* true)
% (set! *unchecked-math* :warn-on-boxed)
% (ns ^{:doc "compute a file checksum."}
%   
%   curate.scripts.checksum
%   
%   (:require [clojure.java.io :as io])
%   (:use [clojure.set :only [difference]])
%   (:gen-class))
% ;;-----------------------------------------------------------------
% (defn checksum [file]
%   (let [input (java.io.FileInputStream. file)
%         digest (java.security.MessageDigest/getInstance "MD5")
%         stream (java.security.DigestInputStream. input digest)
%         bufsize (* 1024 1024)
%         buf (byte-array bufsize)]
% 
%   (while (not= -1 (.read stream buf 0 bufsize)))
%   (apply str (map (partial format "%02x") (.digest digest)))))
% 
% (defn list-dir [dir]
%   (remove #(.isDirectory %)
%           (file-seq (java.io.File. dir))))
% 
% (defn find-dupes [root]
%   (let [files (list-dir root)]
%     (let [summed (zipmap (pmap #(checksum %) files) files)]
%       (difference
%        (into #{} files)
%        (into #{} (vals summed))))))
% 
% (defn remove-dupes [files]
%   (prn "Duplicates files to be removed:")
%   (doseq [f files] (prn (.toString f)))
%   (prn "Delete files? [y/n]:")
%   (if-let [choice (= (read-line) "y")]
%     (doseq [f files] (.delete f))))
% 
% (defn -main [& args]
%   (if (empty? args)
%     (println "Enter a root directory")
%     (remove-dupes (find-dupes (first args))))
%   (System/exit 0))
% \end{lstlisting}
% 
% \lstinputlisting[
% float=htbp,
% caption={[checksum function]Compute a checksum for a file thru Java},
% label=checksum,
% firstline=11,lastline=18]{listings/checksum.clj}
 
%-----------------------------------------------------------------
\setcounter{currentlevel}{\value{baseSectionLevel}-1}
\levelstay{Examples}
 
1d and 2d intervals.
 
% \begin{minipage}{\linewidth}
% \begin{lstlisting}[
% caption={[FileTypeFilter]A file .suffix filter},
% label=FileTypeFilter]
% package img;
% 
% public final class FileTypeFilter 
%   implements java.io.FilenameFilter {
% 
%   private final String _suffix;
% 
%   public final boolean accept (final java.io.File dir,
%                                final String name) {
%     return name.toLowerCase().endsWith(_suffix); }
% 
%   public FileTypeFilter (final String suffix) {
%     super();
%     _suffix = suffix.toLowerCase(); } }
% \end{lstlisting}
% \end{minipage}
% 
% \index{Sets|)}
% 
% \begin{figure}[htbp]
% \centering
% \includegraphics[scale=0.7]{figs/arara.png}
% \caption{Configuring {\TeX}works for \texttt{arara}.}
% \end{figure}

%-----------------------------------------------------------------
\setcounter{currentlevel}{\value{baseSectionLevel}}
%-----------------------------------------------------------------
\setcounter{currentlevel}{\value{baseSectionLevel}}
\levelstay{Functions and relations}
\label{sec:Functions-and-relations}

In mathematics literature, \textit{function} is usually defined as
a special kind of \textit{relation} --- essentially a set of
ordered pairs with certain properties.

This approach is useful for some purposes, but here we will be
more interested in ``computable'' functions, at least in the sense
that a function is a ``machine'' that takes an element of one set
as input and returns an element of another, with the constraint
that a given input always returns the same output.

Because we need the notion of ``relation'' anyway, I'm going to
provide both definitions.

%-----------------------------------------------------------------
\setcounter{currentlevel}{\value{baseSectionLevel}-1}
\levelstay{Relations}
\label{sec:Relations}

A \textit{relation}, 
$\Set{R}$, on $\Set{S}_0, \Set{S}_1, \ldots \Set{S}_{n-1}$,  
is any subset 
$\Set{R} \subseteq \Set{S}_0 \times \Set{S}_1 \times \ldots 
\times \Set{S}_{n-1}$,
that is, a set of tuples.

Ambiguity note: a given set of tuples, $\Set{R}$, can be
considered as a relation over many cartesian product sets.
The minimal such set is 
$\Set{R}_0 \times \Set{R}_1 \times \ldots \times
\Set{R}_{n-1}$, 
where 
$\Set{R}_i = \SetSpec{ x }{ \exists r 
\in \Set{R} \text{ s.t. } r_i = x }$.

A general \textit{binary relation} is a set of ordered
pairs $\Set{R} \subseteq \Set{S}_0 \times \Set{S}_1$.
It's common to write a binary relation as a predicate, 
$r(s_0,s_1) = ([s_0 \, s_1] \in \Set{R})$,
or $(r \, s_0 \, s_1)$ in pseudo-code,
or as a binary operation $s_0 \, R \, s_1$.

An important special case
are binary relations on $\Set{S}^2 = \Set{S} \times \Set{S}$
(often just written as ``binary relation on $\Set{S}$''). 
In this case we can define certain properties:

\begin{description}
\item[Transitive]
$r(s_0,s_1) \text{ and } r(s_1,s_2) \Rightarrow r(s_0,s_2)$.
\item[Reflexive] $r(s,s)$ is always true.
\item[Symmetric] For $s_0 \neq s_1$, $r(s_0,s_1)$ implies
$r(s_1,s_0)$.
\item[Antisymmetric] For $s_0 \neq s_1$, $r(s_0,s_1)$ implies not
$r(s_1,s_0)$.
\end{description}
These properties determine 2 important classes of relations:
\begin{description}
\item[Equivalence] Transitive, reflexive, and symmetric.
\item[Partial order] \label{def:partial-order}
Transitive, reflexive, and antisymmetric.
\end{description}

%-----------------------------------------------------------------
\setcounter{currentlevel}{\value{baseSectionLevel}-1}
\levelstay{Functions}
\label{sec:Functions}

A \textit{functional} binary relation, $\Set{F}$ on $\Set{X}
\times \Set{Y}$ has exactly one $y \in \Set{Y}$ for each
$x \in \Set{X}$ such that $[x \, y] \in \Set{F}$.
More conventional notation writes the \textit{function}, 
$f : \Set{X} \rightarrow \Set{Y}$ as $y = f(x)$.
In that case, $\Set{X}$ is the \textit{domain} of $f$
and $\Set{Y}$ is the $codomain$.

It is nearly universal to extend or identify a function $f$ that maps 
elements of the domain $\Set{X}$
to elements of 
the codomain $\Set{Y}$
the closely related function that maps
subsets of $\Set{X}$ to subsets of $\Set{Y}$,
via $f\left( \Set{A} \right) = 
\SetSpec{y \in \Set{Y}}{\exists \, x \in \Set{A} 
\text{ such that } f\left( x \right) = y}$

The \textit{range} of $f$ is then $f\left( \Set{X} \right)$.

Arbitrariness of codomain:
Note that $\text{range}\left( f \right)$ 
is unambiguous.
If  $f_0 : \Set{X} \rightarrow \Set{Y}_0$,
and $\Set{Y}_0 \subset \rightarrow \Set{Y}_1$,
then is $f_1 : \Set{X} \rightarrow \Set{Y}_1$
defined by $f_1(x) = f_0(x)$
the same function as $f_0$?

We will also frequently encounter the notion of the 
\textit{support} of a function, which is less consisntently
defined. In general, the support is a subset of the domain
where the function takes on some non-trivial value, for example,
where a numerically valued function is non-zero, or sometimes,
finite.

Partial functions: support vs domain.

Any function, $f : \Set{X} \rightarrow \Set{Y}$ defines an
equivalence relation on $\Set{X}$ via 
$\Set{E}_f = \SetSpec{[x_0 \, x_1]}{f(x_0) = f(x_1)}$
(see~\ref{sec:Implicit-functions}).

$\Set{F}\left( \Set{D}, \Set{C} \right)$ is the set of all 
functions with domain $\Set{D}$ and codomain $\Set{C}.$



%-----------------------------------------------------------------
\setcounter{currentlevel}{\value{baseSectionLevel}-1}
\levelstay{Equivalence classes and quotient sets}

If $\Set{E}$ is an equivalence relation on $\Set{S}^2$, then we
can define the \textit{equivalence class} of $s_0$, $E(s_0) = \{
\SetSpec{s_1 \in \Set{S}}{[s_0 \, s_1] \in \Set{E} }$.
The set of distinct equivalence classes partitions $\Set{S}$,
and is called the \textit{quotient set}: $\Set{S} / \Set{E}$.

In the case where the equivalence relation is derived from a
function, we write $\Set{S} / f$.

Equivalence classes and quotient sets will turn out to be
important. A common representational/implementation trick is to
use a larger but simpler (in some sense) space for calculations
which are meant to apply to a quotient space that actually has the
properties of interest. Homogeneous coordinates for affine and
projective spaces  are an important example.
The first one we will encounter here will be the rational numbers,
which are represented/implemented as pairs of integers
$[\text{numerator} \, \text{denominator}]$, but the actual
rational numbers are the equivalence classes defined by 
$\SetSpec{ [[p_0 \, q_0] \, [p_1 \, q_1]]}{p_0*q_1 = p_1*q_0 }$,
that is, $[1 \, 2]$ and $[2 \, 4]$ represent the same rational.

%-----------------------------------------------------------------
\setcounter{currentlevel}{\value{baseSectionLevel}-1}
\levelstay{Inverses and pseudo-inverses}
\label{sec:Inverses-and-pseudo-inverses}

Given a function $f : \Set{D} \mapsto \Set{C}$,
the \textit{inverse} of $f$ is 
$f^{-1}(c) = \SetSpec{d}{f(d) = c}$.
Note that $f^{-1} : \Set{C}  \mapsto \PowerSet{\Set{D}}$.

$\Set{L}\left( f , c \right) = f^{-1}(c)$ is a \textit{level set} of
$f$ at $c$.

For more symmetry, we can extend $f$ and $f^{-1}$ to 
$\PowerSet{\Set{D}}  , \PowerSet{\Set{C}}$
by defining
$f(\Set{S}) = 
\SetSpec{c}{\exists \, d \, \in \Set{S} \; \text{s.t.} \; c = f \left( d \right)}$

The usual definition of inverse treats $f^{-1}$
as a function from $\Set{C} \mapsto \Set{D}$,
which is undefined where the value of the true
inverse is not a set containing a single point.

\textbf{TODO:} When is a function 
$\PowerSet{\Set{D}}  \mapsto \PowerSet{\Set{C}}$
derivable from a function $\Set{D} \mapsto \Set{C}$?

Inverse function theorem~\cite[Theorem~2-1]{spivak-1965}
for $\Space{R}^n$. 

Generalizations?~\cite{wiki:Inverse-function-theorem}

%-----------------------------------------------------------------
\setcounter{currentlevel}{\value{baseSectionLevel}-1}
\levelstay{Level sets and implicit functions}
\label{sec:Implicit-functions}

Suppose 
$f : \left( \Set{D}_0 \times \Set{D}_1 \right) \mapsto \Set{C}$.
Then each level set $f^{-1}\left( c \right)$ defines a relation on 
$\Set{R}_{f^{-1}(c)} \left( \Set{D}_0 , \Set{D}_1 \right)$.
If that relation is a function
(one $d_1$ paired to each $d_0$),
we call it an \textit{implicit function},
which we might write $g_{f^{-1}(c)} : \Set{D}_0 \mapsto \Set{D}_1$

Implicit functions are generally difficult to use,
because they don't tell us how to compute 
$g_{f^{-1}(c)} \left( d_0 \right)$.
Essentially, one has to use an iterativc zero-finding 
algorithm, which is difficult and non-robust except
for very low dimensional problems.
An alternative is to minimize something like
$\| c - f\left( d_0, d_1 \right) \|^2$ over $d_0$,
and check that the minimum value is close enough to $0$.

\textbf{TODO:} When does a level set correspond to a function?
Implicit function theorem~\cite[Theorem~2-2]{spivak-1965}
for $\Space{R}^n$. 

Generalizations?~\cite{wiki:Implicit-function-theorem,
wiki:Inverse-function-theorem}

%-----------------------------------------------------------------
\setcounter{currentlevel}{\value{baseSectionLevel}-1}
\levelstay{Multiple arguments}
\label{sec:Multiple-arguments}

Often useful to consider 
functions of several arguments
$f \left( x_0, x_1, \ldots x_{n-1}\right)$.

Simpler in general to deal with functions of one argument.
Two ways to identify a function of several argmuments
with a single argument function:
\begin{description}
\item[Catesian product domains]
Define $f \left( x_0, x_1, \ldots x_{n-1} \right) 
= f \left( \left[x_0, x_1, \ldots , x_{n-1} \right]\right)$
where $\left[x_0, x_1, \ldots x_{n-1} \right] \in
\Set{X}_0 \times \Set{X}_1 \times \ldots \times \Set{X}_{n-1}$. 
(Note that domain/support may be a strict subset of full cartesian 
product.)
\item[Currying]
In pseudo-lisp: $\left(f \, x_0 \, x_1 \, \ldots x_{n-1} \right) 
= 
\left(
\left(
\ldots 
\left( 
\left( 
f 
\, x_0 
\right) 
\, x_1
\right) 
\ldots 
\right) 
\, x_{n-1}
\right) 
$.
This is easier to follow in the $2$ argument case.
Then we interpret $f : \Set{X}_0 \rightarrow 
\Set{F}\left(\Set{X}_1, \Set{Y} \right)$,
that is, the value $f \left( x_0 \right)$ is a function 
$\Set{X}_1 \mapsto \Set{Y}$,
so, in confusing infix notation,
$f \left( x_0 , x_1 \right) = 
f \left( x_0 \right) \left( x_1 \right) \Rightarrow y \in \Set{Y}$,
or, in pseudo-lisp, 
$\left( f \, x_0 \, x_1 \right) = \left( \left( f \, x_0 \right) x_1 \right) \Rightarrow y$
\end{description}

%-----------------------------------------------------------------
\setcounter{currentlevel}{\value{baseSectionLevel}-1}
%\levelstay{Computable functions}

%-----------------------------------------------------------------
\setcounter{currentlevel}{\value{baseSectionLevel}-1}
%\levelstay{implementation}

%-----------------------------------------------------------------
\setcounter{currentlevel}{\value{baseSectionLevel}-1}
%\levelstay{examples}


%-----------------------------------------------------------------
\setcounter{currentlevel}{\value{baseSectionLevel}}
%-----------------------------------------------------------------
\setcounter{currentlevel}{\value{baseSectionLevel}}
\levelstay{Cartesian products}
\label{sec:Cartesian-products}

Combining and factoring sets and functions.

Factor set into product of other sets.

Factor function into 'coordinate' functions:
$f : \Set{X} \rightarrow \Set{Y}_0 \times \Set{Y}_1$
via
$f \left( x \right) = 
\left[ f_0\left( x \right) , f_1\left( x \right) \right]$

Also 'reduce':
$f : \Set{X}_0 \times \Set{X}_1 \rightarrow \Set{Y}$
via 
 $f\left( \left[ x_0, x_1 \right] \right)
= r \left(
\left[ f_0 \left( x_0 \right) , f_1 \left( x_1 \right) \right] 
\right)$,
eg, for linear functions, 
 $f\left( \left[ x_0, x_1 \right] \right)
=  f_0 \left( x_0 \right) + f_1 \left( x_1 \right)$,

%-----------------------------------------------------------------
\setcounter{currentlevel}{\value{baseSectionLevel}-1}
\levelstay{Cartesian product sets}
\label{sec:Cartesian-product-sets}

% \gls{tuple}
A tuple is a fixed length list, which I will write, for example,
$[a \, b \, c]$, in a Clojure-like syntax,
or sometimes as $a \times b \times c$.

% \gls{cartesian-product}
The cartesian product set 
$\Set{A} \times \Set{B} =
\SetSpec{[a \, b] }{ a \in \Set{A} \text{ and } b \in \Set{B} }$.

Extending this notation beyond two terms,
$\Set{A} \times \Set{B} \times \Set{C}$,
introduces ambiguity typically ignored in mathematics literature,
using the implied natural identity mentioned above.
Strictly speaking,
\begin{itemize}
  \item $\Set{A} \times \Set{B} \times \Set{C} = 
  \SetSpec{[a \, b \, c] }{ \ldots }$
  \item $\Set{A} \times (\Set{B} \times \Set{C}) = 
  \SetSpec{[a \, [b \,c]] }{ \ldots }$
  \item $(\Set{A} \times \Set{B}) \times \Set{C} = 
  \SetSpec{[[a \, b] \, c]] }{ \ldots }$
\end{itemize}
are 3 distinct sets.

%-----------------------------------------------------------------
\setcounter{currentlevel}{\value{baseSectionLevel}-1}
\levelstay{Generalized tuples}
\label{sec:Generalized-tuples}

Tuples as functions from index set to set elements.

'Subspace' corresponds to subset of index set.

Canonical projections and embeddings.

Generalized coordinates.

%-----------------------------------------------------------------
\setcounter{currentlevel}{\value{baseSectionLevel}-1}
\levelstay{Cartesion product functions}
\label{sec:Cartesion-product-functions}

Aka \textit{coordinate functions}.

Suppose $f : \Set{X} \rightarrow \Set{Y}_0 \times \Set{Y}_1$,
and $f \left( x \right) = \left[ y_0 ,y_1 \right]$
Then we can write $f = f_0 \times f_1$ where
$f_0 \left( x \right) = y_0 $
and
$f_1 \left( x \right) = y_1 $.

\textbf{TODO:} $f$ corresponds to a functional relation on 
$\Set{X} \times \left( \Set{Y}_0 \times \Set{Y}-1 \right)$).
Does this imply functional relations $f_i$ on
$\Set{X} \times \Set{Y}_i$?
How about $f_i = e_i \circ f$,
where $e_i : \times_j \Set{X}_j  \rightarrow \Set{X}_i$ by
$e_i \left( [x_0, x_1, \ldots , x_{n-1} \right) = x_i$ 

%-----------------------------------------------------------------
\setcounter{currentlevel}{\value{baseSectionLevel}-1}
\levelstay{Implementation}

%-----------------------------------------------------------------
\setcounter{currentlevel}{\value{baseSectionLevel}-2}
\levelstay{Java}
\lstset{language=Java}

Java provides an interface \lstinline|java.util.Set| intended for
possibly mutable, finite sets (\autoref{java.util.Set:general},
 \autoref{java.util.Set:countable}, 
 \autoref{java.util.Set:finite}, 
 and
\autoref{java.util.Set:optional}).

%-----------------------------------------------------------------
\begin{lstlisting}[
 caption={[\texttt{java.util.Set} general]\texttt{java.util.Set} 
 operations applicable to any set.}, 
 label=java.util.Set:general]
boolean contains (Object o) //$ \; o \in \Set{S}$
boolean containsAll (Collection c) //$\;\Set{C}\subseteq\Set{S}$ 
boolean isEmpty () //$ \; \Set{S} = \varnothing $
boolean equals (Object o) //$\; \Set{S} = \Set{O} $
\end{lstlisting}
%-----------------------------------------------------------------
\begin{lstlisting}[
 caption={[\texttt{java.util.Set} countable]\texttt{java.util.Set} 
 operations requiring countable sets. Note, however, that 
 iterators that never end will cause havoc with almost all Java
 code}, 
 label=java.util.Set:countable]
Iterator iterator ()
Spliterator spliterator ()
\end{lstlisting}
%-----------------------------------------------------------------
\begin{lstlisting}[
 caption={[\texttt{java.util.Set} finite]\texttt{java.util.Set} 
 operations requiring finite sets. Note that changing to
 \texttt{size()} to be Object valued would enable representing sets
 of arbitrary cardinality.}, 
 label=java.util.Set:finite] 
int size () 
Object[] toArray ()
Object[] toArray (Object[] a)
\end{lstlisting}
%-----------------------------------------------------------------
\begin{lstlisting}[
 caption={[\texttt{java.util.Set}]\texttt{java.util.Set} optional
 operations, requiring mutable sets.}, 
 label=java.util.Set:optional,]
boolean add(E e) //$\; \Set{S} \leftarrow \Set{S} \cup \{e\} $
boolean addAll(Collection c) //$\;\Set{S}\leftarrow\Set{S}\cup\Set{C}$ 
void  clear() //$\; \Set{S} \leftarrow \varnothing $ 
boolean remove (Object o) //$\;\Set{S}\leftarrow\Set{S}\setminus\{e\}$
boolean removeAll (Collection c) //$\;\Set{S}\leftarrow\Set{S}\setminus\Set{C}$ 
boolean retainAll (Collection c) //$\;\Set{S}\leftarrow\Set{S}\cap\Set{C}$
\end{lstlisting}
%-----------------------------------------------------------------

More Java set classes? Guava or Apache Commons?
Set operations, union, cartesian products, \ldots.

%-----------------------------------------------------------------
\setcounter{currentlevel}{\value{baseSectionLevel}-2}
\levelstay{Clojure}
\lstset{language=Clojure}

Idiomatic Clojure sets are immutable, although it provides easy
access to any mutable or immutable Java implementation of
\lstinline|java.util.Set|.

Clojure provides an (unfortunate) literal syntax for finite
enumerated sets: 
\lstinline|#{0 1 2}|, and 4 ways to create sets:
\begin{description}
\item[\texttt{hash-set}] A function equivalent to the
literal syntax \lstinline|#{}|
\item[\texttt{sorted-set}/\texttt{sorted-set-by}] Returns
sets that iterate over their elements in their natural order (in the
order of the supplied comparator). Like Java sorted collections,
can't handle partial orderings.
\item[\texttt{set}] Coerce any collection into a set.
\item[\texttt{into}] A general way to coerce any collection
into another type.
\end{description}

Idiomatic Clojure provides an informally specified functional
'API' for finite sets. Most of these functions do will something
for all Clojure collections, not always what you would expect:
\begin{lstlisting}[
 caption={Clojure set 'API'}, 
 label=Clojure:set-API,]
(contains? s x) ;;  $x \in \Set{S}$.
(empty? s) ;; $\Set{S} = \varnothing$
(count s) ;; $\text{cardinality}(\Set{S})$
(conj s x) ;; $\Set{S} \cup \{x\}$
(disj s x) ;; $\Set{S} \setminus \{x\}$
(clojure.set/intersection s0 s1 s2 $\ldots$) ;; $\Set{S}_0 \cap \Set{S}_1 \cap \Set{S}_2 \cap \ldots$ 
(clojure.set/union s0 s1 s2 $\ldots$) ;; $\Set{S}_0 \cup \Set{S}_1 \cup \Set{S}_2 \cup \ldots$ 
(clojure.set/difference s0 s1 s2 $\ldots$) ;; $\Set{S}_0 \setminus (\Set{S}_1 \cup \Set{S}_2 \cup \ldots)$
\end{lstlisting}

Confusion with dictionary 'API':
 \lstinline|(s x)| and \lstinline|(get s x)| are similar to
 \lstinline|(contains? s x)| except \lstinline|(contains? s x)| returns
 \lstinline|true| or \lstinline|false|, while the other two return
 \lstinline|x| if \lstinline|x| is in \lstinline|s| \lstinline|nil| otherwise.
The behavior of \lstinline|(s x)| and \lstinline|(get s x)| reflect an
incomplete/inconsistent API treating Clojure collections as
dictionaries (key-value pairs), modeling sets as dictionaries
where the key and value are constrained to be the same.
However, other functions designed for dictionaries (eg
\lstinline|keys|) don't work for sets.
 
%-----------------------------------------------------------------
\setcounter{currentlevel}{\value{baseSectionLevel}-2}
\levelstay{Les Elemens}

Set interface:

% \begin{lstlisting}[
% caption={[checksum namespace]Checksum and related functions via
% Java calls}, label=checksum-namespace,]
% (set! *warn-on-reflection* true)
% (set! *unchecked-math* :warn-on-boxed)
% (ns ^{:doc "compute a file checksum."}
%   
%   curate.scripts.checksum
%   
%   (:require [clojure.java.io :as io])
%   (:use [clojure.set :only [difference]])
%   (:gen-class))
% ;;-----------------------------------------------------------------
% (defn checksum [file]
%   (let [input (java.io.FileInputStream. file)
%         digest (java.security.MessageDigest/getInstance "MD5")
%         stream (java.security.DigestInputStream. input digest)
%         bufsize (* 1024 1024)
%         buf (byte-array bufsize)]
% 
%   (while (not= -1 (.read stream buf 0 bufsize)))
%   (apply str (map (partial format "%02x") (.digest digest)))))
% 
% (defn list-dir [dir]
%   (remove #(.isDirectory %)
%           (file-seq (java.io.File. dir))))
% 
% (defn find-dupes [root]
%   (let [files (list-dir root)]
%     (let [summed (zipmap (pmap #(checksum %) files) files)]
%       (difference
%        (into #{} files)
%        (into #{} (vals summed))))))
% 
% (defn remove-dupes [files]
%   (prn "Duplicates files to be removed:")
%   (doseq [f files] (prn (.toString f)))
%   (prn "Delete files? [y/n]:")
%   (if-let [choice (= (read-line) "y")]
%     (doseq [f files] (.delete f))))
% 
% (defn -main [& args]
%   (if (empty? args)
%     (println "Enter a root directory")
%     (remove-dupes (find-dupes (first args))))
%   (System/exit 0))
% \end{lstlisting}
% 
% \lstinputlisting[
% float=htbp,
% caption={[checksum function]Compute a checksum for a file thru Java},
% label=checksum,
% firstline=11,lastline=18]{listings/checksum.clj}
 
%-----------------------------------------------------------------
\setcounter{currentlevel}{\value{baseSectionLevel}-1}
\levelstay{Examples}
 
% \begin{minipage}{\linewidth}
% \begin{lstlisting}[
% caption={[FileTypeFilter]A file .suffix filter},
% label=FileTypeFilter]
% package img;
% 
% public final class FileTypeFilter 
%   implements java.io.FilenameFilter {
% 
%   private final String _suffix;
% 
%   public final boolean accept (final java.io.File dir,
%                                final String name) {
%     return name.toLowerCase().endsWith(_suffix); }
% 
%   public FileTypeFilter (final String suffix) {
%     super();
%     _suffix = suffix.toLowerCase(); } }
% \end{lstlisting}
% \end{minipage}
% 
% \index{Sets|)}
% 
% \begin{figure}[htbp]
% \centering
% \includegraphics[scale=0.7]{figs/arara.png}
% \caption{Configuring {\TeX}works for \texttt{arara}.}
% \end{figure}

%-----------------------------------------------------------------
\setcounter{currentlevel}{\value{baseSectionLevel}}
%-----------------------------------------------------------------
\setcounter{currentlevel}{\value{baseSectionLevel}}
\levelstay{Algebraic structures}
\label{sec:Algebraic-structures}

\textbf{TODO:} redo as Universal Algebras?
Implies more operations, typically binary, unary, and nullary
(nullary function equivalent to element of set).

An algebraic 
structure~\cite{
wiki:Algebraic-structure,
wiki:Mathematical-structure,
wiki:Outline-of-algebraic-structures,
BWilliams:Algebraic-structure-and-protocols,
BWilliams:Algebra-of-predicates-and-sorting-functions,
BWilliams:Semirings-and-predicates}
consists of:
\begin{itemize}
  \item A primary set --- the \textit{elements} of the structure.
  \item zero or a few auxilliary sets.
  \item functions (called \textit{operations})
that take a small number of arguments from one or more of the sets
and return elements of the sets.
\end{itemize}

The type of an algebraic structure corresponds to identities
that the operations satisfy.
(\textbf{TODO:} Examples of operation identities?)

Unfortunately, the names for algebraic structures 
are, as a rule, not very informative.

%-----------------------------------------------------------------
\setcounter{currentlevel}{\value{baseSectionLevel}-1}
\levelstay{One set, one operation}
%-----------------------------------------------------------------
\setcounter{currentlevel}{\value{baseSectionLevel}-2}
\levelstay{Monoid}
\label{sec:Monoid}

Set $\Set{S}$ and operation $\diamond$ such that,
if $a, b, c \in \Set{S}$, then
\begin{description}
\item[Closed] $a \diamond b = \diamond \left( a, b \right) 
= \left( \diamond \; a \; b \right) \in \Set{S}$
\item[Associative] $a \diamond b \diamond c =
 \left( a \diamond b \right) \diamond c =  
 a \diamond \left( b \diamond c \right) $
 \item[Identity] There is an $i \in \Set{S}$ such that 
 $i \diamond a = a \; \forall a \in \Set{S}$.
 Exercise: show that $i$ is unique.
 \end{description}

Example: 

$\Set{S}$ the functions from some domain $\Set{X}$ to itself. 
Operation $\diamond$ is function composition:
$\left( f \diamond g \right) (x) = 
f \left( g \left( x \right) \right)$.


%-----------------------------------------------------------------
\setcounter{currentlevel}{\value{baseSectionLevel}-2}
\levelstay{Group}

Monoid $\left[ \Set{S}, \diamond \right]$ such that
\begin{description}
 \item[Inverse] For every $a \in \Set{S}$ there exists an
 $a^{-1}$ such that $a^{-1} \diamond a = i$.
 Exercise: show that this implies that $a \diamond a^{-1} = i$ .
\end{description}

%-----------------------------------------------------------------
\setcounter{currentlevel}{\value{baseSectionLevel}-2}
\levelstay{Commutative group}

(aka abelian group.)

A group $\left[ \Set{S}, \diamond \right]$ where
\begin{description}
 \item[Commutative] $a \diamond b = b \diamond a \; \forall a,b \in \Set{S}$.
\end{description}

Example: 
$\left[ \glssymbol{Integers}, *_{\glssymbol{Integers}} \right]$

%-----------------------------------------------------------------
\setcounter{currentlevel}{\value{baseSectionLevel}-1}
\levelstay{One set, two operations}
%-----------------------------------------------------------------
\setcounter{currentlevel}{\value{baseSectionLevel}-2}
\levelstay{Semiring}
\label{sec:Semiring}

A set and $2$ operations: $\left[ \Set{S}, +, * \right]$
where
\begin{description}
  \item[Addition] $+$ is commutative, associative, and has an 
  identity element ($0$):
  $a + b = b + a$; 
  $a + \left( b + c \right) =\left( a + b  \right) + c $;
   and $ 0 + a = a$, for all $a,b,c \in \Set{S}$.
  \item[Multiplication] $*$ is associative, and has an 
  identity element ($1$):
 $a * \left( b * c \right) =\left( a * b \right) * c $
  and $ 1 * a = a$, for all $a,b,c \in \Set{S}$.]
  \item[Distributive] $a * \left( b + c \right) 
  = \left( a * b \right) + \left( a * c \right)$
\end{description}

Example: 

The \textit{\gls{NaturalNumbers}}: 
$\glssymbol{NaturalNumbers} = \glsdesc{NaturalNumbers}$.
(sec~\ref{sec:Natural-numbers}).

%-----------------------------------------------------------------
\setcounter{currentlevel}{\value{baseSectionLevel}-2}
\levelstay{Ring}
\label{sec:Ring}
\cite{wiki:Ring-mathematics}

A set and $2$ operations: $\left[ \Set{S}, +, * \right]$
where
\begin{description}
  \item[Addition group] $\left[ \Set{S}, + \right]$ 
  is a commutative group (with the identity written $0$).
  \item[Multiplication monoid] $\left[ \Set{S}, * \right]$ 
  is a monoid (with the identity written $1$).
  Note this means $*$ may not commute. If it does, 
  then this is a \textit{commutative ring}.
  \item[Distributive] $a * \left( b + c \right) 
  = \left( a * b \right) + \left( a * c \right)$
\end{description}

Example:

The \textit{\gls{Integers}}: 
$\glssymbol{Integers} = \{\cdots, -2, -1, 0, 1, 2, \cdots\}$.

%-----------------------------------------------------------------
\setcounter{currentlevel}{\value{baseSectionLevel}-2}
\levelstay{Field}
\label{sec:Field}
\cite{wiki:Field-mathematics}

A ring $\left[ \Set{S}, +, * \right]$ where
$*$ is commutative and the nonzero elements of $\Set{S}$
have multiplicative inverses.

Without refering to other structures:
A \textit{field} is set and 2 operations: 
$\left[ \Set{S}, +, * \right]$
where
\begin{description}
  \item[Additive closure] $a + b \in \Set{S} \; 
  \forall \, a,b \in \Set{S}$ 
  \item[Additive associativity] 
  $\left( a + b \right) + c = a + \left( b + c \right)$ 
  \item[Additive commutativity] $a + b = b + a$ 
  \item[Additive identity] $\exists \, 0 \in \Set{S} \text{ s.t. } 
  a + 0 = 0 + a = a \; \forall \, a \in \Set{S}$ 
  \item[Additive inverse] $\forall a \in \Set{S} \; 
  \exists \, {-a} \in \Set{S} 
  \text{ s.t. }  a + \left( -a \right) = \left( -a \right) + a = 0$ 
  \item[Multiplicative closure] $a * b \in \Set{S} \; \forall \,a,b \in \Set{S}$ 
  \item[Multiplicative associativity] 
  $\left( a * b \right) * c = a * \left( b * c \right)$ 
  \item[Multiplicative commutativity] $a * b = b * a$ 
  \item[Multiplicative identity] $\exists 1 \in \Set{S}
   \text{ s.t. } 
  a * 1 = 1 * a = a \; \forall \, a \in \Set{S}$ 
  \item[Multiplicative inverse] $\forall a \ne 0 \in \Set{S} 
  \exists a^{-1} \in \Set{S}
  \text{ s.t. }  a * a^{-1} = a^{-1} * a = 1$ 
  (Note the restriction to elements other than the additive identity.) 
  \item[Distributive] $a * \left( b + c \right) 
  = \left( a * b \right) + \left( a * c \right)$
\end{description}


Examples: 
\glssymbol{RationalNumbers}, \glssymbol{RealNumbers}.

%-----------------------------------------------------------------
\setcounter{currentlevel}{\value{baseSectionLevel}-1}
\levelstay{Two sets, one operation}
%-----------------------------------------------------------------
\setcounter{currentlevel}{\value{baseSectionLevel}-2}
\levelstay{Category}

A base set, $\Set{O}$, 
whose elements are ususally refered to as \textit{objects}.
A second set, $\Set{M}$, of \textit{morphisms}, 
each of which depends on an ordered pair 
$\left( \text{source} , \text{target} \right)$ of elements of 
$\Set{O}$. 
A \textit{composition} operation, $\circ$, on 
the subset of the pairs of morphisms $f , g \in \Set{M}$
where $\text{source}(f) = \text{target}(g)$, satisfying:
\begin{description}
\item[Associativity:] $f \circ g \circ g \defeq 
\left( f \circ g \right) \circ h 
= f \circ \left( g \circ h \right)$
\item[Identity:] For each $x \in \Set{O}$, there exists an element
$1_x \in \Set{M}$ such that 
$x = \text{source}(1_x) = \text{target}(1_x)$, and 
$1_x \circ f = f$ (when $x = \text{target(f)}$) and
$f \circ 1_x = f$ (when $x = \text{source(f)}$).
\end{description}

 \begin{example}[The Set category]
 $\Set{O} = $ some collection of sets;
 $\Set{M} = $ the functions between those sets;
 $\circ = $ function composition.  
 \end{example}
%-----------------------------------------------------------------
 
%-----------------------------------------------------------------
\setcounter{currentlevel}{\value{baseSectionLevel}}
%-----------------------------------------------------------------
\setcounter{currentlevel}{\value{baseSectionLevel}}
\levelstay{Numbers}
\label{sec:Numbers}

% In this chapter, I'm listing a number of things I'm taken as \emph{given,} in
% the sense that I'm assuming we both know what I mean, without further
% explanation. 
% If you doubt the correctness of that assumption, I'll either provide a
% reference, or you can take the corresponding Wikipedia entry as good enough.
% 
% The main purpose for listing these things is to make it clear what I'm
% not going to explain, and, secondarily, to establish notation, though my intent
% is to stick to the most widely used notation, except when I given a reason not
% to.


%-----------------------------------------------------------------
\lstset{language=Clojure}

%-----------------------------------------------------------------
\setcounter{currentlevel}{\value{baseSectionLevel}-1}
\levelstay{Natural numbers}
\label{sec:Natural-numbers}

The \textit{\gls{NaturalNumbers}} (aka non-negative integers, aka counting
numbers):\label{NaturalNumbers}
$\glssymbol{NaturalNumbers} = \glsdesc{NaturalNumbers}$.\footnote{ 
Some
equate \gls{NaturalNumbers} with \glssymbol{PositiveIntegers} 
or $\glssymbol{NaturalNumbers}^{+}$ and use
$\glssymbol{NaturalNumbers}^{0}$ for the non-negative integers. 
See~\autoref{sec:math-sets} if the set specification
notation is new to you.}

The \textit{strictly positive integers}: 
$\glssymbol{PositiveIntegers} = \glsdesc{PositiveIntegers}$, which 
may also be written $\glssymbol{NaturalNumbers}_{+}$.

Origin: counting things.

Addition $+_{\glssymbol{NaturalNumbers}}$ 
derived from counting problem.
Associative, commutative, identity element $0$ $\Rightarrow$ 
commutative monoid (sec~\ref{sec:Monoid}).

Multiplication $*_{\glssymbol{NaturalNumbers}}$
derived from addition problem (?).
Associative, commutative, identity element $1$,
additive identity $0$ annihilates,
$*_{\glssymbol{NaturalNumbers}}$ distributes over 
$+_{\glssymbol{NaturalNumbers}}$ 
 $\Rightarrow$
commutative semiring 
(sec~\ref{sec:Semiring}).


%-----------------------------------------------------------------
\setcounter{currentlevel}{\value{baseSectionLevel}-1}
\levelstay{Cyclic natural numbers}

${\glssymbol{NaturalNumbers}}_{k} = 
\left[ \left\{ 0, 1, \ldots, k-1 \right\},
+_{\glssymbol{NaturalNumbers}_{k}},
*_{\glssymbol{NaturalNumbers}_{k}} \right]$

where
$a +_{\glssymbol{NaturalNumbers}_{k}} b =
\left( a +_{\glssymbol{NaturalNumbers}} b \right) \text{mod} k$

Also a commutative semiring.

%-----------------------------------------------------------------
\setcounter{currentlevel}{\value{baseSectionLevel}-1}
\levelstay{\texttt{unsigned int}}
\label{sec:unsigned-int}

Most C family languages provide \texttt{unsigned} integers
of various lengths, typically: 
\texttt{unsigned short} $\in \left[ 0, 2^{16} - 1 \right]$,
\texttt{unsigned int} $\in \left[ 0, 2^{32} - 1 \right]$,,
\texttt{unsigned long} $\in \left[ 0, 2^{64} - 1 \right]$,.

Implementations of $\glssymbol{NaturalNumbers}_{k}$,
for $k = 2^{16}, 2^{32}, 2^{64}$.

%-----------------------------------------------------------------
\setcounter{currentlevel}{\value{baseSectionLevel}-1}
\levelstay{Integers}
\label{sec:Integers}

Given $a,b,c \in \glssymbol{NaturalNumbers}$,
solve $a = b + c$ for $c$.

The \textit{\gls{Integers}}: 

$\glssymbol{Integers} = \{\cdots, -2, -1, 0, 1, 2, \cdots\}$.

A ring (sec~\ref{sec:Ring}).

%-----------------------------------------------------------------
\setcounter{currentlevel}{\value{baseSectionLevel}-1}
\levelstay{Cyclic integers}
\label{sec:Cyclic-integers}

$\Space{Z}_{k} = \left[ \Set{Z}_{k}, +_k, *_k, 0, 1 \right]$
is a ring (?):
\begin{itemize}
  \item $\Set{Z}_{k} = \left\{ 0. \ldots k-1  \right\}$
  \item $ a +_k b = \left( a + b \right) \text{mod} k$
  \item $ a *_k b = \left( a * b \right) \text{mod} k$
\end{itemize}

%-----------------------------------------------------------------
\setcounter{currentlevel}{\value{baseSectionLevel}-1}
\levelstay{\texttt{byte}, \texttt{short}, \texttt{int}, \texttt{long}}
\label{sec:int}

Clojure favors \texttt{long} --- why?

%-----------------------------------------------------------------
\setcounter{currentlevel}{\value{baseSectionLevel}-1}
\levelstay{\texttt{BigInteger}}
\label{sec:BigInteger}

%-----------------------------------------------------------------
\setcounter{currentlevel}{\value{baseSectionLevel}-1}
\levelstay{Rational numbers}
\label{sec:Rational-numbers}

The \textit{\gls{RationalNumbers}}: 
$\glssymbol{RationalNumbers} = \glsdesc{RationalNumbers}$.

%-----------------------------------------------------------------
\setcounter{currentlevel}{\value{baseSectionLevel}-1}
\levelstay{\texttt{BigFraction}}
\label{sec:BigFraction}

Performance comparisons.

%-----------------------------------------------------------------
\setcounter{currentlevel}{\value{baseSectionLevel}-1}
\levelstay{Real numbers}
\label{sec:Real-numbers}

The \textit{\gls{RealNumbers}}, \glssymbol{RealNumbers},
can be defined in a number of ways. 
Qualitatively, \glssymbol{RealNumbers} is the completion of
\glssymbol{RationalNumbers}, in the sense it provides a limiting value for every
convergent sequence in \glssymbol{RationalNumbers}.

What algebraic structure if extended with $\pm\infty$?
'Not even a semigroup' wikipedia.
$\infty - \infty$, etc. \textit{undefined} values $\Rightarrow$
\texttt{NaN}.
Something like a field with an extra $\text{undefined}$
element --- operations obey field constraints unless value is
$\text{undefined}$.

Can this be fixed by changing $=$ to handle $\text{undefined}$
appropriately?

%-----------------------------------------------------------------
\setcounter{currentlevel}{\value{baseSectionLevel}-1}
\levelstay{IEEE 754 floating point}

What algebraic structure?

Not associative.

\texttt{NaN} absorbing in arithmetic, not self equal.

Similar to extended real numbers, but also distinguish $\pm 0$.

%-----------------------------------------------------------------
\setcounter{currentlevel}{\value{baseSectionLevel}-1}
\levelstay{\texttt{float}, \texttt[double]}

%-----------------------------------------------------------------
\setcounter{currentlevel}{\value{baseSectionLevel}-1}
\levelstay{\texttt{DoubleDouble}}

%-----------------------------------------------------------------
\setcounter{currentlevel}{\value{baseSectionLevel}-1}
\levelstay{Extending floating point to be a field}

Fix associativity by carrying accumulator bank,
as in exact sum.

But \texttt{NaN}: $0$ doesn't annihilate,
no additive or multiplicative inverse.

Change behavior of $0 / 0$, etc.?

%-----------------------------------------------------------------
\setcounter{currentlevel}{\value{baseSectionLevel}-1}
\levelstay{Floating point numbers as an algebraic structure}

Two non-associative commutative operations on a finite,
almost ordered set.
\begin{example}[Floating point arithmetic is not associative]

clojure/java code showing how different results can be\ldots
 
\end{example}

%-----------------------------------------------------------------
\setcounter{currentlevel}{\value{baseSectionLevel}-1}
\levelstay{'Exact' floating point arithmetic}
\cite{Higham2002ASNA,Muller-et-al-2010,Zhu:2010:A9O:1824801.1824815}

\textbf{TODO:} do TwoPlus and TwoMutliply 
suggest any way to get associative operations?
Idea is to carry error accumulators so 
any sequence of additions and multiplies 
returns the correct \glssymbol{RealNumbers} value rounded to float.
EG: $ a +_{\text{fl}} b = c +_{\text{fl}} \epsilon$
where $\text{fl} c +_{\glssymbol{RealNumbers}} \epsilon) = c$
Absorbing elements at $\pm\infty$, \texttt{NaN} make that hard.
Is it possible to replace $\pm\infty$ and \texttt{NaN}
by symbolic expressions that can later be combined to get 
accurate finite values?


\textbf{TODO:} is it possible for a clojure macro/compiler
to support exact floating operations generated from
$+$ and $*$ within a function, by incorporating an error
accumulator?

%-----------------------------------------------------------------
\setcounter{currentlevel}{\value{baseSectionLevel}-1}
\levelstay{Others}
(Other number sets:
$\text{Constructible numbers} = \glssymbol{RationalNumbers}
+ \text{square roots}$

$\glssymbol{RationalNumbers}
\subset 
\text{Constructible Numbers}
\subset 
\glssymbol{RealNumbers}$.

Transcendental, Algebraic, \ldots.)

It should not be a surprise that:
$\glssymbol{NaturalNumbers} 
\subset 
\glssymbol{Integers}
\subset 
\glssymbol{RationalNumbers}
\subset 
\glssymbol{RealNumbers}$.
However, note that here we are silently assuming every integer
\emph{is} a real number, as opposed to considering \glssymbol{Integers} and
\glssymbol{RealNumbers} as distinct spaces, with the natural coercion
from \glssymbol{Integers} to a subset of \glssymbol{RealNumbers},
which is closer to how they are implemented.

Any of these ``number space'' symbols, \glssymbol{GenericSpace},
can be modified with sub-scripts to indicate restrictions or
extensions. 
(Using super-scripts here is more common --- I choose sub-scripts
to avoid confusion in things like $\Space{R}^{\infty}$,
which might be $\Space{R}$ extended with ${\infty}$, 
or might an infinite dimensional real vector space.)

For example, the spaces can be 
augmented by adding one or more values at infinity, which I will
write as $\glssymbol{GenericSpace}_{\infty}$,
or $\glssymbol{GenericSpace}_{\pm\infty}$ when I want to be 
clear that there are separate values for $-\infty$ and $+\infty$.

Or $\glssymbol{GenericSpace}_{+}$ ($\glssymbol{GenericSpace}_{-}$)
for the positive (negative) values, with 
$\glssymbol{GenericSpace}_{> 0}$, $\glssymbol{GenericSpace}_{\geq 0}$ 
when I feel the need to be clear about, eg, non-negative versus
strictly positive.

%-----------------------------------------------------------------
\setcounter{currentlevel}{\value{baseSectionLevel}-1}
\levelstay{Generating spaces by 'closure'}

Generating these 'spaces' by closure under arithmetic operations
\cite{pickert-gorke-real-numbers-1974}.

A common pattern: a problem defined in terms of an existing mathematical
structure. Some combination of convenience, simplicity, computational
efficiency, and/or simplicity leads us to define a new structure, 
often a larger enclosing one.

We can use the number 'spaces' as an example.
Imagine, for the moment, that you only know the 
the natural numbers, $\glssymbol{NaturalNumbers}$, with a single operation: $+$.
Suppose we have a problem to solve:
\begin{math}
a + b = c
\end{math}
where $a,b,c\in \glssymbol{NaturalNumbers}$ and we know $b$ and $c$,
but not $a$.

%\begin{minipage}{\linewidth}
\begin{lstlisting}[
caption={[Re-inventing subtraction]Reinventing subtraction},
captionpos=b,
label=reinvent-subtraction,
mathescape=true,
%escapeinside={<*}{*>}
] 
(when (<= $b$ $c$)
  (loop [$a$ 0]
    (cond 
      (= $c$ (+ $a$ $b$)) $a$
      (< $c$ (+ $a$ $b$)) :no-solution
      :else (recur (+ 1 $a$)))))
\end{lstlisting}
%\end{minipage}

Rationals: 
\begin{math}
a * b = c
\end{math}

Reals: closure under sequence limit.
Suppose we have $x_0, x_1, \ldots \in \glssymbol{RealNumbers}$
where, for any $\epsilon>0$ there exists an $n$ such that
$|x_i - x_j| < \epsilon$ for any $i,, j > n$.

Insert example sequence that converges to $\sqrt{2}$ or $\pi$
with quick arg or ref to why it converges, but not to a rational
number.

%-----------------------------------------------------------------
\setcounter{currentlevel}{\value{baseSectionLevel}-1}
\levelstay{Intervals}

Open and closed intervals, half open, etc.:
$[a,b] \subset \glssymbol{GenericSpace} = 
\SetSpec{s \glssymbol{elementOf} \glssymbol{GenericSpace}}
{a \leq s \leq b}$;
$(a,b) \subset \glssymbol{GenericSpace} = 
\SetSpec{s \glssymbol{elementOf} \glssymbol{GenericSpace}}
{a < s < b}$;
$[a,b) \subset \glssymbol{GenericSpace} = 
\SetSpec{s \glssymbol{elementOf} \glssymbol{GenericSpace}}
{a \leq s < b}$;
and so on.

%-----------------------------------------------------------------
\setcounter{currentlevel}{\value{baseSectionLevel}-1}
\levelstay{Java}
\lstset{language=Java}

%-----------------------------------------------------------------
%-----------------------------------------------------------------
\setcounter{currentlevel}{\value{baseSectionLevel}-2}
\levelstay{Primitives}

%-----------------------------------------------------------------
\setcounter{currentlevel}{\value{baseSectionLevel}-2}
\levelstay{Integers}
Finite subsets of \glssymbol{Integers} are exactly
represented by the primitive integer types using two's complement arithmetic, 
covering $[−2^{n−1}, 2^{n−1} − 1]$ where $n$ is the number of bits used:
\begin{description}
\item[\ttfamily {byte}] $8$ bits.
\item[\texttt{short}] $16$ bits.
\item[\texttt{int}] $32$ bits.
\item[\texttt{long}] $64$ bits.
\end{description}

Elements of \glssymbol{RealNumbers} are approximated using floating
point:
\begin{description}
\item[\texttt{float}] $32$ bits; range of finite values:
$\pm 3.40282347 x 10^{38}$, with special values for $\pm\infty$;
smallest non-zero value $1.40239846 x 10^{-45}$.
\item[\texttt{double}] $64$ bits;
range of finite values: 
$\pm 1.7976931348623157 x 10^{308}$, with special values for $\pm\infty$;
smallest non-zero value $4.9406564584124654 x 10^{-324}$.
\end{description}

%-----------------------------------------------------------------
\setcounter{currentlevel}{\value{baseSectionLevel}-2}
\levelstay{Objects}

\begin{figure}[htbp]
\centering
%\includegraphics[scale=0.5]{figs/arara.png}
\caption{Java \texttt{Number} classes.}
\label{fig:java-number-classes}
\end{figure}

Boxed vs primitive arithmetic benchmark; 
int vs long vs float vs double vs Integer \ldots vs BigDecimal.


%-----------------------------------------------------------------
\setcounter{currentlevel}{\value{baseSectionLevel}-1}
\levelstay{Clojure}
\lstset{language=Clojure}

Clojure provides the primitive and boxed object numbers from Java,
except:
\begin{itemize}
  \item Full support for primitive type hints (eg for function arguments and
  return values) is only available for \lstinline|long| and \lstinline|double|.
  \item Idiomatic Clojure obscures whether primitive or boxed values will be
  used in any particular chunk of code (even more than recent versions  Java)
  Because of this, it is good practice to begin every namespace with
\begin{lstlisting}[
caption={[Boxed arithmetic warnings]}, label=unchecked-math,]  
(set! *unchecked-math* :warn-on-boxed)
\end{lstlisting}
which will generate compile-time warnings
\end{itemize}

Clojure adds rational numbers, which turn out to rarely be useful.\\
\lstinline|clojure.lang.Ratio| rational numbers
\lstinline|clojure.lang.BigInt|~\cite[p.~428]{Emerick2012ClojureProgramming}

%-----------------------------------------------------------------
\setcounter{currentlevel}{\value{baseSectionLevel}}
%-----------------------------------------------------------------
\levelstay{Polynomials}
\label{ch:Polynomials}

\cite{wiki:Polynomial,wiki:FactorizationOfPolynomials}
%-----------------------------------------------------------------
%\setcounter{currentlevel}{\value{baseSectionLevel}}
%\levelstay{Simplexes}
%-----------------------------------------------------------------
%\leveldown{Orientation}
%-----------------------------------------------------------------
%\levelstay{Implementation}
%-----------------------------------------------------------------
%\levelstay{examples}
%-----------------------------------------------------------------
\setcounter{currentlevel}{\value{baseSectionLevel}}
%-----------------------------------------------------------------
\levelstay{Spaces}
%-----------------------------------------------------------------
\leveldown{Topological spaces}
\label{sec:Topological-spaces}

A generalization of
\cite[chapter~1]{spivak-1965}.

\textit{Open sets}, \textit{neighborhoods}.

Finite intersection of open sets is open.
Arbitrary union is is open. 

\textit{Interior}: elements in set 
with a neighborhood contained
in the set.

\textit{Exterior}: elements not in set
with a neighborhood not intersecting the
in the set.

\textit{Boundary}: elements where every neighborhood
intersects both the interior and the exterior.

\textit{Open cover}. 

\textit{Compact} set has finite open cover.

A set is \textit{closed} if its complement is open.

Connectivity: set is connected if no disjoint open sets contain
whole set.

Limits.

\textit{Continuity} of functions: 
(See \cite[Theorem~1-8]{spivak-1965}.)
$f \, : \, \Set{D} \mapsto \Set{C}$,
for topological spaces $\Set{D}$ and $\Set{C}$,
is \textit{continuous}
if for any open set $\Set{O}_{\Set{C}} \subset \Set{C}$,
$f^{-1}\left( \Set{O}_{\Set{C}} \right) = 
\Set{O}_{\Set{D}}$, an open set $\subset \Set{D}$.
(\textbf{TODO:} Is this assuming domain and codomain are open?
Is this generalization of Spivak 1.8 really correct?)

Example: open intervals in $\Space{R}$,
open balls in $\Space{R}^n$ with $l_1$, $l_2, l_{\infty}$ distances
(what is required to work?).
Figures for open balls with various metrics.

%-----------------------------------------------------------------
\levelstay{Metric spaces}
\label{sec:Metric-spaces}

Open sets generated by distance function.

%-----------------------------------------------------------------
\levelstay{Linear spaces}
\label{sec:Linear-spaces}

\epigraph{Throughout these courses the infusion of a geometrical
point of view is of paramount importance. A vector
is geometrical; it is an element of a vector space, defined
by suitable axioms—whether the scalars be real numbers or
elements of a general field. A vector is not an n-tuple of
numbers until a coordinate system has been chosen. Any
teacher and any text book which starts with the idea that vectors
are n-tuples is committing a crime for which the proper
punishment is ridicule. The n-tuple idea is not ‘easier,’ it is
harder; it is not clearer, it is more misleading. By the same
token, linear transformations are basic and matrices are their
representations\ldots}
{MacLane, \textit{Of course and courses}\cite{MacLane:1954}}

My approach to linear (aka vector) spaces is largely based on
the texts I used as a college freshman for linear algebra and
multivariate calculus: Halmos \cite{halmos-1958}
and Spivak \cite{spivak-1965}.

\begin{definition}[Linear space]
\bigskip
A \textit{linear space} 
$\Space{V} = \left[ \Set{V}, \Space{K}, \text{linear-combination} \right]$
 is:
\begin{itemize}
  \item a set of \textit{vectors} $\Set{V}$,
  \item a field  of scalars $\Space{K}$,
  \item a linear combination function: 
\begin{equation}
\left( \text{linear-combination} 
\, a_0 \, \Vector{v}_0 \, a_1 \, \Vector{v}_1 \right) \; 
 \rightarrow \; \Vector{v}_2  \in \Set{V}
\end{equation}
for $\Vector{v}_0, \Vector{v}_1 \in \Set{V} $
and $a_0, a_1 \in \Space{K}$.
Linear combination is often defined in terms of
$2$ binary operations:
scalar multiplication $a * \Vector{v} \in \Set{V}$,
and vector addition $\Vector{v}_0 + \Vector{v}_1 \in \Set{V}$:
\begin{equation}
\left( \text{linear-combination} 
\, a_0 \, \Vector{v}_0 \, a_1 \, \Vector{v}_1 \right) \; 
= \; a_0*\Vector{v}_0 + a_1*\Vector{v}_1
\end{equation}
\textbf{TODO:} required identities for $+$ and $*$ from Spivak or Halmos.
\end{itemize}
\end{definition}
Usually the distinction between $\Set{V}$ and $\Space{V}$ 
is ignored, and we will say, for example, 
$\Vector{v} \in \Space{V}$.

Examples:

\begin{example}[$\Space{K}^n$]
%{}
%\bigskip
Where $\Space{K}$ is any field.
%\leavevmode \vspace{-\baselineskip}
\begin{itemize}
  \item vectors:
  $\Set{V} = \Space{K}^n = \{ \Vector{x}
  = \left[ x_0, \ldots , x_{n-1} \right] \}$,
  tuples of $n$ elements $x_i \in \Space{K}$.
  \item scalars: $\Space{K}$
  \item scalar multiplication:
  $ a *_{\Space{K}^n} \left[ x_0, \ldots , x_{n-1} \right] =
  \left[ \ldots , a *_{\Space{K}} x_{i} , \ldots \right]$
  \item vector addition:
  $\Vector{x} +_{\Space{K}^n} \Vector{y}
  = \left[ \ldots , \left( x_i +_{\Space{K}} y_i \right) ,\ldots \right]$
\end{itemize}
\end{example}

\begin{example}[$\Space{R}^n$]
\vspace{\topsep}
\setlength{\parskip}{5pt}
\setlength{\parindent}{0pt}
See \cite[chapter~1]{spivak-1965} and \cite[chapter~1]{halmos-1958}.
%\leavevmode \vspace{-\baselineskip}
\begin{itemize}
  \item vectors:
  $\Set{V} = \Space{R}^n = \{ \Vector{x}
  = \left[ x_0, \ldots , x_{n-1} \right] \}$,
  tuples of $n$ elements $x_i \in \Space{R}$.
  \item scalars: $\Space{R}$
  \item scalar multiplication:
  $ a*\left[ x_0, \ldots , x_{n-1} \right] =
  \left[ \ldots , \left( a*x_i \right) , \ldots \right]$
  \item vector addition:
  $\Vector{x} + \Vector{y}
  = \left[ \ldots , \left( x_i + y_i \right) , \ldots \right]$
\end{itemize}

\textbf{TODO: DANGER:} Apple-orange mistakes resulting from
using $\Space{R}^n$ in problems where coordinates don't mean the
same thing, so canonical inner product and $l2$ distance
aren't correct.

Homogeneous problems often don't have meaningful coordinates.

Can only approximate $\Space{R}^n$ with tuples of \texttt{double},
which is fundamentally different due to lack of associativity,
which leads to accumulation of rounding error.

\end{example}

\textbf{TODO:} Exact float arithmetic as an alternative?

\begin{example}[$\Space{Q}^n$]
%{}\bigskip
Like $\Space{R}^n$, only over rational rather than real numbers.
Has the advantage that it can be implemented accurately
using arbitrary precision fractions, though at considerable
space-time cost.

\textbf{TODO:} measure cost compared to \texttt{double}
approximation to $\Space{R}^n$
\end{example}


\begin{example}[{ $ \Space{F} \left[ \Set{D}, \Space{V} \right] $ }]
The functions from any domain to some linear space.
\textbf{TODO:} lisp notation for clarity below.
\begin{itemize}
  \item vectors: $\Set{F} = $ any function on $\Set{D}$
  that returns values in the linear space $\Space{V}$.
  \item scalars: $\Space{K}$, the same scalar field used by
  $\Space{V}=\left[ \Set{V}, \Space{K}, +, * \right]$.
  \item scalar multiplication:
  $ \left(a*\Vector{f}\right) : \Set{D} \rightarrow \Space{V}$
  is the function defined by
   $ \left(a*\Vector{f}\right) (x)
   = a*\left(\Vector{f}\left( x \right) \right) $
  \item vector addition:
  $\left( \Vector{f} + \Vector{g} \right) $
  is the function defined by
  $\left( \Vector{f} + \Vector{g} \right) \left( x \right) =
  \Vector{f} \left( x \right) + \Vector{g} \left( x\right)$
\end{itemize}
Canonical coordinates: 
$\Vector{f}\left( d \right) \; \forall d \in \Set{D}$
\end{example}

\begin{example}[{ $\Space{R}^n$ as a function space }]
%{}
%\bigskip
We can identify~\cite[sec.~24]{Halmos1958Finite} 
$\Space{R}^n$ and 
$ \Space{F} \left[ \{ 0, 1, \ldots, {n-1} \}, \Space{R} \right] $
by
$\Vector{x} \Leftrightarrow f_{\Vector{x}}$
where $f_{\Vector{x}} \left( i \right) = \Vector{x}_i$
for $\Vector{x} \in \Space{R}^n$ and 
$i \in \{ 0, 1, \ldots, {n-1} \}$
\end{example}

%-----------------------------------------------------------------
\begin{definition}[Linear dependence]
Suppose $\Space{V}$ is a linear space and
$\Vector{v}_0, \ldots , \Vector{v}_{n-1} \in \Space{V}$.
If there exists $a_0, \ldots , a_{n-1}$ such that
$\Vector{0} = \sum a_i \Vector{v}_i$ then the $\{ \Vector{v}_i \}$
are \textit{linearly dependent}.
\cite[section~5]{halmos-1958}

Otherwise they are \textit{linearly independent}.
\end{definition}
%-----------------------------------------------------------------
\levelstay{Bases}
%-----------------------------------------------------------------
\levelstay{Dimension}
%-----------------------------------------------------------------
\levelstay{Subspaces}

A subset which is also a linear space with the
same scalars and operations.


\begin{example}[Canonical subspaces of a function space]
\bigskip
Let $\Set{D}_0 \subset \Set{D}_1$ be a proper subset.
Consider 
$\Space{F}_0 = \Space{F} \left[ \Set{D}_0, \Space{V} \right] \}$
and
$\Space{F}_1 = \Space{F} \left[ \Set{D}_1, \Space{V} \right] \}$.
\end{example}

\begin{example}[Canonical subspaces of $\Space{R}^n$]
\bigskip
Finite index set of integers, a real value for each.
Relationships between index sets define super/sub space relations
intersections.
\end{example}
%-----------------------------------------------------------------
\levelstay{Normed linear spaces}
%-----------------------------------------------------------------
\levelstay{Inner product (linear) spaces}
Let $\Space{V}$ be an $n$-dimensional real inner product space.
Let $\v, \w \in \Space{V}$.

\begin{itemize}
\item The inner (dot) product on $\Reals^n$:
\begin{equation}
\v \bullet \w \; \equiv \; \sum_{i=0}^{n-1} v_i w_i
\end{equation}

\item The euclidean ($l_2$) norm:
\begin{equation}
\| \v \|^2 \; \equiv \; \v \bullet \v
\end{equation}

\item $\theta(\v,\w)$ is the angle between $\v$ and $\w$
and is defined by:
\begin{eqnarray}
\v \bullet \w \; = \; \| \v \| \| \w \| \cos(\theta(\v,\w))
\\
\theta(\v,\w)
\; \equiv \;
\cos^{-1} \left(\frac{ \v \bullet \w }{\| \v \| \| \w \| } \right)
\nonumber
\end{eqnarray}

\item The tensor (outer) product:

Let $\v, \u \in \Space{V}, \w \in \Space{W}.$
$\w \otimes \v$ is a rank 1 linear map
from $\Space{V}$ to $\Space{W}$, defined by:
\begin{equation}
(\w \otimes \v)(\u) \; \equiv \; \w (\v \bullet \u)
\end{equation}

Note: this is an abuse of the usual definition of tensor product $\otimes$.
This operation, which takes a pair of vectors and returns a linear map,
is more conventionally referred to as the 'outer product',
and written $\w \v^{\dagger}$.
However, because I am working in spaces other than $\Reals^n$
(eg. $\Space{L}(\Space{V},\Space{W})$, the space of linear maps
between 2 vector spaces),
I want to avoid notations that suggest thinking in terms
of 'row' and 'column' vectors.

The following is a useful identity.
If $\t \in \Space{T}$, $\u, \v \in \Space{V}$, and $\w \in \Space{W}.$
then
\begin{equation}
\label{eq:tensor-dot}
(\t \otimes \u) (\v \otimes \w)(\u) = (\u \bullet \v) (\t \otimes \w)
\end{equation}

\item Elementary orthogonal projection:
\begin{equation}
\Projection_{\w} \v
\; \equiv \;
\left( \frac{ \w }{ \| \w \| } \otimes \frac{ \w }{ \| \w \| } \right) \v
\; = \;
\left( \frac{\w }{\|\w\|} \bullet \v \right) \frac{\w}{\|\w\|}
\end{equation}

\item Orthogonal complement:
\begin{equation}
\perp_{\w} \v
\; \equiv \;
\v \perp \w
\; \equiv \;
\v \; - \; \Projection_{\w} \v
\; = \;
\v \; - \; \left( \frac{\w}{\|\w\|} \bullet \v \right) \frac{\w}{\|\w\|}
\end{equation}

\end{itemize}

%-----------------------------------------------------------------
\setcounter{currentlevel}{\value{baseSectionLevel}}
\leveldown{Affine spaces}
\label{sec:affine-spaces}
\leveldown{Euclidean space}
%-----------------------------------------------------------------
%\setcounter{currentlevel}{\value{baseSectionLevel}}
%\leveldown{Projective spaces}
%\label{sec:Projective-spaces}
%-----------------------------------------------------------------
%\setcounter{currentlevel}{\value{baseSectionLevel}}
%\leveldown{Oriented projective spaces}
%\label{sec:Oriented-projective-spaces}
%\cite{Stolfi1991opg}
%-----------------------------------------------------------------
%\setcounter{currentlevel}{\value{baseSectionLevel}}
%\leveldown{Barycentric (convex) spaces and functions}
%-----------------------------------------------------------------
%\setcounter{currentlevel}{\value{baseSectionLevel}}
%\leveldown{Projective spaces}
%-----------------------------------------------------------------
%\setcounter{currentlevel}{\value{baseSectionLevel}}
%\leveldown{Metric spaces}
%\label{sec:Metric-spaces}
%-----------------------------------------------------------------
%\setcounter{currentlevel}{\value{baseSectionLevel}}
%\leveldown{Spherical spaces}
%-----------------------------------------------------------------
%\setcounter{currentlevel}{\value{baseSectionLevel}}
%\leveldown{Manifolds}
%\label{sec:Manifolds}
%-----------------------------------------------------------------
\setcounter{currentlevel}{\value{baseSectionLevel}}
\levelstay{Functions between spaces}
\label{sec:Functions-between-spaces}

In general, the functions discussed here map between real inner product spaces:
$\f:\Space{V} \mapsto \Space{W}$, where $\Space{V}$ is the
\textit{domain} and $\Space{W}$ is the \textit{codomain}.
The real inner product spaces are almost derived from some $\Reals^n$.

The \textit{range} of $\f$, $\range(f)$, is the set $\f(\Space{V})$,
which may be a proper subset of its codomain $\Space{W}$.
The \textit{kernel} of $\f$, $\kernel(f)$, is the set
$\kernel(\f) = \{ \v \in \Space{V} : \f(\v) = \0 \}$.

When I want to distinguish between real- and vector-valued functions,
I may use 'function' for vector-valued functions and
'functional' for real-valued ones.

I use $\Space{U}$, $\Space{V}$, $\Space{W}$ for generic linear spaces,
$\u$, $\v$, $\w$, etc., for elements of linear spaces,
usually called \textit{vectors}
and
$\f$, $\g$, $\h$ for vector-valued functions.
I generally do not distinguish $\Reals$, the real numbers,
and $\Reals^1$, or any other 1-dimensional real linear space.
I sometimes use $f$, $g$, $h$ for extra clarity in the special
case of real-valued functions.

The domains of many interesting functions,
such as those that depend on vertex positions,
are direct sum of inner product spaces.
The \textit{direct sum} $\Space{V} \oplus \Space{W}$ is the inner product space
consisting of the ordered pairs $\{ (\v,\w) : \v \in \Space{V}, \w \in \Space{W} \}$
inheriting the inner product space operations in the obvious way:
$(\v_0,\w_0) \bullet (\v_1,\w_1) = (\v_0 \bullet \v_1) + (\w_0 \bullet \w_1).$
I will usually write an element of $\oplus^n \Space{V}$ as
$(\v_0,\ldots,\v_{n-1})$
and use
$\f(\v_0,\v_1,\ldots,\v_{n-1})$
for a function that depends on $n$ vectors.

%-----------------------------------------------------------------
\leveldown{Linear functions}
\label{sec:linear-functions}

A function $\Lmap(\v):\Space{V} \mapsto \Space{W}$
is \textit{linear} iff
$\Lmap(a_0 \v_0 + a_1 \v_1) = a_0 \Lmap(\v_0) + a_1 \Lmap(\v_1)$.
I will often write $\Lmap\v \equiv \Lmap(\v)$.

Its not hard to see that, for a linear function,
the range and kernel are linear subspaces of the codomain and
domain, respectively.
Thus any linear function between inner product spaces
divides its domain and codomain each into 2 orthogonal subspaces.
The domain is divided into $\Space{V} = \kernel(\Lmap) \oplus \kernel^{\perp}(\Lmap)$,
and the codomain is divided into $\Space{W} = \range(\Lmap) \oplus \range^{\perp}(\Lmap)$.

The most common representation for linear functions is the \textit{matrix:}
Let $\Lmap(\v):\Space{V} \mapsto \Space{W}$ be linear,
$\{ \e_0^{\Space{V}} \ldots  \e_{m-1}^{\Space{V}} \}$ an orthonormal basis for $\Space{V}$,
and
$\{ \e_0^{\Space{W}} \ldots \e_{n-1}^{\Space{W}} \}$ an orthonormal  basis for $\Space{W}$
Then $\Lmap$ can be expressed as
\begin{equation}
\Lmap
 =
\sum_{i=0}^{m-1} \sum_{j=0}^{n-1} L_{ij} ( \e_i^{\Space{W}} \otimes \e_j^{\Space{V}} )
\end{equation}
$(L_{ij})$ is the matrix representation of $\Lmap$ with respect to
the two bases~\cite{halmos-1958}.

It is important to note that there are many usful
representations for linear functions other than 
matrices~\cite{mcdonald-1989b}.
Sometimes other representations are used for convenience,
or to enforce some constraint like symmetry.
In some cases, a non-matrix representation must be used,
because a particular linear transformation
cannot be accurately represented by a matrix of floating point numbers.

Examples:

\begin{itemize}

\item Column-wise:
$\Lmap = \sum_{j=0}^{n-1} ( \c_j^{\Lmap} \otimes \e_j^{\Space{V}} )$

$\c_j^{\Lmap} \in \Space{W}$ are the 'columns' of $\Lmap$.
$\linearspan\{ \c_0^{\Lmap} \ldots \c_{n-1}^{\Lmap} \} = \range(\Lmap)$
(see \autoref{sec:spans-and-projections}).

\item Row-wise:
$\Lmap = \sum_{i=0}^{m-1} ( \e_i^{\Space{W}} \otimes  \r_i^{\Lmap} )$

$\r_i^{\Lmap} \in \Space{V}$ are the 'rows' of $\Lmap$.
$\linearspan\{ \r_0^{\Lmap} \ldots \r_{m-1}^{\Lmap} \} =  \kernel(\Lmap)^{\perp}$
(see \autoref{sec:spans-and-projections}).

\item Householder:
$\h_{\v} = \Identity_{\Space{V}} - \frac{2}{\| \v \|^2} (\v \otimes \v)$

Householder maps are usually chosen to zero the elements of
a vector, or a row or column of a matrix, for a contiguous range of
indices, say, $[i_0,\ldots,i_n)$.

\end {itemize}

%-----------------------------------------------------------------
\levelstay{Affine functions}
\label{sec:affine-functions}

A function $\Amap(\v):\Space{V} \mapsto \Space{W}$
is \textit{affine} if distributes over affine combinations:
$\Amap(\sum_{i=0}^{n-1} a_i \v_i) = \sum_{i=0}^{n-1} a_i \Amap(\v_i) $
for all $\{a_i\}$ such that $1 = \sum_{i=0}^{n-1} a_i$.
(Note that I am describing affine functions on vector (linear) spaces,
rather than the slightly more general notion of affine functions on affine spaces.)
Any linear function between linear spaces is automatically affine.
The other major class of affine functions on linear spaces are the translations.
A \textit{translation,} $\Tmap_{\t}$, $\Space{V} \mapsto \Space{V}$,
simply adds a vector ($\t$) to its argument:
$\Tmap_{\t} \v = \v + \t$.
It's not hard to see that any affine function between two linear spaces
can be represented as the sum of a linear function and a translation.
A typical representation for a general affine function $\Amap : \Space{V} \mapsto \Space{W}$
is as a pair $(\Lmap,\t)$ where $\Lmap : \Space{V} \mapsto \Space{W}$ is linear,
$\t \in \Space{W}$, and $\Amap(\v) = \Lmap(\v) + \t$.

%-----------------------------------------------------------------
\levelstay{Spans and projections}
\label{sec:spans-and-projections}

Let $\Space{V}$ be an $n$-dimensional inner product space.

The \textit{linear span} of a set of $m$ vectors in $\Space{V}$
is the set of linear combinations of those vectors:
\begin{equation}
\linearspan\{ \v_0 \ldots \v_{m-1} \} = \{\v \in \Space{V} : \v = \sum_{i=0}^{m-1} a_i \v_i\}
\end{equation}
$\linearspan\{ \v_0 \ldots \v_{m-1} \}$ is a linear subspace of $\Space{V}$.

The \textit{projection} $\Projection_{\Set{S}} \v$ of a vector $\v \in \Space{V}$
onto an arbitrary subset $\Set{S} \subset \Space{V}$
is the closest point in $\Set{S}$ to $\v$.
Projection onto a linear subspace is a linear function and
can be computed by summing
elementary orthogonal projections onto an orthonormal basis for the subspace.

An orthonormal basis for $\linearspan\{ \v_0 \ldots \v_{m-1} \}$
(and $\linearspan\{ \v_0 \ldots \v_{m-1} \}^\perp$)
can be computed using the QR decomposition
of the function $\Vmap = \sum_{i=0}^{m-1} \v_i \otimes \e_i$,
(the $n \times m$ matrix whose columns are the $\v_i$)
(see \cite[sec.~5.2 ]{golub-vanloan-2012}).

The \textit{affine span} of a set of $m+1$ vectors in $\Space{V}$
is the set of affine combinations of those vectors:
\begin{equation}
\affinespan\{ \p_0 \ldots \p_{m} \} = \{\v \in \Space{V} : \v = \sum_{i=0}^{m} b_i \p_i;
1 = \sum_{i=0}^{m} b_i \}.
\end{equation}
$\affinespan\{ \p_0 \ldots \p_{m} \}$ is an affine subspace of $\Space{V}$.
$\b = ( b_0 \ldots b_m )$ are \textit{barycentric coordinates}
for $\v$ with respect to $\{ \p_0 \ldots \p_{m} \}$.
The barycentric coordinates are unique if $\{ \p_0 \ldots \p_{m} \}$
are affinely independent.

Any affine subspace, $\Space{A}$, of a linear space, $\Space{V}$ can be represented as
as a translation of a linear subspace of $\Space{V}$:
$\Space{A} = \Space{T}(\Space{A}) + \t$,
$\Space{T}(\Space{A})$ is the set of differences of elements of $\Space{A}$,
a linear subspace of $\Space{V}$.
If $\t$ is any element of $\Space{A}$.
then projection onto $\Space{A}$
can be computed as a translation of an orthogonal projection onto $\Space{T}(\Space{A})$:
$\Projection_{\Space{A}} (\p) = \t + \Projection_{\Space{T}(\Space{A})} (\p - \t)$.
Typically, we pick $\t$ to be the smallest element of $\Space{A}$.
Projection onto an affine space is clearly an affine function.

We can represent the affine span of a set of $m+1$ vectors
as a translation of a linear span:
\begin{equation}
\affinespan\{ \p_0 \ldots \p_{m} \} = \p_m + \linearspan\{\v_0 \ldots \v_{m-1}\}
\end{equation}
where $\v_i = \p_i - \p_m$,
which allows us to compute the projection onto
$\affinespan\{ \p_0 \ldots \p_{m} \}$
again using the QR decomposition
of $\Vmap = \sum_{i=0}^{m-1} \v_i \otimes \e_i$.

%-----------------------------------------------------------------
\levelstay{Linear inverses and pseudo-inverses}
\label{sec:Linear-inverses-and-pseudo-inverses}

A convenient definition for the \textit{true inverse}
of a function $\f(\v):\Space{V} \mapsto \Space{W}$ is
$\f^{-1}(\w) = \{ \v : \f(\v) = \w \}$.
The usual definition of inverse treats $\f^{-1}$
as a function from $\Space{W} \mapsto \Space{V}$,
which is undefined where the value of the true
inverse is not a set containing a single point.

For functions between inner product spaces,
the \textit{pseudo-inverse}, $f^{-}$, is a function $\Space{W} \mapsto \Space{V}$
defined everywhere on $\Space{W}$.
Let $\hat{\w}$ be an element of $\Space{W}$ closest to $\w$
such that $\f^{-1}(\w)$ is not empty.
Let $\hat{\v}$ be a minimum norm element of $\f^{-1}(\hat{\w})$.
Then $\f^{-}(\w) = \hat{\v}$.

If $\Lmap$ is linear, then it's not hard to see that
$\hat{\w} = \pi_{\range(\Lmap)} \w$, the projection of $\w$
on the range of $\Lmap$
and
$\hat{\v}$ is the unique element of $\kernel^{\perp}(\Lmap)$
such that $\Lmap(\hat{\v}) = \hat{\w}$.

The pseudo-inverse of a linear function can be characterized
by the four Moore-Penrose 
conditions~\cite[sec.~5.5.2]{golub-vanloan-2012}:
\begin{enumerate}
\item $\Lmap \Lmap^{-} \Lmap = \Lmap$
\item $\Lmap^{-} \Lmap \Lmap^{-} = \Lmap^{-}$
\item $\left( \Lmap \Lmap^{-} \right)^{\dagger} = \Lmap \Lmap^{-}$
\item $\left( \Lmap^{-} \Lmap \right)^{\dagger} = \Lmap^{-} \Lmap$
\end{enumerate}

When the 'columns' of $\Lmap$, $\r_j^{\Lmap}$
($\Lmap = \sum_{j=0}^{n-1} ( \Lmap_j^{\Space{W}} \otimes \e_j^{\Space{V}} )$)
are linearly independent,
then a useful identity is:
\begin{equation}
\label{eq:full-rank-pseudo-inverse}
\Lmap^{-} = \left( \Lmap^{\dagger} \Lmap \right)^{-1} \Lmap^{\dagger}
\end{equation}

The pseudoinverse can be computed
using standard matrix decompositions such as
the QR and SVD \cite{golub-vanloan-2012}.
The pseudoinverse is an example of a linear transformation
which should {\em not} be represented by a matrix
\cite{mcdonald-1989b}.

If $\Amap$ is affine,
let $\Amap = \Lmap + \t$,
where $\Lmap$ is linear,
and $\t$ is an element of $\range(\Amap)$.
Then $\Amap^{-}(\w) = \Lmap^{-}( \w - \t )$.


%-----------------------------------------------------------------
\setcounter{currentlevel}{\value{baseSectionLevel}}
%-----------------------------------------------------------------
\setcounter{currentlevel}{\value{baseSectionLevel}}
\levelstay{Derivatives}
\label{sec:Derivatives}

One way to view the derivative of a 
function~\cite{wiki:Frechet-derivative}
$\f:\Space{V} \mapsto \Space{W}$,
at a point $\v$,
is as the linear function $\Lmap:\Space{V} \mapsto \Space{W}$,
that best approximates the local 'slope' of $\f$ at $\v$.
(In the following, $\v$, $\u$, and $\t$ are elements of $\Space{V}$.)
To be a little more precise, we want
\begin{equation}
\lim_{ \|{\bf \delta}  \| \mapsto 0}
\frac{ \| \f(\v + {\bf \delta}) - (\f(\v) + \Lmap({\bf\delta})) \|}
{\|{\bf \delta}  \| }
 = 0
\end{equation}

\textbf{Note:} for a linear function $\Lmap$,
the derivative is constant over the domain
and the value is $\Lmap$ itself.

\textbf{TODO:} G\^{a}teaux derivative: collection of all
directional derivatives.

\textbf{TODO:} Generalized functions/distributions.

\textbf{TODO:}
relationship to currying and standard notation ambiguity vs
lisp notation.

For a concise and correct discussion, 
see Spivak \cite[chapter 2]{spivak-1965}.

\textbf{TODO:} better notation for derivatives, especially 
'evaluated at' and partial.

\begin{itemize}

\item $\Da{\f}$

In its most general form,
I denote the derivative of $\f$ by $\Da{\f}$.
Note that this is linear-function-valued function of the domain of $\f$.

\item $\Db{\f}{\u}$

I denote the derivative of $\f$ at $\u$ by $\Db{\f}{\u}$.
$\Db{\f}{\u}$ is a specific linear transformation from
the domain of $\f$ to the codomain of $\f$.

\item $\Dc{\f}{\u}{\t}$

The derivative is most often represented by the \textit{Jacobian},
the $m \times n$ matrix of partial derivatives
with respect to some bases for $\Space{V}$ and $\Space{W}$.
However, it's often easier to express the derivative clearly if we
explicitly include the argument of the linear transformation.
In this case, I write $\Dc{\f}{\u}{\t}$
for the derivative of $f$ at the point $\u$
applied to the vector $\t$.

\item $\Dd{\v_i}{\f}{(\u_0 \ldots \u_{n-1})}{\t_i}$

For functions on direct sum spaces,
$\f(\v_0,\v_1, \ldots, \v_{n-1})$, $\v_i \in \Space{V}_i$,
it's often easier to consider the derivative
with respect to one argument at a time.
I write $\Dd{\v_i}{\f}{(\u_0 \ldots \u_{n-1})}{\t_0 \ldots \t_{n-1}}$
for the derivative of $\f$ with respect to $\v_i$,
at the point $(\u_0 \ldots \u_{n-1}) \in \oplus_{i=0}^{n-1} \Space{V}_i$,
applied to the vector $\t_i \in \Space{V}_i$.

\item $\da{j}{\f} = \da{v_j}{\f}$

The traditional partial derivative of $\f$ is with respect to
a single coordinate $v_j$ of the domain.
More formally, this is the directional derivative of $\f$
in the direction of the $j$-th canonical basis vector $\e_j^{\Space{V}}$.

$\da{j}{\f}$ is a function from the domain of $\f$ to the co-domain of $\f$.
$\db{j}{\f}{\u}$ is the value of that function at $\u$.
The partial derivative is related to the derivative by
\begin{eqnarray}
\label{eq:partial-full-dervatives}
\Db{\f}{\u}
& = &
\sum_{j=0}^{m-1} \db{j}{\f}{\u} \otimes \e_j^{\Space{V}}
\\
\db{j}{\f}{\u}
& = &
\Db{\f}{\u} \e_j^{\Space{V}}
\nonumber
\end{eqnarray}

\item $\da{j}{\f_i}$

The Jacobian partial derivatives are the derivatives of
a particular coordinate of $\f$, $\f_i$, with respect to
a single coordinate $v_j$ of the domain.
$\da{j}{\f_i}$ is a real-valued function on the domain of $\f$.
$\db{j}{\f_i}{\u}$ is the value of that function at $\u$.
The Jacobian partial derivatives form the 'matrix' representation of the derivative:
\begin{equation}
\Db{\f}{\u} =
\sum_{i=0}^{m-1}
\sum_{j=0}^{n-1}
\db{j}{\f_i}{\u} \left( \e_i^{\Space{W}} \otimes \e_j^{\Space{V}} \right)
\end{equation}

\end{itemize}

In minimizing a real-valued function, $f(\v)$, $\v \in \Space{V}$,
we frequently need to know both the direction of maximum increase of $f$
the rate of increase, or slope, of $f$ in that direction.

$\Ga{f}$ is the \textit{gradient} of $f$.
The gradient has a close relationship to the derivative, $\Da{f}$,
and the two are often confused.
Recall that the derivative is a linear transformation
from the domain of $f$ to its codomain.
In the case of real-valued functions,
this means the derivative is a linear function on $\Space{V}$,
an element of the dual space of $\Space{V}$, a 'row' vector.
It's easy to see that the gradient is simply the dual (the 'transpose')
of the derivative, $\Ga{f} = (\Da{f})^{\dagger}$
(see Spivak \cite[p.~96, ex.~4-18]{spivak-1965}).

$\Ga{f}$ maps $\Space{V} \mapsto \Space{V}$.
$\Gb{f}{\u} \in \Space{V}$ is the gradient of $f$ at $\u \in \Space{V}$;
it points in the direction of most rapid increase of
$f$ and its magnitude $\| \Gb{f}{\u} \|$ is the
slope of $f$ in that direction.

Notation for the various versions of the gradient
follows that for derivatives:
$\Gc{\v_i}{f}{\u}$ is the partial gradient of $f$ with respect to $\v_i$ at
$\u = \left( \u_0 \ldots \u_{n-1} \right) \in \Space{V} = \oplus_{i=0}^{n-1} \Space{V}_i$
$\Gc{\v_i}{f}{\u}$ is an element of $\Space{V}_i$.

$(\Gb{f}{\u}) \bullet  \t$
and
$(\Gc{\v_i}{f}{\u}) \bullet \t_i$
are the analogs to exressing the derivative as a linear transformation
with an explicit argument.
$(\Gb{f}{\u}) \bullet  \t$ is a real number.
If we take $t$ to be the canonical basis for $\Space{V}$
we get an expression for $\Ga{f}$ in terms of the partial derivatives of $f$:
\begin{equation}
\label{eq:gradient-from-partials}
\Gb{f}{\u} = \sum_{j=0}^{m-1} \left( \db{j}{f}{\u} \right) \e_j^{\Space{V}}
\end{equation}

$\Ga{\f_i}$ is the gradient of a particular (real-valued) coordinate
of a vector-valued function. It is related to the derivative $\Da{\f}$
in a way simlilar to the relationship between $\Da{\f}$ and its partials $\da{j}{\f}$.
\begin{equation}
\Db{\f}{\u} = \sum_{i=0}^{n-1}  \e_i^{\Space{W}} \otimes \Gb{\f_i}{\u}
\end{equation}

The most general identity used in computing derivatives is the \textit{chain rule.}
Suppose
$\f:\Space{U} \mapsto \Space{V}$,
$\g:\Space{V} \mapsto \Space{W}$,
and
$\h = \g \circ \f : \Space{U}_0 \mapsto \Space{W}$
Then
\begin{equation}
\label{eq:chain-rule}
\Db{\h}{\u}
=  \Db{(\g \circ \f)}{\v}
=  \Db{\g}{\f(\v)}  \circ  \Db{\f} {\v}.
\end{equation}

It is sometimes useful to express this in terms of the partial derivatives:
\begin{equation}
\label{eq:chain-rule_partials}
\Db{\h}{\u} =  \sum_{i=0}^{n-1} \db{i}{\g}{\f(\u)} \otimes  \Gb{\f_i}{\u}.
\end{equation}

See Spivak \cite[Theorem~2-2]{spivak-1965}.

%-----------------------------------------------------------------
\setcounter{currentlevel}{\value{baseSectionLevel}-1}
\levelstay{Vector-valued functions}
\label{sec:Derivatives-of-Vector-valued-functions}

%-----------------------------------------------------------------
\setcounter{currentlevel}{\value{baseSectionLevel}-2}
\levelstay{Implicit functions}
\label{sec:Derivatives-of-implicit-functions}

Suppose 
$\Vector{f} : \Space{X} \times \Space{Y} \rightarrow \Space{Y}$,
for topological linear spaces $\Space{X}$ and $\Space{Y}$.

WLOG, consider the level set 
$\SetSpec{\left[ \Vector{x} , \Vector{y} \right]}
{\Vector{0} = \Vector{f}\left( \left[ \Vector{x} , \Vector{y} \right] \right)}$
(other level sets are equivalent to modifying $\Vector{f}$
by adding or subtracting 
an element of $\Space{Y}$).

All level sets are relations on $\Space{X} \times \Space{Y}$.
Under conditions on $\Da{\Vector{f}}$
(Spivak \cite[Theorem~2-12]{spivak-1965}),
there is a subset of the relation which is a 
function.

\textbf{TODO:} Implicit functions without differentiability
conditions, eg, when the level set isn't a smooth manifold,
maybe only piecewise smooth?

Specifically, 
there exists an open subset $\Set{D}  \subset \Space{X}$
and a differentiable function $\Vector{g} : \Set{D} \rightarrow \Space{Y}$
such that 
\begin{equation}\label{eq:Implicit-function}
\Vector{0} = \Vector{f}\left( \left[ \Vector{x} , \Vector{g} \left( \Vector{x} \right) \right] \right)
\end{equation}
for all 
$\Vector{x} \in  \Set{D}$.

\textbf{TODO:} Not really a very useful theorem --- 
how do you find $\Vector{g}$ and $\Set{D}$?
When is $\Set{D}$ big enough?

\textbf{TODO:} Can we extend $\Vector{g}$ to all of $\Space{X}$?

Note that 
\begin{equation}
\begin{aligned}
\Vector{h}\left(\Vector{x}\right) 
& = 
\Vector{f}\left( \left[ \Vector{x} , \Vector{g} \left( \Vector{x} \right) \right] \right)
\\
& =
\left( \Vector{f} \circ \left( \Vector{I}_{\Space{X}} \times \Vector{g} \right) \right) 
\left( \Vector{x} \right)
\end{aligned} 
\end{equation}

Equation~\ref{eq:Implicit-function}, 
the chain rule (equation~\ref{eq:chain-rule}),
and the linearity of $\Vector{I}_{\Space{X}}$
imply
\begin{equation}\label{eq:Implicit-derivative}
\begin{aligned}
\Vector{0} & = \Db{\Vector{h}}{\Vector{x}_0}
\\
& = \Db{\Vector{f}}{\left[ \Vector{x}_0 , \Vector{g}\left( \Vector{x}_0 \right) \right]}
\circ \left( 
\Vector{I}_{\Space{X}} 
\times 
\Db{\Vector{g}}{\Vector{x}_0}
\right)
\\
& = 
\left(
\De{\Vector{x}}{\Vector{f}}{\left[ \Vector{x}_0 , \Vector{g}\left( \Vector{x}_0 \right) \right]}
\circ 
\Vector{I}_{\Space{X}}
\right)
+ 
\left(
\De{\Vector{y}}{\Vector{f}}{\left[ \Vector{x}_0 , \Vector{g}\left( \Vector{x}_0 \right) \right]}
\circ 
\De{\Vector{x}}{\Vector{g}}{\Vector{x}_0}
\right)
\\
& = 
\De{\Vector{x}}{\Vector{f}}{\left[ \Vector{x}_0 , \Vector{g}\left( \Vector{x}_0 \right) \right]}
+ 
\left(
\De{\Vector{y}}{\Vector{f}}{\left[ \Vector{x}_0 , \Vector{g}\left( \Vector{x}_0 \right) \right]}
\circ 
\De{\Vector{x}}{\Vector{g}}{\Vector{x}_0}
\right)
\end{aligned}
\end{equation}
which implies
\begin{equation}
\De{\Vector{x}}{\Vector{g}}{\Vector{x}_0}
=
\left(
\De{\Vector{y}}{\Vector{f}}{\left[ \Vector{x}_0 , \Vector{g}\left( \Vector{x}_0 \right) \right]}
\right)^{-1}
\circ 
\De{\Vector{x}}{\Vector{f}}{\left[ \Vector{x}_0 , \Vector{g}\left( \Vector{x}_0 \right) \right]}
\end{equation}

When $\Space{Y}=\Space{R}$, this reduces to:
\begin{equation}
\De{\Vector{x}}{\Vector{g}}{\Vector{x}_0}
=
\dfrac{
\De{\Vector{x}}{\Vector{f}}{\left[ \Vector{x}_0 , \Vector{g}\left( \Vector{x}_0 \right) \right]}
}{
\db{\Vector{y}}{\Vector{f}}{\left[ \Vector{x}_0 , \Vector{g}\left( \Vector{x}_0 \right) \right]}
}
\end{equation}

\begin{example}[Circle]
$x^2 + y^2 - d = 0$
\end{example}

%-----------------------------------------------------------------
\setcounter{currentlevel}{\value{baseSectionLevel}-2}
\levelstay{Multilinear functions}
\label{sec:Derivatives-of-multilinear-functions}

A function
 $\f(\v_0 \ldots \v_k):\Space{V}_0 \oplus \ldots \oplus \Space{V}_k \mapsto \Space{W}$
is \textit{multilinear} if
\begin{equation}
\f(a_{00} \v_{00} + a_{01} \v_{01}, \ldots, a_{k0} \v_{k0} + a_{k1} \v_{k1})
 =  \sum_{i_0 \ldots i_k = 0,1} (a_{0i_0} \ldots a_{ki_k}) \f(\v_{0i_0} \ldots \v_{ki_k}).
\end{equation}

The derivative of $\f$
at the point $(\v_0 \ldots \v_k)$, applied to the vector $(\u_0 \ldots \u_k)$ is

\begin{equation}
\Dc{\f}{(\v_0 \ldots \v_k)}{\u_0 \ldots \u_k}
 =  \sum_{i=0,k} \f(\v_0 \ldots \v_{i-1},\u_i,\v_{i+1} \ldots \v_k).
\end{equation}

See Spivak \cite[ex.~2-14]{spivak-1965}.

%-----------------------------------------------------------------
\setcounter{currentlevel}{\value{baseSectionLevel}-2}
\levelstay{Bilinear functions}
\label{sec:Derivatives-of-bilinear-functions}

Bilinear functions are a useful special case of multilinear functions.

A function $\f(\v,\u):\Space{V}_0 \oplus \Space{V}_1 \mapsto \Space{W}$
is \textit{bilinear} if
\begin{eqnarray}
\f(a_0 \v_0 + a_1 \v_1, b_0 \u_0 + b_1 \u_1)
& =  & a_0 b_0 f(\v_0,\u_0)
+  a_0 b_1 f(\v_0,\u_1)
\\
& +  & a_1 b_0 f(\v_q,\u_0)
 +  a_1 b_1 f(\v_q,\u_1).
\nonumber
\end{eqnarray}

The derivative of $\f$
at the point $(\v_0,\u_0)$, applied to the vector $(\v,\u)$ is
\begin{equation}
\label{eq:bilinear-derivative}
\Dc{\f}{(\v_0,\u_0)}{\v,\u} = \f(\v_0,\u) + \f(\v,\u_0).
\end{equation}

See Spivak \cite[ex.~2-12]{spivak-1965}.

%-----------------------------------------------------------------
\setcounter{currentlevel}{\value{baseSectionLevel}-3}
\levelstay{Cross products}
\label{sec:Derivatives-of-cross-products}

We can view the 3-dimensional cross product
$ \times $
as a bilinear function
$\times(\v,\u) = \v \times \u : \Reals^3 \oplus \Reals^3 \mapsto \Reals^3$.
From equation \ref{eq:bilinear-derivative},
$\Dc{\times}{(\v_0,\u_0)}{\v,\u} = \v_0 \times \u + \v \times \u_0$.

Suppose
$\f:\Space{V} \mapsto \Reals^3$, and
$\g:\Space{V} \mapsto \Reals^3$.
The derivative of $\f \times \g$ is:
\begin{eqnarray}
\Dc{(\f \times \g)}{\v_0}{\v}
& =
& \Db{\times}{(\f(\v_0),\g(\v_0))} \circ (\Dc{\f}{\v_0}{\v}, \Dc{\g}{\v_0}{\v})
\\
& =
& \f(\v_0) \times \Dc{\g}{\v_0}{\v} + \Dc{\f}{\v_0}{\v} \times \g(\v_0) \nonumber
\end{eqnarray}

%-----------------------------------------------------------------
\setcounter{currentlevel}{\value{baseSectionLevel}-2}
\levelstay{Scalar products}
\label{sec:Derivatives-of-scalar-products}

Suppose
$f:\Space{V} \mapsto \Reals$, and
$\g:\Space{V} \mapsto \Space{W}$.
It follows from the chain rule that the derivative of $\h = f\g$ is:
\begin{equation}
\label{eq:scalar_product_derivative}
\Db{(f\g)}{\v} =  f(\v) \Db{\g}{\v} + \g(\v) \otimes \Gb{f}{\v}
\end{equation}

%-----------------------------------------------------------------
\setcounter{currentlevel}{\value{baseSectionLevel}-2}
\levelstay{Normalized functions}
\label{sec:Derivatives-of-normalized-functionss}

Let $\tilde{\f}$ be the normalized version of $\f$:
$\tilde{\f}  =  \frac{\f}{\| \f \|}$.
Then, from equations \ref{eq:scalar_product_derivative}
and \ref{eq:norm_derivative}:
\begin{eqnarray}
\Dc{\tilde{\f}}{\v}{\u}
& = &
\Dc{\left( \frac{\f}{\| \f \|}\right)}{\v}{\u}
\\
& = &
\frac{\Dc{\f}{\v}{\u}}{ \| \f(\v) \|}
 +
\f(\v)  \Dc{ \left( \frac{1}{\| \f \|} \right) }{\v}{\u} \nonumber \\
& = &
\frac{\Dc{\f}{\v}{\u}}
{\| \f(\v) \|}
 -
\f(\v)
\frac{\Dc{\| \f \|}{\v}{\u}}
{\|\f(\v)\|^2} \nonumber \\
& = &
\frac{\Dc{\f}{\v}{\u}}{ \| \f(\v) \| }
 -
\f(\v) \left( \frac{\f(\v)^\dagger}{\| \f(\v) \|^3}  \Dc{\f}{\v}{\u} \right) \nonumber \\
& = &
\frac{
\| \f(\v) \|^2 \Dc{\f}{\v}{\u}
 -
\f(\v)\left( \f(\v) \bullet \Dc{\f}{\v}{\u} \right)
}
{\| \f(\v) \|^3}  \nonumber \\
& = &
\frac{\| \f(\v) \|^2 \Identity_{\Space{W}} - \left( \f(\v) \otimes \f(\v) \right)  }
{ \| \f(\v) \|^3 }
\Dc{\f}{\v}{\u} \nonumber \\
& = &
\frac{\Identity_{\Space{W}} - \left( \tilde{\f}(\v) \otimes \tilde{\f}(\v) \right)  }
{\| \f(\v) \|}
\Dc{\f}{\v}{\u} \nonumber
\end{eqnarray}


We can write the derivative above without reference to the argument $\u$:
\begin{equation}
\label{eq:normalized_function_derivative}
\Db{\tilde{\f}}{\v}
 =
\Db{\left( \frac{\f}{\| \f \|} \right)}{\v}
 =
\frac{\Identity_{\Space{W}} - \left( \tilde{\f}(\v) \otimes \tilde{\f}(\v) \right) }
{ \| \f(\v) \| }
\Db{\f}{\v}
\end{equation}

A common, trivial, normalized function is the normalized version of
a vector: $\tilde{\v} =  \frac{\v}{ \| \v \| }$.

From equation \ref{eq:normalized_function_derivative}
it follows that:
\begin{equation}
\label{eq:normalized_vector_derivative}
\Db{\tilde{\v}}{\u}
 =
\Db{ \left( \frac{\v}{ \| \v \| } \right) }{\u}
 =
\frac{\Identity_{\Space{V}} - \left( \tilde{\u} \otimes \tilde{\u} \right) }
{ \| \u \| }
 =
\frac{\| \u \|^2 \Identity_{\Space{V}} - \left( \u \otimes \u \right) }
{\| \u \|^3}
\end{equation}

%-----------------------------------------------------------------
\setcounter{currentlevel}{\value{baseSectionLevel}-1}
\levelstay{Real-valued functions}
\label{sec:derivatives-of-real-valued-functions}

%-----------------------------------------------------------------
\setcounter{currentlevel}{\value{baseSectionLevel}-2}
\levelstay{Inner products}
\label{sec:derivatives-of-inner-products}

We can view the inner product on $\Space{V}$, $\v \bullet \u$,
as a bilinear function $\bullet(\v,\u) : \Space{V} \oplus \Space{V} \mapsto \Reals$.
Thus
\begin{equation}
\Dc{\bullet}{(\v_0,\u_0)}{\v,\u} = \v_0 \bullet \u + \v \bullet \u_0.
\end{equation}

Suppose
$\f:\Space{V} \mapsto \Space{V}$, and
$\g:\Space{V} \mapsto \Space{V}$.
The derivative of $\f \bullet \g$ is:
\begin{eqnarray}
\label{eq:dot_derivative}
\Dc{(\f \bullet \g)}{\v_0}{\v}
& =
& \Db{\bullet}{(\f(\v_0),\g(\v_0))} \circ (\Dc{\f}{\v_0}{\v}, \Dc{\g}{\v_0}{\v})
\\
& =
& \f(\v_0) \bullet \Dc{\g}{\v_0}{\v}  +  \g(\v_0) \bullet \Dc{\f}{\v_0}{\v} \nonumber
\end{eqnarray}

See Spivak \cite[ex.~2-13]{spivak-1965}.

%-----------------------------------------------------------------
\setcounter{currentlevel}{\value{baseSectionLevel}-2}
\levelstay{Angles}
\label{sec:derivatives-of-angles}

The angle between 2 vectors $\v_0, \v_1 \in \Space{V}$,
is the inverse cosine of their normalized inner product:
$\theta(\v_0,\v_1)
=
\cos^{-1} \left( \frac{ \v_0 \bullet \v_1 } {\|\v_0\| \|\v_1\|} \right)$.
Recall that the derivative of the $\cos^{-1}$ is
$\frac{\mathrm d}{\mathrm dx} \cos^{-1}(x) = \frac{-1}{\sqrt{1 - x^2} }$.
It follows that:
\begin{eqnarray*}
\Gc{\v_0}{\theta(\v_0,\v_1)}{\u}
& = &
\frac{-1}
{ \sqrt{1 - \left( \frac{\u_0 \bullet \u_1}{\| \u_0 \| \| \u_1 \|} \right)^2 }}
\Gc{\v_0}{\left( \frac{\u_0 \bullet \u_1}{\| \u_0 \| \| \u_1 \|} \right)}{\u}
\\
& = &
\frac{-\|\u_0\|\|\u_1\|}
{ \sqrt{\|\u_0\|^2\|\u_1\|^2 - \left( \u_0 \bullet \u_1 \right)^2 }}
\left[
\frac{\u_1}{\|\u_0\|\|\u_1\|}
+
\frac{\left( \u_0 \bullet \u_1 \right)}{\| \u1 \|}
\Gc{\v_0}{\left( \frac{1}{\| \v_0 \|} \right)} {\u}
\right]
\nonumber
\\
& = &
\frac{-\|\u_0\|\|\u_1\|}
{ \sqrt{\|\u_0\|^2\|\u_1\|^2 - \left( \u_0 \bullet \u_1 \right)^2 }}
\left[
\frac{\u_1}{\|\u_0\|\|\u_1\|}
-
\frac{\left( \u_0 \bullet \u_1 \right) \u0}{\| \u1 \| \|\u_0\|^3}
\right]
\nonumber
\\
& = &
\frac{-1}
{ \sqrt{\|\u_0\|^2\|\u_1\|^2 - \left( \u_0 \bullet \u_1 \right)^2 }}
\left[
\u_1
-
\frac{\left( \u_0 \bullet \u_1 \right) \u0}{\|\u_0\|^2}
\right]
\nonumber
\end{eqnarray*}
which results in
\begin{eqnarray}
\label{eq:angle_gradient}
\Gc{\v_0}{\theta(\v_0,\v_1)}{\u}
& = &
\frac{- \u_1 \perp \u_0}
{ \sqrt{\|\u_0\|^2\|\u_1\|^2 - \left( \u_0 \bullet \u_1 \right)^2 }}
\\
\Gc{\v_1}{\theta(\v_0,\v_1)}{\u}
& = &
\frac{- \u_0 \perp \u_1}
{ \sqrt{\|\u_0\|^2\|\u_1\|^2 - \left( \u_0 \bullet \u_1 \right)^2 }}
\nonumber
\end{eqnarray}

%-----------------------------------------------------------------
\setcounter{currentlevel}{\value{baseSectionLevel}-2}
\levelstay{Euclidean norm}
\label{sec:derivatives-of-euclidean-norm}

Let $l_2(\v) = \| \v  \|: \Space{V} \mapsto \Reals$
be the usual euclidean norm on $\Space{V}$.
Let $l_2^2(\v) = \| \v  \|^2 $
be its square and $ \| \v  \|^3$ the cube.
\begin{eqnarray}
\label{eq:l2-gradient}
\Gb{l_2}{\v} = \frac{ \v }{ \| \v  \|} &
\Gb{l_2^2}{\v} =  2\v &
\Gb{l_2^3}{\v} = 3 \| \v  \| \v \\
\Db{l_2}{\v} = \frac{ \v^\dagger }{ \| \v  \|} &
\Db{l_2^2}{\v} = 2\v{^\dagger} &
\Db{l_2^3}{\v} = 3 \| \v  \| \v^\dagger \nonumber
\end{eqnarray}

Let $\f(\v) : \Space{V} \mapsto \Space{W}$.
By the chain rule:
$\Db{\| \f \|^2}{\v}  =  2 {\f(\v)}^{\dagger} \Db{\f}{\v} $
and
$\Gb{\| \f \|^2}{\v}  =  2 \Db{\f}{\v}^\dagger \circ \f(\v)$.
\begin{eqnarray}
\label{eq:norm_derivative}
\Db{\| \f \|}{\v}
& = &
\frac{\f(\v)^\dagger}{\| \f(\v) \|} \Db{\f}{\v}  \\
\Gb{\| \f \|}{\v}
& = &
\left(\Db{\f}{\v}\right)^\dagger \circ  \frac{\f(\v)}{ \| \f(\v)  \|}
\label{eq:norm_gradient}
\end{eqnarray}

%-----------------------------------------------------------------
\setcounter{currentlevel}{\value{baseSectionLevel}-3}
\levelstay{Canonical vector 'volume'}
\label{sec:Derivative-of-canonical-vector-volume}

Let $\text{volume} : \times^{n} \Space{R} \mapsto \Space{R}$ 
be defined as
\begin{equation}
\text{volume} \left( x_0 , \ldots , x_{n-1} \right) = \prod_{i=0}^{n-1} x_i
\end{equation}
This is multilinear as a functional on $\times^{n} \Space{R}$.

We can also interpret this as a multilinear function on the
inner product space
$\oplus^{n} \Space{R}$.
In that case, $\text{volume}$ is the volume of the coordinate axis aligned
$n$-rectangle whose diagonal is $\Vector(x)$.

Note the dependence on both the choice of inner product, 
and the coordinate system.

\begin{equation}
\Dc{\text{volume}}{(\v_0 \ldots \v_{n-1})}{\u_0 \ldots \u_{n-1}}
 =  \sum_{i=0}^{n-1} \u_i \left( \prod_{j \neq i} \v_j \right).
\end{equation}

%-----------------------------------------------------------------
\setcounter{currentlevel}{\value{baseSectionLevel}-1}
\levelstay{Linear-function-valued functions}
\label{sec:Derivatives-of-linear-function-valued-functions}

The set of linear functions between two inner product spaces
$\{ \Lmap : \Space{V} \mapsto \Space{W} \}$
is itself a inner product space $\Space{L}(\Space{V},\Space{W})$,
with the inner product defined by
$\Lmap \bullet \Mmap = \sum_{i=0}^{m-1} \sum_{j=0}^{n-1} \Lmap_{ij} \Mmap_{ij}$.
The set of linear functions
$\Emap_{ij}^{\Space{L}(\Space{V},\Space{W})}  = \e_i^{\Space{W}} \otimes \e_j^{\Space{V}}$
are the canonical basis vectors for $\Space{L}(\Space{V},\Space{W})$.

If $\f$ is a function between spaces of linear functions,
$\f : \Space{L}(\Space{V}_0,\Space{W}_0) \mapsto \Space{L}(\Space{V}_1,\Space{W}_1)$,
its derivative, $\Da{\f}$,
is a function from a space of linear functions
to a space of linear functions between two
spaces of linear functions:
$\Da{\f} : \Space{L}(\Space{V}_0,\Space{W}_0) \mapsto
\Space{L}(\Space{L}(\Space{V}_0,\Space{W}_0), \Space{L}(\Space{V}_1,\Space{W}_1))$.
This can get a little confusing,
and it often helps to consider both the partial derivatives of $\f$
and the gradients of the coordinates of $\f$,
which can make it easier to apply the chain rule to
compositions of functions of functions via equation \ref{eq:chain-rule_partials}.

$\da{ij}{\f}$ is the partial derivative with respect to its $ij$-th matrix coordinate,
that is, the directional derivative of $\f$ in the direction
of the $ij$-th canonical basis vector, $\Emap_{ij}^{\Space{L}(\Space{V}_0,\Space{W}_0)}$.
As usual the value of the partial derivative at a specific
$\Lmap_0 \in  \Space{L}(\Space{V}_0,\Space{W}_0)$,
$\db{ij}{\f}{\Lmap_0}$ is an element of the co-domain of $\f$,
a linear function in  $\Space{L}(\Space{V}_1,\Space{W}_1)$.

$\Ga{\f_{kl}}$ is the gradient of the $kl$-th matrix coordinate of the value of $\f$.
As usual, the value of the gradient at a specific $\Lmap_0$,
$\Gb{\f_{kl}}{\Lmap_0}$ is an element of the domain of $\f$,
a linear function in $\Space{L}(\Space{V}_0,\Space{W}_0)$.

Note that nether of these are elements of the Jacobian of $\f$,
which needs 4 indexes: $\da{ij}{\f_{kl}}$.

I am particularly interested in computing the derivative of the
pseudo-inverse: $\Pseudoinverse(\Lmap) \equiv \Lmap^{-}$.
The set of full rank linear functions is an open set,
and we can define the derivative of $\Pseudoinverse(\Lmap)$ there.
For full rank linear functions,
we can use the chain rule and the identity
$\Lmap^{-} = \left( \Lmap^{\dagger} \Lmap \right)^{-1} \Lmap^{\dagger}$
(equation \ref{eq:full-rank-pseudo-inverse})
to compute the derivative of the pseudo-inverse
(\autoref{sec:Derivative-of-pseudo-inverse}).

To do this I will first establish partial derivatives and gradients of:
\begin{equation}
\begin{aligned}
\label{eq:transpose-derivative}
&\Transpose(\Lmap) \equiv \Lmap^{\dagger}
&&\db{ij}{\Transpose}{\Lmap} =  \e_j^{\Space{V}} \otimes \e_i^{\Space{W}}
\forall \Lmap
\\
&\h( \Lmap ) = \f ( \Lmap ) \g ( \Lmap )
&&\text{Section~\ref{sec:Derivatives-of-function-products} }
\\
&\LTL(\Lmap) \equiv \Lmap^{\dagger} \Lmap
&&\text{Section~\ref{sec:Derivatives-of-LTL} }
\\
&\Inverse(\Lmap) \equiv \Lmap^{-1}
&&\text{Section~\ref{sec:Derivative-of-inverse} }
\end{aligned}
\end{equation}

%-----------------------------------------------------------------
\setcounter{currentlevel}{\value{baseSectionLevel}-2}
\levelstay{Function products}
\label{sec:Derivatives-of-function-products}

Let
$\f : \Space{L}(\Space{V}_0,\Space{W}_0) \mapsto \Space{L}(\Space{V}_1,\Space{W}_1)$,
$\g : \Space{L}(\Space{V}_0,\Space{W}_0) \mapsto \Space{L}(\Space{U}_1,\Space{V}_1)$,
and
$\h = \f\g : \Space{L}(\Space{V}_0,\Space{W}_0) \mapsto \Space{L}(\Space{U}_1,\Space{W}_1)$.
Note that
$\db{ij}{\f}{\Lmap} \in  \Space{L}(\Space{V}_1,\Space{W}_1)$,
$\db{ij}{\g}{\Lmap} \in  \Space{L}(\Space{U}_1,\Space{V}_1)$,
and
$\db{ij}{\h}{\Lmap} \in  \Space{L}(\Space{U}_1,\Space{W}_1)$.
Consider the matrix representation of $\db{ij}{\h}{\Lmap}$:
\begin{eqnarray}
\left( \db{ij}{\h}{\Lmap} \right)_{kl}
& = &
\db{ij}{\h_{kl}}{\Lmap}
\\
& = &
\db{ij}{\left( \sum_{m} \f_{km} \g_{ml} \right)}{\Lmap}
\nonumber
\\
& = &
\sum_{m}  \left[
\left( \db{ij}{\f_{km}}{\Lmap} \right) \g_{ml}(\Lmap)
+
\f_{km}(\Lmap) \left( \db{ij}{\g_{ml}}{\Lmap} \right)
\right]
\nonumber
\\
& = &
\left[
\left( \db{ij}{\f}{\Lmap} \right) \g(\Lmap)
+
\f(\Lmap) \left( \db{ij}{\g}{\Lmap} \right)
\right]_{kl}
\nonumber
\end{eqnarray}
Therefore
\begin{equation}
\label{eq:function-product-derivative}
\db{ij}{\h}{\Lmap}
 =
\left( \db{ij}{\f}{\Lmap} \right) \g(\Lmap)
+
\f(\Lmap) \left( \db{ij}{\g}{\Lmap} \right)
\end{equation}

%-----------------------------------------------------------------
\setcounter{currentlevel}{\value{baseSectionLevel}-2}
\levelstay{$\Lmap^{\dagger} \Lmap$}
\label{sec:Derivatives-of-LTL}

A simple function on linear functions
is $\LTL(\Lmap) \equiv \Lmap^{\dagger} \Lmap
: \Space{L}(\Space{V},\Space{W}) \mapsto \Space{L}(\Space{V},\Space{V})$.

The partial derivative is computed using equations
\ref{eq:transpose-derivative}
and
\ref{eq:function-product-derivative}:

\begin{equation}
\db{ij}{\LTL}{\Lmap}
=
\left( \e_j^{\Space{V}} \otimes \e_i^{\Space{W}} \right) \Lmap
+
\Lmap^{\dagger} \left( \e_i^{\Space{W}} \otimes \e_j^{\Space{V}} \right)
=
\left( \e_j^{\Space{V}} \otimes \r_i^{\Lmap} \right)
+
\left( \r_i^{\Lmap} \otimes \e_j^{\Space{V}} \right)
\end{equation}
where $\r_i^{\Lmap} \in \Space{V}$ is the $i$th 'row' of $\Lmap$
in the representation $\Lmap = \sum_{i=0}^{m-1} \e_i^{\Space{W}} \otimes \r_i^{\Lmap}$.

The Jacobian, which has 4 indexes here, is given by:
\begin{equation}
\db{ij}{\LTL_{kl}}{\Lmap}
 =
\left( \db{ij}{\LTL}{\Lmap} \right)_{kl}
=
\delta_{jl} \Lmap_{ik}
+
\delta_{jk} \Lmap_{il}
\end{equation}
where, as usual, $\delta_{ij} = 1$ if $i=j$ and  $0$ if $i \neq j$.
From the Jacobian, we can compute the gradients of $\LTL_{kl}$
using equation \ref{eq:gradient-from-partials}
and the fact that
$\Emap_{ij}^{\Space{L}(\Space{V},\Space{W})}  = \e_i^{\Space{W}} \otimes \e_j^{\Space{V}}$
are the canonical basis vectors for $\Space{L}(\Space{V},\Space{W})$:
\begin{eqnarray}
\Gb{\LTL_{kl}}{\Lmap}
& = &
\sum_{ij}
\left( \db{ij}{\LTL_{kl}}{\Lmap} \right)
\left( \e_i^{\Space{W}} \otimes \e_j^{\Space{V}} \right)
\\
& = &
\sum_{ij}
\left( \delta_{jl} \Lmap_{ik} + \delta_{jk} \Lmap_{il} \right)
\left( \e_i^{\Space{W}} \otimes \e_j^{\Space{V}} \right)
\nonumber
\\
& = &
\sum_{i}
\left(
\Lmap_{ik}  \e_i^{\Space{W}} \otimes \e_l^{\Space{V}}
\right)
+
\sum_{i}
\left(
\Lmap_{il}  \e_i^{\Space{W}} \otimes \e_k^{\Space{V}}
\right)
\nonumber
\\
& = &
\left(
\c_k^{\Lmap} \otimes \e_l^{\Space{V}}
\right)
+
\left(
\c_l^{\Lmap} \otimes \e_k^{\Space{V}}
\right)
\nonumber
\end{eqnarray}
where $\c_j^{\Lmap} \in \Space{W}$ is the $j$th 'column' of $\Lmap$
in the representation
$\Lmap = \sum_{j=0}^{n-1} \c_j^{\Lmap} \otimes \e_j^{\Space{V}}$.

%-----------------------------------------------------------------
\setcounter{currentlevel}{\value{baseSectionLevel}-2}
\levelstay{Inverse}
\label{sec:Derivative-of-inverse}

$\Inverse()$ here is interpreted in the traditional sense:
$\Lmap^{-1}(\w) = \v$ if there exists a unique $\v$ such that $\w = \Lmap(\v)$,
and is either considered undefined, or assigned an arbitrary
value, such as $\0$, otherwise.
A function $\Lmap : \Space{V} \mapsto \Space{W}$ is \textit{invertible}
if, for all $\w \in \Space{W}$, there exists a $\v$ such that
$\w = \Lmap \v$.
In any reasonable topology,
the set of invertible linear functions $\Space{V} \mapsto \Space{W}$
is an open subset of the set of all linear functions,
and $\Inverse()$ is continuous and differentiable there.

The partial derivative is the value of the following, when the limit exists:
\begin{displaymath}
\db{ij}{\Inverse()}{\Lmap}
 =
\lim_{ h \mapsto 0}
\frac{ \left( \Lmap + h (\e_i^{\Space{W}} \otimes \e_j^{\Space{V}}) \right)^{-1} - \Lmap^{-1} }{h}
\end{displaymath}
Note that
\begin{displaymath}
\Lmap + h (\e_i^{\Space{W}} \otimes \e_j^{\Space{V}})
 =
\left( \Identity^{\Space{W}} - ( -h ( \e_i^{\Space{W}} \otimes \e_j^{\Space{V}} )) \Lmap^{-1} \right) \Lmap
\end{displaymath}
and
\begin{eqnarray*}
\left( \Lmap + h (\e_i^{\Space{W}} \otimes \e_j^{\Space{V}}) \right)^{-1}
& = &
\Lmap^{-1} \left( \Identity^{\Space{W}} - ( -h )( \e_i^{\Space{W}} \otimes \e_j^{\Space{V}} ) \Lmap^{-1} \right)^{-1}
\\
& = &
\Lmap^{-1} \sum_{k=0}^{\infty} \left( -h ( \e_i^{\Space{W}} \otimes \e_j^{\Space{V}} ) \Lmap^{-1} \right)^{k}
\nonumber
\end{eqnarray*}
Therefore
\begin{displaymath}
\frac{ \left( \Lmap + h (\e_i^{\Space{W}} \otimes \e_j^{\Space{V}}) \right)^{-1} - \Lmap^{-1} }{h}
 =
- \Lmap^{-1} ( \e_i^{\Space{W}} \otimes \e_j^{\Space{V}} )  \Lmap^{-1} + O(h)
\end{displaymath}
which implies
\begin{equation}
\da{ij}{\Lmap^{-1}}
 =
- \left[
\Lmap^{-1}
\left( \e_i^{\Space{W}} \otimes \e_j^{\Space{V}} \right)
\Lmap^{-1}
\right]
\end{equation}

\newgeometry{onecolumn=true}

%-----------------------------------------------------------------
\setcounter{currentlevel}{\value{baseSectionLevel}-2}
\levelstay{Pseudo-inverse}
\label{sec:Derivative-of-pseudo-inverse}

$\Pseudoinverse(\Lmap) \equiv \Lmap^{-}$

If $\kernel(\Lmap) = \0$, $\Lmap$ is said to have \textit{full rank}.
The set of full rank linear functions is an open set,
and we can define the derivative of $\Pseudoinverse(\Lmap)$ there.
For a full rank function,
$\Lmap^{-} = \left( \Lmap^{\dagger} \Lmap \right)^{-1} \Lmap^{\dagger}$
(see equation \ref{eq:full-rank-pseudo-inverse}).
It follows from equation \ref{eq:function-product-derivative} that
\begin{eqnarray}
\db{ij}{\Pseudoinverse}{\Lmap}
& = &
\db{ij}{\Inverse(\LTL())\Transpose()}{\Lmap}
\\
& = &
\left[
\left( \db{ij}{\Inverse(\LTL())}{\Lmap} \right)
\Lmap^{\dagger}
\right]
+
\left[
\left( \Lmap^{\dagger} \Lmap \right)^{-1}
\db{ij}{\Transpose()}{\Lmap}
\right]
\nonumber
\\
& = &
\left[
\left( \db{ij}{\Inverse(\LTL())}{\Lmap} \right)
\Lmap^{\dagger}
\right]
+
\left[
\left( \Lmap^{\dagger} \Lmap \right)^{-1}
\left( \e_j^{\Space{V}} \otimes \e_i^{\Space{W}} \right)
\right]
\nonumber
\end{eqnarray}

By the chain rule
\begin{eqnarray}
\Db{\Inverse(\LTL())}{\Lmap}
& = &
\sum_{kl}
\db{kl}{\Inverse}{\Lmap^{\dagger}\Lmap}
\otimes
\Gb{\LTL_{kl}}{\Lmap}
\\
& = &
\sum_{kl}
- \left[
\left( \Lmap^{\dagger} \Lmap \right)^{-1}
\left( \e_k^{\Space{V}} \otimes \e_l^{\Space{V}} \right)
\left( \Lmap^{\dagger} \Lmap \right)^{-1}
\right]
\otimes
\left[
\left( \c_k^{\Lmap} \otimes \e_l^{\Space{V}} \right)
+
\left( \c_l^{\Lmap} \otimes \e_k^{\Space{V}} \right)
\right]
\nonumber
\end{eqnarray}

To minimize confusion,
recall that $\Db{\Inverse(\LTL())}{\Lmap}$ is
a linear function from $\Space{L}(\Space{V},\Space{W}) \mapsto \Space{L}(\Space{V},\Space{V})$.
Note that the central tensor product ($\otimes$) above
is a product of
$
- \left[
\left( \Lmap^{\dagger} \Lmap \right)^{-1}
\left( \e_k^{\Space{V}} \otimes \e_l^{\Space{V}} \right)
\left( \Lmap^{\dagger} \Lmap \right)^{-1}
\right]
$,
an element of $\Space{L}(\Space{V},\Space{V})$
and
$
\left[
\left( \c_k^{\Lmap} \otimes \e_l^{\Space{V}} \right)
+
\left( \c_l^{\Lmap} \otimes \e_k^{\Space{V}} \right)
\right]
$,
an element of $\Space{L}(\Space{V},\Space{W})$.

It follows from equation \ref{eq:partial-full-dervatives} that
\begin{eqnarray}
\db{ij}{\Inverse(\LTL())}{\Lmap}
& = &
\Db{\Inverse(\LTL())}{\Lmap}
\left( \e_i^{\Space{W}} \otimes \e_j^{\Space{V}} \right)
\\
& = &
\sum_{kl}
- \left[
\left( \Lmap^{\dagger} \Lmap \right)^{-1}
\left( \e_k^{\Space{V}} \otimes \e_l^{\Space{V}} \right)
\left( \Lmap^{\dagger} \Lmap \right)^{-1}
\right]
\otimes
\left[
\left( \c_k^{\Lmap} \otimes \e_l^{\Space{V}} \right)
+
\left( \c_l^{\Lmap} \otimes \e_k^{\Space{V}} \right)
\right]
\left( \e_i^{\Space{W}} \otimes \e_j^{\Space{V}} \right)
\nonumber
\\
& = &
\sum_{kl}
- \left[
\left( \Lmap^{\dagger} \Lmap \right)^{-1}
\left( \e_k^{\Space{V}} \otimes \e_l^{\Space{V}} \right)
\left( \Lmap^{\dagger} \Lmap \right)^{-1}
\right]
\left[
\delta_{jl}
\Lmap_{ik}
+
\delta_{jk}
\Lmap_{il}
\right]
\nonumber
\\
& = &
-
\left( \Lmap^{\dagger} \Lmap \right)^{-1}
\left[
\sum_{k}
\Lmap_{ik}
\left(
\left( \e_k^{\Space{V}} \otimes \e_j^{\Space{V}} \right)
+
\left( \e_j^{\Space{V}} \otimes \e_k^{\Space{V}} \right)
\right)
\right]
\left( \Lmap^{\dagger} \Lmap \right)^{-1}
\nonumber
\\
& = &
-
\left( \Lmap^{\dagger} \Lmap \right)^{-1}
\left[
\left( \r_i^{\Lmap} \otimes \e_j^{\Space{V}} \right)
+
\left( \e_j^{\Space{V}} \otimes \r_i^{\Lmap} \right)
\right]
\left( \Lmap^{\dagger} \Lmap \right)^{-1}
\nonumber
\end{eqnarray}

Putting it all together:
\begin{eqnarray}
\db{ij}{\Pseudoinverse}{\Lmap}
& = &
\left[
-
\left( \Lmap^{\dagger} \Lmap \right)^{-1}
\left[
\left( \r_i^{\Lmap} \otimes \e_j^{\Space{V}} \right)
+
\left( \e_j^{\Space{V}} \otimes \r_i^{\Lmap} \right)
\right]
\left( \Lmap^{\dagger} \Lmap \right)^{-1}
\Lmap^{\dagger}
\right]
+
\left[
\left( \Lmap^{\dagger} \Lmap \right)^{-1}
\left( \e_j^{\Space{V}} \otimes \e_i^{\Space{W}} \right)
\right]
\nonumber
\\
& = &
\left( \Lmap^{\dagger} \Lmap \right)^{-1}
\left[
\left( \e_j^{\Space{V}} \otimes \e_i^{\Space{W}} \right)
-
\left(
\left[
\left( \r_i^{\Lmap} \otimes \e_j^{\Space{V}} \right)
+
\left( \e_j^{\Space{V}} \otimes \r_i^{\Lmap} \right)
\right]
\left( \Lmap^{\dagger} \Lmap \right)^{-1}
\Lmap^{\dagger}
\right)
\right]
\nonumber
\\
& = &
\left( \Lmap^{\dagger} \Lmap \right)^{-1}
\left[
\left( \e_j^{\Space{V}} \otimes \e_i^{\Space{W}} \right)
-
\left(
\left( \r_i^{\Lmap} \otimes \e_j^{\Space{V}} \right)
+
\left( \e_j^{\Space{V}} \otimes \r_i^{\Lmap} \right)
\right)
\Lmap^{-}
\right]
\end{eqnarray}

\restoregeometry


