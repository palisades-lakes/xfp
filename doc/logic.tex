%-----------------------------------------------------------------
%-----------------------------------------------------------------
%-----------------------------------------------------------------
\setcounter{currentlevel}{\value{baseSectionLevel}}
\levelstay{Logic}
\label{sec:Logic}

\cite{wiki:Logic}

\setcounter{currentlevel}{\value{currentlevel}-1}

%-----------------------------------------------------------------
\levelstay{Zeroth order logic}
\label{sec:Zeroth_order_logic}

Classical Zeroth-order (propositional) 
logic~\cite{iep:Propositional_logic,
wiki:Propositional_calculus,
wiki:Zeroth_order_logic}

Subtle difference sometimes between propositional
and zeroth-order logic 
(=binary truth functional propositional logic).

Formal system:
language with atomic symbols
and logical operators defining well-formed formulas,
and inference rules that a set of axiom formulas 
and return deduced formulas.
\begin{description}
\item[Atoms]  $A, B, \ldots P, Q, \ldots$, 
that may be asisgned values \textsf{true} or \textsf{false}.

\item[Operators] $\lnot P$, $P \wedge Q$, $P \vee Q$, 
$P \Rightarrow Q$, $P \Leftrightarrow Q$. 
Minimal set is $\lnot$ and 
any $2$ of  $\wedge, \vee, \Rightarrow$;
sufficient to define $2$ remaining operators.

\item[Propositions] (Well formed formulas)
Recursively: atoms and propostions combined with operators,
including parens for grouping: $(P \wedge Q) \Rightarrow R$.

\item[Inference rules] ${P \Rightarrow Q), P} \vdash Q$:
Takes a set of propositions (with truth assignments?)
and returns another proposition.
\end{description}

A variety of such formal languages.

Syntatic entailment: formula derived from set of axioms 
using inference rules ina fininte number of steps.

Semantic entailment: formula evaluates to \textsf{true}
under all possible truth assignements to variables
in axioms.

Issues are 
\begin{description}
\item[Soundness]
All syntactically entailed formulas
are semantically entailed, that is,
a sequence of inferences will never lead to a contradiction.

\item[Completeness] 
all semantically entailed formaulas 
are syntactically entailed, that is,
any statement which is consistent with the axioms under all 
possible truth assignements is derivable (somehow) using the 
inference rules.
\end{description}soundness 

%-----------------------------------------------------------------
\levelstay{Intuitionist zeroth order logic}
\label{sec:Intuitionist_zeroth_order_logic}

\cite{wiki:Intuitionistic_logic}

No law of excluded middle~\cite{wiki:Law_of_excluded_middle}.

%-----------------------------------------------------------------
\levelstay{Modal logic}
\label{sec:Modal_logic}

\cite{wiki:Modal_logic}

Possibility vs necessity.

Epistemic: state of kn owledge

Temporal 

%-----------------------------------------------------------------
\levelstay{Paraconsistent logic}
\label{sec:Paraconsistent_logic}

\cite{wiki:Paraconsistent_logic}

Get rid of "from a contradiction, anything follows".
%-----------------------------------------------------------------
\levelstay{First order logic}
\label{sec:First_order_logic}

First-order (aka predicate) logic.~\cite{wiki:First_order_logic,
sep:logic_firstorder_emergence}

\begin{description}
\item[Truth values] Usually $\{\mathsf{true},\mathsf{false}\}$,
but more than $2$ values is possible.
\item[Domain of discourse] A set $\Set{D}$.
\item[Constants] $A,B,C, \ldots$ names for values in the
domain.
\item[Variables] $a,b,c, \ldots x,y,\dots$ taking on values in the
domain.~\cite{wiki:Free_variables_and_bound_variables}
\item[Equality] combine atoms and variables.
\item[Functions] combine atoms and variables.
\item[Quantifiers]   
$\exists x \in \Set{S}$, $\forall x \in \Set{S}$ 
(where $\Set{S} \subseteq \Set{D}$);
$\exists x$ means $\exists x \in \Set{D}$, 
and
$\forall x$ means $\forall x \in \Set{D}$) 
bind variables in
formulas.~\cite{wiki:Quantifier_logic}
More specialized quantifiers are possible:
$\exists \textrm{ a unique } x \in \Set{S}$
\item[Predicates] combine atoms and variables.
\end{description} 

Differentiate 1st order \textit{language} 
(no domain of discourse, just symbols)
from \textit{interpretation} 
(assignment of terms to elements of $\Set{D}$).

See also models/structures.~\cite{wiki:Model_theory}

(Circularity w.r.t set theory?
Zermelo-Fraenkel uses 1st order logic to define sets,
but 1st order logic is defined using sets\ldots?)

\setcounter{currentlevel}{\value{currentlevel}-1}

\levelstay{First order language}
\label{sec:First_order_language}

Syntactic rules without domain of discourse.

Signature: arities of predicates and functions.

Prenex normal form (PNF):~\cite{wiki:Prenex_normal_form} 
\textit{prefix} containing all quantifiers 
followed by quantifier-free \textit{matrix}.

In classical logic, every wff has a prenex equivalent;
not true for intuitionistic logic.

Example:
$\forall x 
((\exists y\phi (y))
\lor 
((\exists z\psi (z))\rightarrow \rho (x)))$ is not prenex.
$\forall x\exists y\forall z
(\phi (y)\lor (\psi (z)\rightarrow \rho (x)))$ 
is equivalent prenex. 
(does this fail in intuitionistic logic?)



\levelstay{First order model}
\label{sec:First_order_model}

Domain of discourse and evaluation of formulas.

Equality.
(does this belong here? can also be axioms on a theory.)

\levelstay{First order theory}
\label{sec:First_order_theory}

Axioms (and axiom schemas).~\cite{wiki:List_of_first_order_theories}

Consistency: no contradiction derivable from axioms.

Completeness: any formula can be proven 
\textsf{true} or \textsf{false} 
(axioms plus formula permit derivation of 
\textsf{true} or \textsf{false} but not both).

\levelstay{First order deductive systems}
\label{sec:First_order_deductive_systems}

Rules of inference.

\levelstay{Monadic first order logic}
\label{sec:Monadic_first_order_logic}

All predicates and functions are unary 
(1 argument).~\cite{wiki:Monadic_predicate_calculus}

Decidable, not very expressive.

\levelstay{Many sorted first order models}
\label{sec:Many_sorted_first_order_models}

Hilbert's function calculus.~\cite{sep:logic_firstorder_emergence}

Domain of discourse has multiple sets;
variables 'typed' as having values in some particular 
set.~\cite{sep:modeltheory_fo}

(Is this really different from
axiom restricting variable to subset of domain?()


\setcounter{currentlevel}{\value{currentlevel}+1}
%-----------------------------------------------------------------
\levelstay{Second order logic}
\label{sec:Second_order_logic}

Second-order logic~\cite{wiki:Second_order_logic,
wiki:Second_order_propositional_logic}

%-----------------------------------------------------------------
\levelstay{Higher order logic}
\label{sec:Higher_order_logic}

Higher-order logic~\cite{wiki:Higher_order_logic}


 %-----------------------------------------------------------------
\setcounter{currentlevel}{\value{baseSectionLevel}}
\levelstay{Constructivism}
\label{sec:Constructivism}

\cite{Feferman:2000,Diez:2002,sep:mathematics-constructive}

As opposed to classical logic~\cite{wiki:Classical_logic}.

Usually drop law of excluded middle 
($P \vee \lnot P$).\cite{wiki:Law_of_excluded_middle}
It is possible that both $P$ and $\lnot P$ are not provable,
so $P \vee \lnot P$ is not provable.

But keep law of non-contradiction
($\lnot ( P \wedge (\lnot P))$).~\cite{wiki:Law_of_noncontradiction}
In other words, it is impossible to prove both.

Witnesses: Proving $\exists _{x\in \Set{X}}P(x)$ requires
constructing a specific $a \in \Set{X}$ such that $P(a)$ is true; 
$a$ is a \textit{witness} for $\exists _{x\in \Set{X}}P(x)$.

People\cite{wiki:Constructivism_philosophy_of_mathematics}:
\begin{description}
\item[Leopold Kronecker] old constructivism, semi-intuitionism.
\item[L. E. J. Brouwer] forefather of intuitionism.
\item[A. A. Markov ] forefather of Russian school of constructivism.
\item[Arend Heyting] formalized intuitionistic logic and theories.
\item[Per Martin-Löf] founder of constructive type theories.
\item[Errett Bishop] promoted a version of constructivism 
claimed to be consistent with classical mathematics.
\item[Paul Lorenzen] developed constructive analysis.
\end{description}

%-----------------------------------------------------------------
\setcounter{currentlevel}{\value{currentlevel}-1}
\levelstay{Constructive analysis}
\label{Constructive analysis}

In constructive 
analysis~\cite{wiki:Constructive_analysis,
Bridger:2019,henle:2012:numbers},
varieties of constructionism are divided by
what sort of countable $\rightarrow$ countable functions can be 
constructed?~\cite{wiki:Constructivism_philosophy_of_mathematics}
\begin{description}
\item[Free choice sequences] 
\autoref{sec:Choice_sequence}\cite{wiki:Choice_sequence}.
\item[Algorithms/Computable functions] 
\autoref{sec:Computable_function}~\cite{wiki:Computable_function}
 results in computable numbers (reals?).
\end{description}

\setcounter{currentlevel}{\value{currentlevel}-1}
\levelstay{Choice sequences}
\label{sec:Choice_sequences}

\begin{description}
\item[Lawlike] determined completely (somehow?),
eg, $\mathbb{N}$ or functions $\mathbb{N} \mapsto \mathbb{N}$,
functions from the first $k$ elements of the sequence to 
$\mathbb{N}$.
(How is this different from a computable function?)
\item[Lawless] only a finite prefix determined at any point.
Example is rolls of a die. (Very dubious\ldots.)
\end{description}

Axioms 
(Are these only for lawless sequences?
They don't seem 'constructive' to me\ldots):
\begin{description}
\item[open data] Let $\alpha \in n$ mean that the first items
in the choice sequence $\alpha$ match the finite sequence $n$.
Then for any predicate $A$,
$A(\alpha )$ implies  $\exists n$ such that
$\alpha \in n$ and $\forall \beta \in n$ we have $A(\beta )$.
In other words, the truth of any predicate on a sequence can be
determined from a finite prefix.
\item[density] $\forall n\,\exists \alpha [\alpha \in n]$.
There exists some sequence that has any given finite prefix.
\end{description}


\levelstay{Computable function}
\label{sec:Computable_function}
\cite{wiki:Computable_function}

%-----------------------------------------------------------------
\levelstay{Cardinality of $\mathbb{R}$}
\label{sec:Cardinality_of_R}

Issues with Cantor's diagonal 
argument~\cite{wiki:Cantors_diagonal_argument}.

(Starts with a given enumeration of (say) binary expansions of 
elements of $\mathbb{R}$, then \textit{constructs} a missing
element. But where does the enumeration come from?
Seems like a proof by contradiction\ldots?)

%-----------------------------------------------------------------
\levelstay{Axiom of choice}
\label{sec:Axiom_of_choice}

One interpretation of 'constructive' is
'provable in $\textsf{ZF}$ set theory without the axiom of 
choice.'~\cite{iep:Set_theory,wiki:Axiom_of_choice}
(See \autoref{sec:Zermelo-Fraenkel-set-theory} for \textsf{ZF}.)
But \textsf{ZF} may be considered 'not constructive'
itself.

In some set theories, the axiom of choice implies
the law of excluded middle; in others it doesn't.
When it does, constructionists use weaker versions
that don't imply $P \, \wedge \, \lnot P$.
So some constructionists restrict themselve

 %-----------------------------------------------------------------
\levelstay{Measure theory}
\label{sec:Measure_theory}

Bishop~\cite{bishop1967foundations,bishop1972constructive,bishop1985constructive}

 %-----------------------------------------------------------------
\setcounter{currentlevel}{\value{currentlevel}+1}
 