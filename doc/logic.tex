%-----------------------------------------------------------------
%-----------------------------------------------------------------
%-----------------------------------------------------------------
\setcounter{currentlevel}{\value{baseSectionLevel}}
\levelstay{Logic}
\label{sec:Logic}

\cite{wiki:Logic}

\setcounter{currentlevel}{\value{currentlevel}-1}

%-----------------------------------------------------------------
\levelstay{Formal languages}
\label{sec:Formal_languages}

\levelstay{Proof theory}
\label{sec:Proof_theory}

\levelstay{Model theory}
\label{sec:Model_theory}

%-----------------------------------------------------------------
\levelstay{Zeroth order logic}
\label{sec:Zeroth_order_logic}

Classical zeroth-order (propositional) 
logic~\cite{iep:Propositional_logic,
wiki:Propositional_calculus,
wiki:Zeroth_order_logic}

Subtle difference sometimes between propositional
and zeroth-order logic 
(=binary truth functional propositional logic).

Formal system:
language with atomic symbols
and logical operators defining well-formed formulas,
and inference rules that a set of axiom formulas 
and return deduced formulas.
\begin{description}
\item[Atoms]  $A, B, \ldots P, Q, \ldots$, 
that may be asisgned values \textsf{true} or \textsf{false}.

\item[Operators] $\lnot P$, $P \wedge Q$, $P \vee Q$, 
$P \Rightarrow Q$, $P \Leftrightarrow Q$. 
Minimal set is $\lnot$ and 
any $2$ of  $\wedge, \vee, \Rightarrow$;
sufficient to define $2$ remaining operators.

\item[Propositions] (Well formed formulas)
Recursively: atoms and propostions combined with operators,
including parens for grouping: $(P \wedge Q) \Rightarrow R$.

\item[Inference rules] ${P \Rightarrow Q), P} \vdash Q$:
Takes a set of propositions (with truth assignments?)
and returns another proposition.
\end{description}

A variety of such formal languages.

Syntatic entailment: formula derived from set of axioms 
using inference rules ina fininte number of steps.

Semantic entailment: formula evaluates to \textsf{true}
under all possible truth assignements to variables
in axioms.

Issues are 
\begin{description}
\item[Soundness]
All syntactically entailed formulas
are semantically entailed, that is,
a sequence of inferences will never lead to a contradiction.

\item[Completeness] 
all semantically entailed formaulas 
are syntactically entailed, that is,
any statement which is consistent with the axioms under all 
possible truth assignements is derivable (somehow) using the 
inference rules.
\end{description}soundness 

%-----------------------------------------------------------------
\levelstay{Intuitionist zeroth order logic}
\label{sec:Intuitionist_zeroth_order_logic}

\cite{wiki:Intuitionistic_logic}

No law of excluded middle~\cite{wiki:Law_of_excluded_middle}.

%-----------------------------------------------------------------
\levelstay{Modal logic}
\label{sec:Modal_logic}

\cite{wiki:Modal_logic}

Possibility vs necessity.

Epistemic: state of kn owledge

Temporal 

%-----------------------------------------------------------------
\levelstay{Paraconsistent logic}
\label{sec:Paraconsistent_logic}

\cite{wiki:Paraconsistent_logic}

Get rid of "from a contradiction, anything follows".
%-----------------------------------------------------------------
\levelstay{First order logic}
\label{sec:First_order_logic}

First-order (aka predicate) logic.~\cite{wiki:First_order_logic,
sep:logic_firstorder_emergence}

\begin{description}
\item[Truth values] Usually $\{\mathsf{true},\mathsf{false}\}$,
but more than $2$ values is possible.
\item[Domain of discourse] A set $\Set{D}$.
\item[Constants] $A,B,C, \ldots$ names for values in the
domain.
\item[Variables] $a,b,c, \ldots x,y,\dots$ taking on values in the
domain.~\cite{wiki:Free_variables_and_bound_variables}
\item[Equality] combine atoms and variables.
\item[Functions] combine atoms and variables.
\item[Quantifiers]   
$\exists x \in \Set{S}$, $\forall x \in \Set{S}$ 
(where $\Set{S} \subseteq \Set{D}$);
$\exists x$ means $\exists x \in \Set{D}$, 
and
$\forall x$ means $\forall x \in \Set{D}$) 
bind variables in
formulas.~\cite{wiki:Quantifier_logic}
More specialized quantifiers are possible:
$\exists \textrm{ a unique } x \in \Set{S}$
\item[Predicates] combine atoms and variables.
\end{description} 

Differentiate 1st order \textit{language} 
(no domain of discourse, just symbols)
from \textit{interpretation} 
(assignment of terms to elements of $\Set{D}$).

See also models/structures.~\cite{wiki:Model_theory}

(Circularity w.r.t set theory?
Zermelo-Fraenkel uses 1st order logic to define sets,
but 1st order logic is defined using sets\ldots?)

\setcounter{currentlevel}{\value{currentlevel}-1}

\levelstay{First order language}
\label{sec:First_order_language}

Syntactic rules without domain of discourse.

Signature: arities of predicates and functions.

Prenex normal form (PNF):~\cite{wiki:Prenex_normal_form} 
\textit{prefix} containing all quantifiers 
followed by quantifier-free \textit{matrix}.

In classical logic, every wff has a prenex equivalent;
not true for intuitionistic logic.

Example:
$\forall x 
((\exists y\phi (y))
\lor 
((\exists z\psi (z))\rightarrow \rho (x)))$ is not prenex.
$\forall x\exists y\forall z
(\phi (y)\lor (\psi (z)\rightarrow \rho (x)))$ 
is equivalent prenex. 
(does this fail in intuitionistic logic?)



\levelstay{First order model}
\label{sec:First_order_model}

Domain of discourse and evaluation of formulas.

Equality.
(does this belong here? can also be axioms on a theory.)

\levelstay{First order theory}
\label{sec:First_order_theory}

Axioms (and axiom schemas).~\cite{wiki:List_of_first_order_theories}

Consistency: no contradiction derivable from axioms.

Completeness: any formula can be proven 
\textsf{true} or \textsf{false} 
(axioms plus formula permit derivation of 
\textsf{true} or \textsf{false} but not both).

\levelstay{First order deductive systems}
\label{sec:First_order_deductive_systems}

Rules of inference.

\levelstay{Monadic first order logic}
\label{sec:Monadic_first_order_logic}

All predicates and functions are unary 
(1 argument).~\cite{wiki:Monadic_predicate_calculus}

Decidable, not very expressive.

\levelstay{Many sorted first order models}
\label{sec:Many_sorted_first_order_models}

Hilbert's function calculus.~\cite{sep:logic_firstorder_emergence}

Domain of discourse has multiple sets;
variables 'typed' as having values in some particular 
set.~\cite{sep:modeltheory_fo}

(Is this really different from
axiom restricting variable to subset of domain?()


\setcounter{currentlevel}{\value{currentlevel}+1}
%-----------------------------------------------------------------
\levelstay{Second order logic}
\label{sec:Second_order_logic}

Second-order logic~\cite{wiki:Second_order_logic,
wiki:Second_order_propositional_logic}

%-----------------------------------------------------------------
\levelstay{Higher order logic}
\label{sec:Higher_order_logic}

Higher-order logic~\cite{wiki:Higher_order_logic}


 %-----------------------------------------------------------------
\setcounter{currentlevel}{\value{baseSectionLevel}}
\levelstay{Constructivism}
\label{sec:Constructivism}

\cite{Feferman:2000,Diez:2002,sep:mathematics_constructive,
wiki:Constructivism_philosophy_of_mathematics}

As opposed to classical logic~\cite{wiki:Classical_logic}.

%-----------------------------------------------------------------
\setcounter{currentlevel}{\value{baseSectionLevel}-1 }
\levelstay{Feferman}
\label{sec:Feferman}
\cite{Feferman1998LightOfLogic,Feferman:2000}

\setcounter{currentlevel}{\value{currentlevel}-1 }

\levelstay{Deciding the undecidable}
\label{sec:Deciding_the_undecidable}

\cite[ch.~1 ``Deciding the undecidable'']{Feferman1998LightOfLogic}

Target general audience.

Background on $3$ of Hilbert's problems
and effect of G\"{o}del incompleteness theorems.

Some discussion of constructive approaches, 
predicative/impredicative, finitary/infinitary.

Metamathematical proof theory as result.

Question of (non)existence 
'genuine absolutely undeciable problems'.

Conclusion mentions recent progess
in purely finitary approaches~\cite[ch.~10]{Feferman1998LightOfLogic}
and 
use of 'proof-theoretically very weak systems'
as basis for 'an enormous amount of scientifically applicable
mathematics.'~\cite[ch.~14]{Feferman1998LightOfLogic}
(!! what I'm looking for !!, ie, 
what's the 'minimal' mathematical structure needed to
solve the real world problems I'm interested in?)

\levelstay{Is Cantor Necessary?}
\label{sec:Is_Cantor_Necessary}
\cite{Feferman1989CantorNecessary}
Cantor diagonal argument:
\hfill\break
$\OneToOne$ map between binary sequences ($2^{\mathbb{N}}$)
and $[0,1]\subset\mathbb{R}$
Does that need a proof?
What about finite sequences vs trailing $1$s? 
\hfill\break
Then assume existence of a $\OneToOne$  
map between $2^{\mathbb{N}}$s 
and $\mathbb{N}$,
without any explicit construction.
Construct an explict missing sequence 
for any candidate $\OneToOne$ map,
by flipping the $i$th bit in the $i$th sequence,
thus a contradiction?
So $\mathfrak{c} = \mathrm{card}(2^{\mathbb{N}}) 
= \mathrm{card}(\mathbb{R})$ must be greater than 
$\aleph_0 = \mathrm{card}(\mathbb{N})$. 
\hfill\break
Claim this can be extended to any set $\Set{S}$,
proving $\mathrm{card}(\Set{S}) < \mathrm{card}(2^{\Set{S}}))$,
including $\mathrm{card}(\mathbb{R}) < \mathrm{card}(2^{\mathbb{R}}))$?
Doesn't seem obvious.
\hfill\break
How does all this play with 
the Hilbert hotel paradox?~\cite{wiki:Hilbert_hotel}
(Problem with Hilbert hotel is 
always space for new customer, if everybody moves one room up,
but infinite time required to make that space.

Question $1$ for Cantor:
\hfill\break
Are there any cardinals other than 
$\mathrm{card}(\mathbb{N}$ and 
$\mathrm{card}(\mathbb{R})$?

Question $2$ for Cantor (Continuum hypothesis aka CH):
\hfill\break
Are there any cardinals between
$\mathrm{card}(\mathbb{N}$ and 
$\mathrm{card}(\mathbb{R})$?

Implicit of axiom of choice (AOC) used to 'prove' 
well ordering (WO) of cardinals.
('Cardinals' not clearly defined.
Should it be equivalence classes of sets where
equivalence means 'there exists'/'we can construct'
a $\OneToOne$ mapping.)
\hfill\break
Well ordering used to show 'scale' of infinite cardinals
$\aleph_0, \aleph_1, \ldots$. (Is this countable?)
\hfill\break
Continuum hypothesis: 
$\mathfrak{c} = \mathrm{card}(2^{\mathbb{N}}) 
> \mathrm{card}(\mathbb{R}) = \aleph_1$
(and there are no cardinalities in between).

Zermelo axioms of set theory, including AC.
Start from axioms defining initial universe:
empty set and $\mathbb{N}$.
Then set operations including predicated subsets,
'finessing' the contradiction of the
cardinality of the set of all cardinalities,
set of all sets,
sets of all sets not containing themselves, 
\ldots.
\hfill\break
'\ldots more fuel for criticism \ldots.'
Weyl and Skolem improvements by basing it on 1st order logic.
(But doesn't a first order language require sets,
eg, a set of atomic symbols?)

P.~$60$ $2$nd paragraph (Skolem's paradox):
\hfill\break
What's the difference between $\Set{A} \sim \PowerSet{\Set{A}}$ 
($\mathrm{card}(\Set{A})=\mathrm{card}(\Set{\PowerSet{\Set{A}}})$
 'externally' vs 'internally'?

\textsf{G\"{o}del's incompleteness theorems:}
don't follow completely.

\textsf{Postscript on second-order logic:}
``second-order logic cannot be axiomatized effectively.''
Don't follow much of argument.

\textsf{Elimination of the law of the excluded middle:}
\hfill\break
LEM implies (for Hilbert) completed infinity (?).
\hfill\break
Heyting arithmetic (HA) excludes LEM;
any true statement in PA (Peano arithmetic) 
(and many other arithmetic systems)
can be translated into
a true statement in HA.
\hfill\break
This (somehow) implies that a finite consistency proof
of PA would not eliminate completed infinity.

\textsf{The elusiveness of Canotr's continuum problem:}
\hfill\break
Hilbert's program can't solve the continuum problem.


\textsf{New axioms?}
\hfill\break
Dubiousness of large cardinals.
\hfill\break
Freiling axiom of 
symmetry (AX)~\cite{freiling1986,wiki:Freilings_axiom_of_symmetry}
'disproves' CH.

%-----------------------------------------------------------------
\setcounter{currentlevel}{\value{currentlevel}+1 }
%-----------------------------------------------------------------
\setcounter{currentlevel}{\value{baseSectionLevel}-1 }
\levelstay{SEP}
\label{sec:Constructivism_SEP}
\cite{sep:mathematics_constructive}

Multiple references to ``the informal, though rigorous, 
style of the practising analyst, algebraist, topologist, \ldots''
???

Section 1 has problems:

Discussion of
$ \forall x \in \mathbb{R} (x = 0 \vee x \ne 0)$.

``However, because the computer can handle real numbers 
only by means of finite rational approximations, 
we have the problem of underflow, 
in which a sufficiently small positive number can be misread as 0 
by the computer;\ldots''

\ldots which is nonsense;
real issue is non-termination of test for $x=0$.

Section 2:

BHK interpretation~\cite{wiki:Brouwer_Heyting_Kolmogorov_interpretation}:
\begin{itemize}
  \item A proof of $P\wedge Q$ 
  is a pair $(a,b)$ where $a$ is a proof of $P$ 
  and $b$ is a proof of $Q$.
\item A proof of $P\vee Q$ is a pair $(a,b)$ 
where $a$ is $0$ and 
$b$ is a proof of $P$, 
or $a$ is $1$ 
and $b$ is a proof of $Q$.
\item A proof of $P\Rightarrow Q$ is 
a function $f$ that converts 
a proof of $P$ into a proof of $Q$.
\item A proof of $\exists x\in S:\varphi (x)$ is 
a pair $(a,b)$ where 
$a$ is an element of $S$, 
and $b$ is a proof of $\varphi (a)$.
\item A proof of $\forall x\in S:\varphi (x)$ is 
a function $f$ that converts 
an element (any element?) $a$ of $S$ into a proof of $\varphi (a)$.
\item The formula $\neg P$ is defined as $P\Rightarrow \bot$, 
so a proof of it is a function $f$ that converts 
a proof of $P$ into a proof of $\bot$.
\item There is no proof of $\bot$
(the absurdity, or bottom type 
(nontermination in some programming languages)).
\end{itemize}
What is a 'function' here? 
Other sources use 'algorithm', also without definition.

Analysis examples too detailed and lacking motivation 
to make point clear, fell cut-and-pasted from the middle
of a more coherent presentation:

Bishop: constructive $\mathbb{R}$ 
from Cauchy sequences 
in $\mathbb{Q}$~\cite{bishop1985constructive}.

Bridges and Vita: constructive $\mathbb{R}$ from sequences
of intervals in $\mathbb{Q}$~\cite{Bridger:2019}.

Intermediate value theorem C: 
``If $f$ is a continuous real-valued mapping 
on the closed interval $[0,1]$ such that $f(0)⋅f(1)<0$, 
then there exists $x$ such that $0<x<1$ and $f(x)=0$.

Proof requirement: 
An algorithm which, applied to the function $f$, 
a modulus of continuity for $f$, and the values $f(0)$ and $f(1)$,
computes an object $x$ and shows that $x$ is a real number 
between $0$ and $1$; and
shows that $f(x)=0$.''~\cite{sep:mathematics_constructive}

Not constructive because it requires
``lesser limited principle of omniscience (LLPO)''.

Intermediate value theorem D:
``If $f$ is a continuous real-valued mapping 
on the closed interval $[0,1]$ 
such that $f(0)⋅f(1)<0$, 
then for each $\epsilon>0$ 
there exists $x$ such that $0<x<1$ and $|f(x)|<\epsilon$.
Proof requirement: 
An algorithm which, applied to the function $f$, 
a modulus of continuity for $f$, the values $f(0)$ and $f(1)$, 
and a positive number $\epsilon$,
computes an object $x$ 
and shows that $x$ is a real number between $0$ and $1$; and
shows that $|f(x)|<\epsilon$.''~\cite{sep:mathematics_constructive}

Can be proved constructively.

Stronger intermediate value thm:
``Let $f$ be a continuous real-valued mapping 
on the closed interval $[0,1]$ 
such that $f(0)⋅f(1)<0$. 
Suppose also that $f$ is locally nonzero, 
in the sense that for each $x \in [0,1]$
and each $r>0$, 
there exists $y$ such that $|x−y|<r$ and $f(y) \neq 0$. 
Then there exists $x$ such that $0<x<1$ and $f(x)=0$.''

Section 3: Varieties

Section 3.1: Inuitionist (Brouwer)

``Brouwer was not the clearest expositor of his ideas, 
as is shown by the following quotation:

Mathematics arises when the subject of two-ness, 
which results from the passage of time, 
is abstracted from all special occurrences. 
The remaining empty form [the relation of n to n+1] 
of the common content of all these two-nesses 
becomes the original intuition of mathematics 
and repeated unlimitedly creates new mathematical subjects. 
(quoted in Kline [1972], 1199–2000)''

(Nederlands to Deutsch to English?)

``According to [Brouwer’s] view and reading of history, 
classical logic was abstracted 
from the mathematics of finite sets and their subsets. \ldots 
Forgetful of this limited origin, 
one afterwards mistook that logic 
for something above and prior to all mathematics, 
and finally applied it, without justification, 
to the mathematics of infinite sets. 
This is the Fall and original sin of set theory, 
for which it is justly punished by the antinomies. 
It is not that such contradictions showed up that is surprising,
 but that they showed up at such a late stage of the game. 
 (Weyl [1946])''
 
Weyl may have worked out a description of
$\mathbb{R}$ without (uncountable) infinity or
axiom of choice?

%-----------------------------------------------------------------
\setcounter{currentlevel}{\value{baseSectionLevel}-1 }
\levelstay{IEP}
\label{sec:Constructivism_IEP}
\cite{iep:Constructive_mathematics}

``An implication ($A \Rightarrow B$) is not equivalent 
to a disjunction ($\lnot A \vee B$), 
and neither are equivalent to a negated conjunction 
($\not (A \wedge \lnot B$)). 
In practice, constructive mathematics may be viewed 
as mathematics done using intuitionistic 
logic.''\cite{iep:Constructive_mathematics}

Hilbert vs Brouwer circa 1900: 
non-constructive existence thms via contradiction.

Usual example: Prove $\exists a,b \notin \mathbb{Q}$
such that $a^b \in \mathbb{Q}$. \hfill\break
Non-constructive: Use law of excluded middle.
If $\sqrt{2}^{\sqrt{2}} \in \mathbb{Q}$, done.
If not, consider ${\sqrt{2}^{\sqrt{2}}}^{\sqrt{2}}$.
\hfill\break
Constructive: consider $a=\sqrt{2}$ and $b=\log_2(9)$.

BHK interpretation of intuitionistic 
logic\cite{wiki:Brouwer_Heyting_Kolmogorov_interpretation}.

\setcounter{currentlevel}{\value{currentlevel}-1 }
\levelstay{Martin-L\"{o}f}
\label{sec:Martin_Lof_IEP}

``However he soon revisited foundations in a different way, 
which harks back to Russell's type theory, 
albeit using a more constructive and less logicist approach. 
The basic idea of Martin-Löf's theory of types (Martin-Löf, 1975) 
is
that mathematical objects all come as types, 
and are always given in terms of a
type (for example, one such type is that of functions 
from natural numbers to
natural numbers), due to an intuitive understanding 
that we have of the notion
of the given type.

The central distinction Martin-Löf makes is that 
between proof and derivation.
Derivations convince us of the truth of a statement, 
whereas a proof contains the data necessary for computational
(that is, mechanical) verification of a proposition. 
Thus what one finds in standard mathematical textbooks are
derivations; 
a proof is a kind of realizability (c.f. Kleene, 1945) and links
mathematics to implementation (at least implicitly).

The theory is very reminiscent of a kind of cumulative hierarchy. 
Propositions can be represented as types 
(a proposition's type is the type of its proofs),
and to each type one may associate a proposition 
(that the associated type is not empty). 
One then builds further types by construction on already existing
types.''

%-----------------------------------------------------------------
\setcounter{currentlevel}{\value{baseSectionLevel}-1 }
\levelstay{Wikipedia}
\label{sec:Constructivism_Wikipedia}

\cite{wiki:Constructivism_philosophy_of_mathematics}

Usually drop law of excluded middle 
($P \vee \lnot P$).\cite{wiki:Law_of_excluded_middle}
It is possible that both $P$ and $\lnot P$ are not provable,
so $P \vee \lnot P$ is not provable.

But keep law of non-contradiction
($\lnot ( P \wedge (\lnot P))$).~\cite{wiki:Law_of_noncontradiction}
In other words, it is impossible to prove both.

Witnesses: Proving $\exists _{x\in \Set{X}}P(x)$ requires
constructing a specific $a \in \Set{X}$ such that $P(a)$ is true; 
$a$ is a \textit{witness} for $\exists _{x\in \Set{X}}P(x)$.

People\cite{wiki:Constructivism_philosophy_of_mathematics}:
\begin{description}
\item[Leopold Kronecker] old constructivism, semi-intuitionism.
\item[L. E. J. Brouwer] forefather of intuitionism.
\item[A. A. Markov ] forefather of Russian school of constructivism.
\item[Arend Heyting] formalized intuitionistic logic and theories.
\item[Per Martin-Löf] founder of constructive type theories.
\item[Errett Bishop] promoted a version of constructivism 
claimed to be consistent with classical mathematics.
\item[Paul Lorenzen] developed constructive analysis.
\end{description}

%-----------------------------------------------------------------
\setcounter{currentlevel}{\value{currentlevel}-1}
\levelstay{Constructive analysis}
\label{Constructive analysis}

In constructive 
analysis~\cite{wiki:Constructive_analysis,
Bridger:2019,henle:2012:numbers},
varieties of constructionism are divided by
what sort of countable $\rightarrow$ countable functions can be 
constructed?~\cite{wiki:Constructivism_philosophy_of_mathematics}
\begin{description}
\item[Free choice sequences] 
\autoref{sec:Choice_sequence}\cite{wiki:Choice_sequence}.
\item[Algorithms/Computable functions] 
\autoref{sec:Computable_function}~\cite{wiki:Computable_function}
 results in computable numbers (reals?).
\end{description}

\setcounter{currentlevel}{\value{currentlevel}-1}
\levelstay{Choice sequences}
\label{sec:Choice_sequences}

\begin{description}
\item[Lawlike] determined completely (somehow?),
eg, $\mathbb{N}$ or functions $\mathbb{N} \mapsto \mathbb{N}$,
functions from the first $k$ elements of the sequence to 
$\mathbb{N}$.
(How is this different from a computable function?)
\item[Lawless] only a finite prefix determined at any point.
Example is rolls of a die. (Very dubious\ldots.)
\end{description}

Axioms 
(Are these only for lawless sequences?
They don't seem 'constructive' to me\ldots):
\begin{description}
\item[open data] Let $\alpha \in n$ mean that the first items
in the choice sequence $\alpha$ match the finite sequence $n$.
Then for any predicate $A$,
$A(\alpha )$ implies  $\exists n$ such that
$\alpha \in n$ and $\forall \beta \in n$ we have $A(\beta )$.
In other words, the truth of any predicate on a sequence can be
determined from a finite prefix.
\item[density] $\forall n\,\exists \alpha [\alpha \in n]$.
There exists some sequence that has any given finite prefix.
\end{description}


\levelstay{Computable function}
\label{sec:Computable_function}
\cite{wiki:Computable_function}

%-----------------------------------------------------------------
\levelstay{Cardinality of $\mathbb{R}$}
\label{sec:Cardinality_of_R}

Issues with Cantor's diagonal 
argument~\cite{wiki:Cantors_diagonal_argument}.

(Starts with a given enumeration of (say) binary expansions of 
elements of $\mathbb{R}$, then \textit{constructs} a missing
element. But where does the enumeration come from?
Seems like a proof by contradiction\ldots?)

%-----------------------------------------------------------------
\levelstay{Axiom of choice}
\label{sec:Axiom_of_choice}

One interpretation of 'constructive' is
'provable in $\textsf{ZF}$ set theory without the axiom of 
choice.'~\cite{iep:Set_theory,wiki:Axiom_of_choice}
(See \autoref{sec:Zermelo-Fraenkel-set-theory} for \textsf{ZF}.)
But \textsf{ZF} may be considered 'not constructive'
itself.

In some set theories, the axiom of choice implies
the law of excluded middle; in others it doesn't.
When it does, constructionists use weaker versions
that don't imply $P \, \wedge \, \lnot P$.
So some constructionists restrict themselve

 %-----------------------------------------------------------------
\levelstay{Measure theory}
\label{sec:Measure_theory}

Bishop~\cite{bishop1967foundations,bishop1972constructive,bishop1985constructive}

 %-----------------------------------------------------------------
\setcounter{currentlevel}{\value{currentlevel}+1}
 