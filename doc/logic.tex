%-----------------------------------------------------------------
%-----------------------------------------------------------------
%-----------------------------------------------------------------
\setcounter{currentlevel}{\value{baseSectionLevel}}
\levelstay{Logic}
\label{sec:Logic}

\cite{wiki:Logic}

\setcounter{currentlevel}{\value{currentlevel}-1}

%-----------------------------------------------------------------
\levelstay{Zeroth order logic}
\label{sec:Zeroth_order_logic}

Classical Zeroth-order (propositional) 
logic~\cite{iep:Propositional_logic,
wiki:Propositional_calculus,
wiki:Zeroth_order_logic}

Subtle difference sometimes between propositional
and zeroth-order logic 
(=binary truth functional propositional logic).

Formal system:
language with atomic symbols
and logical operators defining well-formed formulas,
and inference rules that a set of axiom formulas 
and return deduced formulas.
\begin{description}
\item[Atoms]  $A, B, \ldots P, Q, \ldots$, 
that may be asisgned values \textsf{true} or \textsf{false}.

\item[Operators] $\lnot P$, $P \wedge Q$, $P \vee Q$, 
$P \Rightarrow Q$, $P \Leftrightarrow Q$. 
Minimal set is $\lnot$ and 
any $2$ of  $\wedge, \vee, \Rightarrow$;
sufficient to define $2$ remaining operators.

\item[Propositions] (Well formed formulas)
Recursively: atoms and propostions combined with operators,
including parens for grouping: $(P \wedge Q) \Rightarrow R$.

\item[Inference rules] ${P \Rightarrow Q), P} \vdash Q$:
Takes a set of propositions (with truth assignments?)
and returns another proposition.
\end{description}

A variety of such formal languages.

Syntatic entailment: formula derived from set of axioms 
using inference rules ina fininte number of steps.

Semantic entailment: formula evaluates to \textsf{true}
under all possible truth assignements to variables
in axioms.

Issues are 
\begin{description}
\item[Soundness]
All syntactically entailed formulas
are semantically entailed, that is,
a sequence of inferences will never lead to a contradiction.

\item[Completeness] 
all semantically entailed formaulas 
are syntactically entailed, that is,
any statement which is consistent with the axioms under all 
possible truth assignements is derivable (somehow) using the 
inference rules.
\end{description}soundness 

%-----------------------------------------------------------------
\levelstay{Intuitionist zeroth order logic}
\label{sec:Intuitionist_zeroth_order_logic}

No law of excluded middle~\cite{wiki:Law_of_excluded_middle}

%-----------------------------------------------------------------
\levelstay{Modal logic}
\label{sec:Modal_logic}

\cite{wiki:Modal_logic}

Possibility vs necessity.

Epistemic: state of kn owledge

Temporal 

%-----------------------------------------------------------------
\levelstay{Paraconsistent logic}
\label{sec:Paraconsistent_logic}

\cite{wiki:Paraconsistent_logic}

Get rid of "from a contradiction, anything follows".
%-----------------------------------------------------------------
\levelstay{First order logic}
\label{sec:First_order_logic}

First-order (aka predicate) logic.~\cite{wiki:First_order_logic,
sep:logic_firstorder_emergence}

\begin{description}
\item[Truth values] Usually $\{\mathsf{true},\mathsf{false}\}$,
but more than $2$ values is possible.
\item[Domain of discourse] A set $\Set{D}$.
\item[Constants] $A,B,C, \ldots$ names for values in the
domain.
\item[Variables] $a,b,c, \ldots x,y,\dots$ taking on values in the
domain.~\cite{wiki:Free_variables_and_bound_variables}
\item[Equality] combine atoms and variables.
\item[Functions] combine atoms and variables.
\item[Quantifiers]   
$\exists x \in \Set{S}$, $\forall x \in \Set{S}$ 
(where $\Set{S} \subseteq \Set{D}$);
$\exists x$ means $\exists x \in \Set{D}$, 
and
$\forall x$ means $\forall x \in \Set{D}$) 
bind variables in
formulas.~\cite{wiki:Quantifier_logic}
More specialized quantifiers are possible:
$\exists \textrm{ a unique } x \in \Set{S}$
\item[Predicates] combine atoms and variables.
\end{description} 

Differentiate 1st order \textit{language} 
(no domain of discourse, just symbols)
from \textit{interpretation} 
(assignment of terms to elements of $\Set{D}$).

See also models/structures.~\cite{wiki:Model_theory}

(Circularity w.r.t set theory?
Zermelo-Fraenkel uses 1st order logic to define sets,
but 1st order logic is defined using sets\ldots?)

\setcounter{currentlevel}{\value{currentlevel}-1}

\levelstay{First order language}
\label{sec:First_order_language}

Syntactic rules without domain of discourse.

Signature: arities of predicates and functions.

\levelstay{First order model}
\label{sec:First_order_model}

Domain of discourse and evaluation of formulas.

Equality.
(does this belong here? can also be axioms on a theory.)

\levelstay{First order theory}
\label{sec:First_order_theory}

Axioms (and axiom schemas).

Consistency: no contradiction derivable from axioms.

Completeness: any formula can be proven 
\textsf{true} or \textsf{false} 
(axioms plus formula permit derivation of 
\textsf{true} or \textsf{false} but not both).

\levelstay{First order deductive systems}
\label{sec:First_order_deductive_systems}

Rules of inference.

\levelstay{Many sorted first order models}
\label{sec:Many_sorted_first_order_models}

Domain of discourse has multiple sets;
variables 'typed' as having values in some particular set.

(Is this really different from
axiom restricting variable to subset of domain?()


\setcounter{currentlevel}{\value{currentlevel}+1}
%-----------------------------------------------------------------
\levelstay{Second order logic}
\label{sec:Second_order_logic}

Second-order logic~\cite{wiki:Second_order_logic,
wiki:Second_order_propositional_logic}

%-----------------------------------------------------------------
\levelstay{Higher order logic}
\label{sec:Higher_order_logic}

Higher-order logic~\cite{wiki:Higher_order_logic}


 